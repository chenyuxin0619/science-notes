%% Generated by Sphinx.
\def\sphinxdocclass{jupyterBook}
\documentclass[letterpaper,10pt,english]{jupyterBook}
\ifdefined\pdfpxdimen
   \let\sphinxpxdimen\pdfpxdimen\else\newdimen\sphinxpxdimen
\fi \sphinxpxdimen=.75bp\relax
\ifdefined\pdfimageresolution
    \pdfimageresolution= \numexpr \dimexpr1in\relax/\sphinxpxdimen\relax
\fi
%% let collapsible pdf bookmarks panel have high depth per default
\PassOptionsToPackage{bookmarksdepth=5}{hyperref}
%% turn off hyperref patch of \index as sphinx.xdy xindy module takes care of
%% suitable \hyperpage mark-up, working around hyperref-xindy incompatibility
\PassOptionsToPackage{hyperindex=false}{hyperref}
%% memoir class requires extra handling
\makeatletter\@ifclassloaded{memoir}
{\ifdefined\memhyperindexfalse\memhyperindexfalse\fi}{}\makeatother

\PassOptionsToPackage{warn}{textcomp}

\catcode`^^^^00a0\active\protected\def^^^^00a0{\leavevmode\nobreak\ }
\usepackage{cmap}
\usepackage{fontspec}
\defaultfontfeatures[\rmfamily,\sffamily,\ttfamily]{}
\usepackage{amsmath,amssymb,amstext}
\usepackage{polyglossia}
\setmainlanguage{english}



\setmainfont{FreeSerif}[
  Extension      = .otf,
  UprightFont    = *,
  ItalicFont     = *Italic,
  BoldFont       = *Bold,
  BoldItalicFont = *BoldItalic
]
\setsansfont{FreeSans}[
  Extension      = .otf,
  UprightFont    = *,
  ItalicFont     = *Oblique,
  BoldFont       = *Bold,
  BoldItalicFont = *BoldOblique,
]
\setmonofont{FreeMono}[
  Extension      = .otf,
  UprightFont    = *,
  ItalicFont     = *Oblique,
  BoldFont       = *Bold,
  BoldItalicFont = *BoldOblique,
]



\usepackage[Bjarne]{fncychap}
\usepackage[,numfigreset=1,mathnumfig]{sphinx}

\fvset{fontsize=\small}
\usepackage{geometry}


% Include hyperref last.
\usepackage{hyperref}
% Fix anchor placement for figures with captions.
\usepackage{hypcap}% it must be loaded after hyperref.
% Set up styles of URL: it should be placed after hyperref.
\urlstyle{same}

\addto\captionsenglish{\renewcommand{\contentsname}{Q2 2023}}

\usepackage{sphinxmessages}



        % Start of preamble defined in sphinx-jupyterbook-latex %
         \usepackage[Latin,Greek]{ucharclasses}
        \usepackage{unicode-math}
        % fixing title of the toc
        \addto\captionsenglish{\renewcommand{\contentsname}{Contents}}
        \hypersetup{
            pdfencoding=auto,
            psdextra
        }
        % End of preamble defined in sphinx-jupyterbook-latex %
        

\title{DSF Science Notes}
\date{Sep 21, 2023}
\release{}
\author{DLT Science Foundation}
\newcommand{\sphinxlogo}{\vbox{}}
\renewcommand{\releasename}{}
\makeindex
\begin{document}

\pagestyle{empty}
\sphinxmaketitle
\pagestyle{plain}
\sphinxtableofcontents
\pagestyle{normal}
\phantomsection\label{\detokenize{intro::doc}}


\sphinxAtStartPar
DSF Science Notes consists of high\sphinxhyphen{}quality technical research content focused on blockchain technology. The topics covered fall in these three major categories, namely;
\begin{enumerate}
\sphinxsetlistlabels{\arabic}{enumi}{enumii}{}{.}%
\item {} 
\sphinxAtStartPar
\sphinxstylestrong{Academic Insights}: This category will feature science notes that highlight academic research findings related to blockchain technology, cryptography, distributed ledger technology (DLT), and other relevant topics. Science notes in this category with include a comprehensive overview of recent research papers in a subject\sphinxhyphen{}area. and will be findings\sphinxhyphen{}focused.

\item {} 
\sphinxAtStartPar
\sphinxstylestrong{Industry Perspectives}: This category will include science notes that provide findings and insights focused on the industry applications of blockchain\sphinxhyphen{}related subject matters.

\item {} 
\sphinxAtStartPar
\sphinxstylestrong{Innovation \& Ideation}: This category will focus on highlighting innovative ideas, concepts, and use cases related to blockchain technology. It will feature blog posts that explore potential applications of blockchain in various industries, such as finance, supply chain, healthcare, and more.

\end{enumerate}

\sphinxAtStartPar
\DUrole{xref,download,myst}{Download Science\sphinxhyphen{}Notes as a pdf}

\begin{sphinxadmonition}{note}{DSF Science Notes Editorial Board}

\sphinxAtStartPar
\sphinxstylestrong{Dr Jiahua Xu}, DSF Head of Science
\sphinxstylestrong{Dr Carlo Campajola}, DSF Senior Research Fellow
\end{sphinxadmonition}

\sphinxstepscope


\part{Q2 2023}

\sphinxstepscope


\chapter{Academic Insights}
\label{\detokenize{STRUCTURE/academic:academic-insights}}\label{\detokenize{STRUCTURE/academic:nb-gallery}}\label{\detokenize{STRUCTURE/academic::doc}}
\begin{sphinxuseclass}{sd-container-fluid}
\begin{sphinxuseclass}{sd-sphinx-override}
\begin{sphinxuseclass}{sd-mb-4}
\begin{sphinxuseclass}{sd-row}
\begin{sphinxuseclass}{sd-row-cols-1}
\begin{sphinxuseclass}{sd-row-cols-xs-1}
\begin{sphinxuseclass}{sd-row-cols-sm-2}
\begin{sphinxuseclass}{sd-row-cols-md-2}
\begin{sphinxuseclass}{sd-row-cols-lg-2}
\begin{sphinxuseclass}{sd-g-2}
\begin{sphinxuseclass}{sd-g-xs-2}
\begin{sphinxuseclass}{sd-g-sm-2}
\begin{sphinxuseclass}{sd-g-md-2}
\begin{sphinxuseclass}{sd-g-lg-2}
\begin{sphinxuseclass}{sd-col}
\begin{sphinxuseclass}{sd-d-flex-row}
\begin{sphinxuseclass}{sd-mt-3}
\begin{sphinxuseclass}{sd-mb-0}
\begin{sphinxuseclass}{sd-ml-0}
\begin{sphinxuseclass}{sd-mr-0}
\begin{sphinxuseclass}{sd-card}
\begin{sphinxuseclass}{sd-sphinx-override}
\begin{sphinxuseclass}{sd-w-100}
\begin{sphinxuseclass}{sd-shadow-md}
\begin{sphinxuseclass}{sd-card-hover}
\begin{sphinxuseclass}{sd-text-center}
\begin{sphinxuseclass}{sd-card-body}
\begin{sphinxuseclass}{sd-card-title}
\begin{sphinxuseclass}{sd-font-weight-bold}Governance In DeFi
\end{sphinxuseclass}
\end{sphinxuseclass}




\end{sphinxuseclass}\sphinxhref{https://dlt-science.github.io/science-notes/GOV/gov.html}{}
\end{sphinxuseclass}
\end{sphinxuseclass}
\end{sphinxuseclass}
\end{sphinxuseclass}
\end{sphinxuseclass}
\end{sphinxuseclass}
\end{sphinxuseclass}
\end{sphinxuseclass}
\end{sphinxuseclass}
\end{sphinxuseclass}
\end{sphinxuseclass}
\end{sphinxuseclass}
\begin{sphinxuseclass}{sd-col}
\begin{sphinxuseclass}{sd-d-flex-row}
\begin{sphinxuseclass}{sd-mt-3}
\begin{sphinxuseclass}{sd-mb-0}
\begin{sphinxuseclass}{sd-ml-0}
\begin{sphinxuseclass}{sd-mr-0}
\begin{sphinxuseclass}{sd-card}
\begin{sphinxuseclass}{sd-sphinx-override}
\begin{sphinxuseclass}{sd-w-100}
\begin{sphinxuseclass}{sd-shadow-md}
\begin{sphinxuseclass}{sd-card-hover}
\begin{sphinxuseclass}{sd-text-center}
\begin{sphinxuseclass}{sd-card-body}
\begin{sphinxuseclass}{sd-card-title}
\begin{sphinxuseclass}{sd-font-weight-bold}Blockchain Bridge Security
\end{sphinxuseclass}
\end{sphinxuseclass}




\end{sphinxuseclass}\sphinxhref{https://dlt-science.github.io/science-notes/BBSecurity/bbsecurity.html}{}
\end{sphinxuseclass}
\end{sphinxuseclass}
\end{sphinxuseclass}
\end{sphinxuseclass}
\end{sphinxuseclass}
\end{sphinxuseclass}
\end{sphinxuseclass}
\end{sphinxuseclass}
\end{sphinxuseclass}
\end{sphinxuseclass}
\end{sphinxuseclass}
\end{sphinxuseclass}
\end{sphinxuseclass}
\end{sphinxuseclass}
\end{sphinxuseclass}
\end{sphinxuseclass}
\end{sphinxuseclass}
\end{sphinxuseclass}
\end{sphinxuseclass}
\end{sphinxuseclass}
\end{sphinxuseclass}
\end{sphinxuseclass}
\end{sphinxuseclass}
\end{sphinxuseclass}
\end{sphinxuseclass}
\end{sphinxuseclass}
\sphinxstepscope


\section{Governance in DeFi}
\label{\detokenize{GOV/gov:governance-in-defi}}\label{\detokenize{GOV/gov::doc}}


\sphinxAtStartPar
\sphinxstylestrong{Academic Insight}

\begin{sphinxadmonition}{note}{Key Insights}
\begin{itemize}
\item {} 
\sphinxAtStartPar
The voting power in DeFi protocols becomes increasingly concentrated among a percentage of token holders over time in decentralised exchanges, lending protocols and yield aggregators.

\item {} 
\sphinxAtStartPar
The paramount wallet addresses ranking within the top 5, 100, and 1000, exercise predominant influence over the voting power in the Balancer, Compound, Uniswap, and Yearn Finance protocols, with Compound displaying the least evidence of decentralisation.

\item {} 
\sphinxAtStartPar
The most significant governance challenges identified by DeFi users are voter collusion, low participation rates, and voter apathy.

\item {} 
\sphinxAtStartPar
To address vulnerabilities in DeFi governance, a novel voting mechanism resistant to sybil attacks called bond voting has been proposed.

\item {} 
\sphinxAtStartPar
To enhance the manual parameter section, an AI\sphinxhyphen{}enabled adjustment solution has been demonstrated to automate governance mechanisms.

\end{itemize}
\end{sphinxadmonition}


\subsection{Introduction}
\label{\detokenize{GOV/gov:introduction}}
\sphinxAtStartPar
Decentralised finance (DeFi) has emerged as a potential substitute for traditional financial institutions, offering peer\sphinxhyphen{}to\sphinxhyphen{}peer transactions and a diverse range of services that democratise finance by enabling users to participate in protocol governance. However, several studies have suggested that the current governance mechanisms require improvements. This article provides an overview of findings associated with DeFi governance.


\subsection{Centralisation of Governance in DeFi Protocols}
\label{\detokenize{GOV/gov:centralisation-of-governance-in-defi-protocols}}
\begin{sphinxShadowBox}
\sphinxstylesidebartitle{\sphinxstylestrong{Lending Protocols}}

\sphinxAtStartPar
Lending Protocols are DeFi applications built on top of blockchain technology that allow users to lend and borrow cryptocurrency assets without the need for intermediaries such as banks or traditional financial institutions.
\end{sphinxShadowBox}

\begin{sphinxShadowBox}
\sphinxstylesidebartitle{\sphinxstylestrong{Decentralized Exchanges}}

\sphinxAtStartPar
Decentralized Exchanges (DeXs) are peer\sphinxhyphen{}to\sphinxhyphen{}peer trading platforms built on top of a blockchain that enable the direct exchange of cryptocurrency assets without the need for a central authority or intermediary.
\end{sphinxShadowBox}

\begin{sphinxShadowBox}
\sphinxstylesidebartitle{\sphinxstylestrong{Yield Aggregator}}

\sphinxAtStartPar
Yield Aggregator are DeFi applications that automate the process of seeking out the best yield opportunities for cryptocurrency assets, and provide users with a way to optimize their returns on investment.
\end{sphinxShadowBox}

\sphinxAtStartPar
Centralisation in DeFi has become a growing concern among researchers with several studies identifying a significant level of centrality in the governance mechanisms of DeFi protocol. Barbereau et al., {[}\hyperlink{cite.GOV/gov:id31}{BSP+22a}{]} found that the decentralisation of voting is significantly low with a majority of the voting power concentrated among a percentage of governance token holders. As evidenced by their findings, there was a significant degree of centrality, in lending protocols, decentralisd exchanges and yield aggregators. This research work employed case studies to comprehend the governance mechanisms of these protocols.

\sphinxAtStartPar
Similarly, result by Jensen et al. {[}\hyperlink{cite.GOV/gov:id30}{JvWR21}{]} demonstrate centrality in voting power with the protocols top 5, top 100, and top 1000 wallet addresses controlling majority of the voting power in Balancer, Compound, Uniswap and Yearn Finance protocols. In this study, the token holdings and users’ wallets of protocols were analysed; Compound displayed the most evidence of centrality and Uniswap the least with the top 5 wallet addresses accounting for 42.1\% and 12.05\%, respectively.

\sphinxAtStartPar
Barbereau et al. {[}\hyperlink{cite.GOV/gov:id28}{BSP+22b}{]} ascertained that DeFi protocols become more centralised over time. In this longitudinal study, voting patterns demonstrated changes in the power dynamics as time progressed. The tendency for this centralisation of DeFi protocols is shown in {[}\hyperref[\detokenize{GOV/gov:gov-evolution}]{Fig.\@ \ref{\detokenize{GOV/gov:gov-evolution}}}{]}. Furthermore, in analysing the governance structures of DeFi protocols, Stroponiati et al. {[}\hyperlink{cite.GOV/gov:id27}{S+}{]} ascribed reward\sphinxhyphen{}based economic incentives as the significant cause behind the development of centralised structures.

\begin{figure}[htbp]
\centering
\capstart

\noindent\sphinxincludegraphics[width=750\sphinxpxdimen,height=260\sphinxpxdimen]{{Govern.drawio}.png}
\caption{The Tendency for Centralisation in DeFi Governance.}\label{\detokenize{GOV/gov:gov-evolution}}\end{figure}


\subsection{Challenges \& Vulnerability In DeFi Governance}
\label{\detokenize{GOV/gov:challenges-vulnerability-in-defi-governance}}
\begin{sphinxShadowBox}
\sphinxstylesidebartitle{\sphinxstylestrong{Voter Collusion}}

\sphinxAtStartPar
Voter Collusion refers to a situation where a group of voters collude together to manipulate the outcome of a voting process in their favor, typically by coordinating their votes to create a super majority.
\end{sphinxShadowBox}

\begin{sphinxShadowBox}
\sphinxstylesidebartitle{\sphinxstylestrong{Voter Apathy}}

\sphinxAtStartPar
Voter Apathy refers to a situation where token holders or members of the organisation do not actively participate in the voting process due to a lack of interest
\end{sphinxShadowBox}

\begin{sphinxShadowBox}
\sphinxstylesidebartitle{\sphinxstylestrong{Sybil Attack}}

\sphinxAtStartPar
Sybil attacks occur when an attacker generates multiple false identities to gain significant network control, thereby allocating more votes than expected.
\end{sphinxShadowBox}

\sphinxAtStartPar
In investigating governance challenges, Ekal et al., {[}\hyperlink{cite.GOV/gov:id29}{EAw22}{]} identified voter collusion, low participation rates, and voter apathy as the most significant challenges. This empirical investigation utilised an interview survey approach to collect data from protocol users. Furthermore, to address voter concentration vulnerabilities, Mohan et al. {[}\hyperlink{cite.GOV/gov:id26}{MKB22}{]} proposed a novel voting mechanism called bond voting which is resistant to sybil attacks. The bond voting mechanism issues ‘voting bonds’ to voters, which essentially requires a commitment to stake an amount of tokens, for a time period to gain voting power. Therefore, by combining this time commitment with weighed voting with a time commitment, sybil attacks are more difficult. Quadratic voting, another solution to voting concentration, allows participants to convey both their preferences and the intensity of those preferences, however, the drawback of this mechanism is its vulnerability to sybil attacks, voter collusion and voter fraud {[}\hyperlink{cite.GOV/gov:id36}{KL22}{]}.


\subsection{AI\sphinxhyphen{}enabled On\sphinxhyphen{}chain Governance}
\label{\detokenize{GOV/gov:ai-enabled-on-chain-governance}}
\sphinxAtStartPar
To enhance and automate governance mechanisms, Xu et al., {[}\hyperlink{cite.GOV/gov:id38}{XPFL23}{]} demonstrated an AI\sphinxhyphen{}enabled parameter adjustment solution which is more efficient than current implementations. Specifically, the study employed Deep Q\sphinxhyphen{}network (DQN) reinforcement learning to investigate for automated parameter selection in a DeFi environment. Although a lending protocol was employed in the study, the model’s application can extend to other categories of DeFi protocols as well. In investigating DAOs, Nabben {[}\hyperlink{cite.GOV/gov:id25}{Nab23}{]} observes that GitcoinDAO also employs algorithmic governance in various organisational components such as monitoring the compliance with rules of the organisation.


\subsection{Conclusion}
\label{\detokenize{GOV/gov:conclusion}}
\sphinxAtStartPar
The vision of DeFi is to foster a democratic process of governance and sustain high levels of decentralisation. However, recent studies have highlighted significant centrality in DeFi governance mechanisms, indicating the need for improvements in the existing governance models. The studies analysed in this article have revealed that the majority of the voting power in several protocols is concentrated among the top token holders, with evidence of increasing centralisation over time. Moreover, DeFi has been found to face challenges in the voting and governance process. In view of some of these challenges, researchers have proposed novel solutions such as a bond voting and an AI\sphinxhyphen{}enabled parameter\sphinxhyphen{}selection solution to improve the current mechanisms. Given the importance of decentralisation in the underlying philosophy of DeFi, proposing more solutions to governance challenges is crucial for creating a more inclusive and democratic financial ecosystem. Therefore, continued research and development will certainly be required.




\subsection{References}
\label{\detokenize{GOV/gov:references}}
\sphinxstepscope


\section{Blockchain Bridge Security}
\label{\detokenize{BBSecurity/bbsecurity:blockchain-bridge-security}}\label{\detokenize{BBSecurity/bbsecurity::doc}}


\sphinxAtStartPar
\sphinxstylestrong{Academic Insight}

\begin{sphinxadmonition}{note}{Key Insights}
\begin{itemize}
\item {} 
\sphinxAtStartPar
To mitigate security risks, a cross\sphinxhyphen{}chain bridge that leverages zk\sphinxhyphen{}SNARK technology has been proposed. This provides a secure, trustless cross\sphinxhyphen{}chain bridge, marking the first implementation of Zero\sphinxhyphen{}Knowledge Proofs (ZKP) in a decentralised trustless bridge system.

\item {} 
\sphinxAtStartPar
To facilitate secure cross\sphinxhyphen{}chain interoperability, a Hash time\sphinxhyphen{}lock scheme that does not rely on external trust ensuring transaction security is introduced.

\item {} 
\sphinxAtStartPar
To mitigate token transfer risks, a series of protocols called TrustBoost using smart contracts to achieve a consensus on top of consensus mechanism is proposed.

\item {} 
\sphinxAtStartPar
In a bid to boost interoperability, a groundbreaking framework has been proposed that not only mitigates security risks inherent in cross\sphinxhyphen{}blockchain technology but also simplifies the process of identifying key assumptions and characteristics.

\end{itemize}
\end{sphinxadmonition}


\subsection{Introduction}
\label{\detokenize{BBSecurity/bbsecurity:introduction}}
\sphinxAtStartPar
Blockchain technology has been lauded for its potential to disrupt various industries, given its unique properties such as decentralisation, transparency, and security. One recent advancement in this area is the development of blockchain bridges, which enable interoperability among different blockchains. Bridges facilitate communication between two blockchain ecosystems through the transfer of assets and information. However, as with any innovative technology, these bridges pose new security challenges. In this science note, we delve into the current academic landscape surrounding the security of blockchain bridges and summarise the recent research findings.

\begin{figure}[htbp]
\centering
\capstart

\noindent\sphinxincludegraphics[width=780\sphinxpxdimen,height=456\sphinxpxdimen]{{BSecurity4.drawio}.png}
\caption{Communication through a Blockchain Bridge.}\label{\detokenize{BBSecurity/bbsecurity:bridge-security}}\end{figure}


\subsection{Interoperability and Security Challenges}
\label{\detokenize{BBSecurity/bbsecurity:interoperability-and-security-challenges}}
\begin{sphinxShadowBox}
\sphinxstylesidebartitle{\sphinxstylestrong{Zero\sphinxhyphen{}Knowledge Proofs}}

\sphinxAtStartPar
A zero\sphinxhyphen{}knowledge proof (ZKP) is a cryptographic technique that enables one party, the prover, to convince another party, the verifier, of the validity of a statement or the possession of a secret without revealing any additional information about the underlying secret or data.
\end{sphinxShadowBox}

\begin{sphinxShadowBox}
\sphinxstylesidebartitle{\sphinxstylestrong{zk\sphinxhyphen{}SNARK}}

\sphinxAtStartPar
Zk\sphinxhyphen{}SNARK is an acronym that stands for “Zero\sphinxhyphen{}Knowledge Succinct Non\sphinxhyphen{}Interactive Argument of Knowledge”. A zk\sphinxhyphen{}SNARK is a cryptographic proof that allows one party to prove it possesses certain information without revealing that information.
\end{sphinxShadowBox}

\sphinxAtStartPar
Interoperability in blockchain environments brings forth a series of unique security challenges. Trustless, interoperable, cryptocurrency\sphinxhyphen{}backed assets can be subjected to various threats. In April 2022, attackers were able to obtain five of the nine validator keys, through which they stole 624 million USD by exploiting the Ronin bridge, making it the largest attack in the history of DeFi {[}\hyperlink{cite.BBSecurity/bbsecurity:id101}{KY22}{]}. According to blockchain analytics firm Chainalysis, until August 2022 recurring attacks against bridges have cost users around 1.4 billion USD {[}\hyperlink{cite.BBSecurity/bbsecurity:id100}{Bro22}{]}. In 2022 attacks on bridges accounted for 69\% of total funds stolen {[}\hyperlink{cite.BBSecurity/bbsecurity:id88}{Cha22}{]}.

\sphinxAtStartPar
This necessitates the development of novel security models and protocols to protect against potential attack vectors arising from cross\sphinxhyphen{}chain communication and is particularly true for blockchain bridges that need to uphold the integrity and security of transactions across disparate networks. Most existing solutions rely on the trust assumptions of committees, which lowers security significantly.

\sphinxAtStartPar
Xie et al. {[}\hyperlink{cite.BBSecurity/bbsecurity:id97}{XZC+22}{]} proposed a solution by introducing zkBridge, an efficient cross\sphinxhyphen{}chain bridge that guarantees strong security without external trust assumptions. The main idea is to leverage zk\sphinxhyphen{}SNARK, which are succinct non\sphinxhyphen{}interactive proofs (arguments) of knowledge as a result security is ensured without relying on a committee. zkBridge uses the zk\sphinxhyphen{}SNARK protocol to achieve both reasonable proof generation times and on\sphinxhyphen{}chain verification costs. zkBridge is trustless as it does not require extra assumptions other than those of blockchains and underlying cryptographic protocols. It is the first to use Zero\sphinxhyphen{}Knowledge Proofs (ZKP) to enable a decentralised trustless bridge.

\sphinxAtStartPar
Pillai et al. {[}\hyperlink{cite.BBSecurity/bbsecurity:id99}{PBHouM22}{]} proposed a novel cross\sphinxhyphen{}blockchain integration framework designed to guide the integration of cross\sphinxhyphen{}blockchain technology. The framework aids in identifying crucial assumptions and characteristics, mitigating security risks, enhancing the decision\sphinxhyphen{}making process, and minimising design mistakes and performance issues. It recognises the integration system as the fundamental unit of cross\sphinxhyphen{}blockchain technology, providing comprehensive analysis and addressing security concerns. Moreover, the framework supports businesses in designing and integrating various blockchain applications, while enabling a more accurate evaluation of security assumptions. Thus, it paves the way for effective interoperability among multiple blockchains.


\subsection{The Role of Cryptography in Blockchain Bridge Security}
\label{\detokenize{BBSecurity/bbsecurity:the-role-of-cryptography-in-blockchain-bridge-security}}
\begin{sphinxShadowBox}
\sphinxstylesidebartitle{\sphinxstylestrong{Sidechain}}

\sphinxAtStartPar
A sidechain is a blockchain that communicates with other blockchains via a two\sphinxhyphen{}way peg. It stems from the main blockchain and runs in parallel to it.
\end{sphinxShadowBox}

\begin{sphinxShadowBox}
\sphinxstylesidebartitle{\sphinxstylestrong{Cryptographic Protocol}}

\sphinxAtStartPar
A cryptographic protocol is an abstract or concrete protocol that performs a security\sphinxhyphen{}related function and applies cryptographic methods, often as sequences of cryptographic primitives. A protocol describes how the algorithms should be used and includes details about data structures and representations, at which point it can be used to implement multiple, interoperable versions of a programme.
\end{sphinxShadowBox}

\sphinxAtStartPar
Securing blockchain bridges is greatly dependent on the strength of the cryptographic techniques deployed. The fundamental study by Kiayias et al. {[}\hyperlink{cite.BBSecurity/bbsecurity:id91}{KRDO17}{]} on proof\sphinxhyphen{}of\sphinxhyphen{}stake blockchain protocols is of significant relevance. They outlined a novel cryptographic mechanism that provides transactional security while ensuring transparency.

\sphinxAtStartPar
To mitigate the reliance on external trust assumptions, Li et al. {[}\hyperlink{cite.BBSecurity/bbsecurity:id92}{LYY+23}{]} in their paper proposed a Hash time\sphinxhyphen{}lock scheme that utilises a hash function and time\sphinxhyphen{}lock features to achieve cross\sphinxhyphen{}chain interoperability. The security of the Hash time\sphinxhyphen{}lock scheme is based on cryptographic hardness assumptions. The asset receiver is forced to determine the collection and produce proof of collection to the payer within the cut\sphinxhyphen{}off time, or the asset will be returned via hash\sphinxhyphen{}locks and blockchain time\sphinxhyphen{}locks. The proof of receipt can be used by the payer to acquire assets of equal value on the recipient’s blockchain or trigger other events. However, this scheme only supports monetary exchange and thus has low scalability.

\sphinxAtStartPar
Li et al. {[}\hyperlink{cite.BBSecurity/bbsecurity:id92}{LYY+23}{]}, identified a high\sphinxhyphen{}security and highly scalable option as the sidechains/relay scheme, which supports the interoperability of multiple objects such as assets and other data, thus having high scalability. In particular, the two\sphinxhyphen{}way peg is a mechanism that allows bidirectional communication between blockchains. An example of a two\sphinxhyphen{}way peg is simplified payment verification (SPV) in Bitcoin. Relays represent a mechanism that enables a blockchain network to authenticate data from other blockchain networks, eliminating the need for external third\sphinxhyphen{}party sources. Operating as a light client on a network, a relay system incorporates a smart contract and records block header information from different networks {[}\hyperlink{cite.BBSecurity/bbsecurity:id93}{F+20}{]}. A trade\sphinxhyphen{}off of the sidechain implementation is that the vulnerability might increase in the main chain or other sidechains if there is a compromised sidechain in the network {[}\hyperlink{cite.BBSecurity/bbsecurity:id94}{Szt15}{]}.

\sphinxAtStartPar
Ding et al. {[}\hyperlink{cite.BBSecurity/bbsecurity:id89}{DDJ+18}{]}, proposed a framework for connecting multiple blockchain networks via an intermediary structure known as the InterChain. The InterChain possesses its own validation nodes, while SubChain networks are linked to this InterChain via gateway nodes.

\sphinxAtStartPar
Hardjono et al. {[}\hyperlink{cite.BBSecurity/bbsecurity:id95}{HLP19}{]}, discussed blockchain interoperability by drawing parallels with the design principles of Internet architecture. Just as the internet uses routers to guide message packets across its network at a mechanical level, they propose the use of gateways to direct messages between different blockchain networks.

\sphinxAtStartPar
Such cryptographic protocols can serve as a guiding light for the development of security measures in the context of blockchain bridges.


\subsection{Scalability and Security}
\label{\detokenize{BBSecurity/bbsecurity:scalability-and-security}}
\sphinxAtStartPar
As important as security is for blockchain bridges, it should not compromise the scalability of the systems. Zamyatin et al. {[}\hyperlink{cite.BBSecurity/bbsecurity:id98}{ZHL+19}{]} discussed the scalability\sphinxhyphen{}security trade\sphinxhyphen{}off in their study on interoperable assets. There is a need for a balance that allows for scalability without jeopardising security. Future research in blockchain bridge security needs to address this delicate balance, ensuring the development of robust and efficient interoperable systems.

\sphinxAtStartPar
Zhang et al. {[}\hyperlink{cite.BBSecurity/bbsecurity:id90}{ZLZ20}{]} introduced a method that facilitates asset exchange between inter\sphinxhyphen{}firm alliance chains and private chains. Users from both the sending and receiving chains authenticate their identities and secure a certificate by interacting with the alliance chain. When a cross\sphinxhyphen{}blockchain transfer request is initiated, the alliance chain validates the ownership of the users over the assets, then proceeds with the asset transfer through a cross\sphinxhyphen{}blockchain interaction process.


\subsection{Maintaining Sovereignty of blockchains}
\label{\detokenize{BBSecurity/bbsecurity:maintaining-sovereignty-of-blockchains}}
\sphinxAtStartPar
Existing solutions to boost the trust using a stronger blockchain, e.g., via checkpointing, require the weaker blockchain to give up sovereignty. Wang et al. {[}\hyperlink{cite.BBSecurity/bbsecurity:id96}{WSK+22}{]} in their paper present a series of protocols known as TrustBoost designed to bolster trust across multiple blockchains without compromising their sovereignty. These protocols function through smart contracts, achieving a “consensus on top of consensus” that avoids changes to the blockchains’ consensus layers. TrustBoost operates by allowing cross\sphinxhyphen{}chain communication via bridges, facilitating the sharing of information across smart contracts on different blockchains. This system maintains its security as long as two\sphinxhyphen{}thirds of the participating blockchains are secure. Furthermore, TrustBoost shows potential in mitigating risks associated with cross\sphinxhyphen{}chain token transfers and exhibits promising prospects for future applications, especially as heterogeneous blockchain networks continue to mature.


\subsection{Conclusion}
\label{\detokenize{BBSecurity/bbsecurity:conclusion}}
\sphinxAtStartPar
Blockchain bridges represent an important evolution in blockchain technology, facilitating crucial interoperability. However, the security aspects of these bridges are complex and multifaceted, requiring rigorous academic and industry attention. The body of research surrounding blockchain security provides critical insights that can help guide the development of secure and efficient blockchain bridges. As this field continues to evolve, a focus on understanding and mitigating security risks while maintaining scalability will be paramount.




\subsection{References}
\label{\detokenize{BBSecurity/bbsecurity:references}}
\sphinxstepscope


\chapter{Industry Perspective}
\label{\detokenize{STRUCTURE/industry:industry-perspective}}\label{\detokenize{STRUCTURE/industry::doc}}
\begin{sphinxuseclass}{sd-container-fluid}
\begin{sphinxuseclass}{sd-sphinx-override}
\begin{sphinxuseclass}{sd-mb-4}
\begin{sphinxuseclass}{sd-row}
\begin{sphinxuseclass}{sd-row-cols-1}
\begin{sphinxuseclass}{sd-row-cols-xs-1}
\begin{sphinxuseclass}{sd-row-cols-sm-2}
\begin{sphinxuseclass}{sd-row-cols-md-2}
\begin{sphinxuseclass}{sd-row-cols-lg-2}
\begin{sphinxuseclass}{sd-g-2}
\begin{sphinxuseclass}{sd-g-xs-2}
\begin{sphinxuseclass}{sd-g-sm-2}
\begin{sphinxuseclass}{sd-g-md-2}
\begin{sphinxuseclass}{sd-g-lg-2}
\begin{sphinxuseclass}{sd-col}
\begin{sphinxuseclass}{sd-d-flex-row}
\begin{sphinxuseclass}{sd-mt-3}
\begin{sphinxuseclass}{sd-mb-0}
\begin{sphinxuseclass}{sd-ml-0}
\begin{sphinxuseclass}{sd-mr-0}
\begin{sphinxuseclass}{sd-card}
\begin{sphinxuseclass}{sd-sphinx-override}
\begin{sphinxuseclass}{sd-w-100}
\begin{sphinxuseclass}{sd-shadow-md}
\begin{sphinxuseclass}{sd-card-hover}
\begin{sphinxuseclass}{sd-text-center}
\begin{sphinxuseclass}{sd-card-body}
\begin{sphinxuseclass}{sd-card-title}
\begin{sphinxuseclass}{sd-font-weight-bold}Mobile Theft Prevention using Blockchain
\end{sphinxuseclass}
\end{sphinxuseclass}




\end{sphinxuseclass}\sphinxhref{https://dlt-science.github.io/science-notes/MTP/mtp.html}{}
\end{sphinxuseclass}
\end{sphinxuseclass}
\end{sphinxuseclass}
\end{sphinxuseclass}
\end{sphinxuseclass}
\end{sphinxuseclass}
\end{sphinxuseclass}
\end{sphinxuseclass}
\end{sphinxuseclass}
\end{sphinxuseclass}
\end{sphinxuseclass}
\end{sphinxuseclass}


\begin{sphinxuseclass}{sd-col}
\begin{sphinxuseclass}{sd-d-flex-row}
\begin{sphinxuseclass}{sd-mt-3}
\begin{sphinxuseclass}{sd-mb-0}
\begin{sphinxuseclass}{sd-ml-0}
\begin{sphinxuseclass}{sd-mr-0}
\begin{sphinxuseclass}{sd-card}
\begin{sphinxuseclass}{sd-sphinx-override}
\begin{sphinxuseclass}{sd-w-100}
\begin{sphinxuseclass}{sd-shadow-md}
\begin{sphinxuseclass}{sd-card-hover}
\begin{sphinxuseclass}{sd-text-center}
\begin{sphinxuseclass}{sd-card-body}
\begin{sphinxuseclass}{sd-card-title}
\begin{sphinxuseclass}{sd-font-weight-bold}Financial Institutions and Crypto Customers
\end{sphinxuseclass}
\end{sphinxuseclass}




\end{sphinxuseclass}\sphinxhref{https://dlt-science.github.io/science-notes/LEGACY/legacy.html}{}
\end{sphinxuseclass}
\end{sphinxuseclass}
\end{sphinxuseclass}
\end{sphinxuseclass}
\end{sphinxuseclass}
\end{sphinxuseclass}
\end{sphinxuseclass}
\end{sphinxuseclass}
\end{sphinxuseclass}
\end{sphinxuseclass}
\end{sphinxuseclass}
\end{sphinxuseclass}
\end{sphinxuseclass}
\end{sphinxuseclass}
\end{sphinxuseclass}
\end{sphinxuseclass}
\end{sphinxuseclass}
\end{sphinxuseclass}
\end{sphinxuseclass}
\end{sphinxuseclass}
\end{sphinxuseclass}
\end{sphinxuseclass}
\end{sphinxuseclass}
\end{sphinxuseclass}
\end{sphinxuseclass}
\end{sphinxuseclass}
\sphinxstepscope


\section{Mobile Theft Prevention using Blockchain}
\label{\detokenize{MTP/mtp:mobile-theft-prevention-using-blockchain}}\label{\detokenize{MTP/mtp::doc}}


\sphinxAtStartPar
\sphinxstylestrong{Industry Perspective}

\begin{sphinxadmonition}{note}{Key Insights}
\begin{itemize}
\item {} 
\sphinxAtStartPar
Mobile theft is a major concern for smartphone users worldwide, with an estimated 70 million smartphones lost each year.

\item {} 
\sphinxAtStartPar
Blockchain technology has the potential to provide a secure and decentralized solution to prevent mobile theft.

\item {} 
\sphinxAtStartPar
The proposed model of using blockchain for mobile theft prevention offers several potential advantages over existing methods, including decentralized and tamper\sphinxhyphen{}proof tracking, automation of process, cross\sphinxhyphen{}border usage, and cost reduction.

\item {} 
\sphinxAtStartPar
The smart contract enables the registration of new mobile devices and maps them to their respective phone numbers. It provides a secure and tamper\sphinxhyphen{}proof solution for tracking the status of mobile devices on the blockchain.

\item {} 
\sphinxAtStartPar
The implementation of blockchain\sphinxhyphen{}based mobile theft prevention solutions provides an added layer of security that can greatly benefit mobile phone users, manufacturers, and society at large.

\end{itemize}
\end{sphinxadmonition}


\subsection{Introduction}
\label{\detokenize{MTP/mtp:introduction}}
\sphinxAtStartPar
Mobile theft is a major concern for smartphone users worldwide. With the increasing reliance on mobile devices for personal and professional use, the theft or loss of a smartphone can result in a significant loss of data and privacy. Studies indicate that a staggering number of smartphones, estimated at 70 million, are lost each year, with a meager 7\% recovered {[}\hyperlink{cite.MTP/mtp:id15}{Hom16}{]}. Further, company\sphinxhyphen{}issued smartphones are not immune to these occurrences, as research has shown that 4.3\% of them are lost or stolen annually. Workplace and conference environments are the leading hotspots for smartphone theft, with 52\% and 24\% of devices stolen, respectively. Moreover, these numbers appear to be increasing, with recent studies indicating a rise of 39.2\% between 2019 and 2021 {[}\hyperlink{cite.MTP/mtp:id16}{Hen22}{]}. Given these alarming statistics, there is a growing need for effective mobile theft prevention measures. Blockchain technology has the potential to provide a secure and decentralized solution to prevent mobile theft. By leveraging the immutable and distributed nature of blockchain, it is possible to create a tamper\sphinxhyphen{}proof system that can prevent unauthorized access to mobile devices. In this article, we will explore the potential of blockchain technology for mobile theft prevention, its advantages and limitations, and the future prospects of this emerging field.

\sphinxAtStartPar
The proposed technology of using blockchain for mobile theft prevention is still in the development stage and has not yet been widely adopted on a national or international level. However, there are several companies and organizations that are exploring the use of blockchain for mobile security and anti\sphinxhyphen{}theft solutions. Internationally, companies such as Samsung and Huawei are researching the use of blockchain for mobile security, with Samsung filing several patents for blockchain\sphinxhyphen{}based mobile security solutions {[}\hyperlink{cite.MTP/mtp:id17}{For22}, \hyperlink{cite.MTP/mtp:id18}{Hua18}{]}.

\sphinxAtStartPar
There is currently no known widespread adoption of blockchain for mobile theft prevention. However, governments all over the world have been exploring the use of blockchain for various applications, including supply chain management and digital identity. This indicates that there is an interest in the technology and a potential for the proposed model to be adopted globally.


\subsection{Rationale Behind Mobile Theft Prevention using Blockchain}
\label{\detokenize{MTP/mtp:rationale-behind-mobile-theft-prevention-using-blockchain}}
\sphinxAtStartPar
Mobile theft has become a growing concern for individuals and organizations around the world. In addition to the financial loss associated with the theft, there is also a significant risk of personal data being compromised. The use of blockchain technology for mobile theft prevention offers a secure and efficient solution for preventing mobile theft {[}\hyperlink{cite.MTP/mtp:id19}{Gob18}{]}. This technology can help individuals and organizations protect their mobile devices and personal information by providing a decentralized and tamper\sphinxhyphen{}proof way to track and block stolen mobile devices. By using private blockchains, the proposed model can be implemented in a way that ensures security and privacy, while also reducing the risk of fraud or malicious activity.
\begin{itemize}
\item {} 
\sphinxAtStartPar
\sphinxstylestrong{Decentralized and tamper\sphinxhyphen{}proof:} Blockchain technology enables a decentralized and tamper\sphinxhyphen{}proof system for tracking and disabling stolen mobile devices. This ensures that the information stored on the blockchain is accurate and cannot be tampered with, making it a reliable source for tracking stolen devices {[}\hyperlink{cite.MTP/mtp:id20}{Chi23}{]}.

\item {} 
\sphinxAtStartPar
\sphinxstylestrong{Secure and private:} The proposed model uses a private blockchain network that connects the mobile manufacturing companies and their nodes {[}\hyperlink{cite.MTP/mtp:id21}{Ire21}{]}. This helps to ensure the security of the network and the data stored in it, and also helps to maintain the privacy of the users.

\item {} 
\sphinxAtStartPar
\sphinxstylestrong{Automation of process:} Smart contracts can be programmed to automatically disable the device once the signal is sent, reducing human error and increasing the efficiency {[}\hyperlink{cite.MTP/mtp:id22}{DD21}{]}.

\item {} 
\sphinxAtStartPar
\sphinxstylestrong{Cross\sphinxhyphen{}border usage:} The proposed model can be used in cross\sphinxhyphen{}border cases, making it more efficient and effective than existing methods {[}\hyperlink{cite.MTP/mtp:id23}{Ram21}{]}.

\item {} 
\sphinxAtStartPar
\sphinxstylestrong{Cost reduction:} By reducing the number of mobile thefts, the proposed model can also have a positive economic impact. This can include reducing the costs associated with mobile theft for consumers, mobile carriers, and insurance companies {[}\hyperlink{cite.MTP/mtp:id24}{Ali20}{]}.

\end{itemize}


\subsection{Alternative Technologies Available under Development}
\label{\detokenize{MTP/mtp:alternative-technologies-available-under-development}}\begin{itemize}
\item {} 
\sphinxAtStartPar
\sphinxstylestrong{IMEI blocking:} One of the most common methods for preventing mobile theft is to block the IMEI (International Mobile Equipment Identity) number of a stolen device. This can be done by reporting the theft to the mobile carrier, who will then blacklist the IMEI number and prevent the device from connecting to the network {[}\hyperlink{cite.MTP/mtp:id25}{Hic22}{]}.

\item {} 
\sphinxAtStartPar
\sphinxstylestrong{SIM card blocking:} Similar to IMEI blocking, SIM card blocking involves disabling the SIM card of a stolen device. This can be done by reporting the theft to the mobile carrier, who will then deactivate the SIM card and prevent the device from connecting to the network {[}\hyperlink{cite.MTP/mtp:id26}{Tre15}{]}.

\item {} 
\sphinxAtStartPar
\sphinxstylestrong{Remote wipe:} Some mobile devices include a remote wipe feature, which allows the device owner to remotely delete all of the data on their device if it is lost or stolen {[}\hyperlink{cite.MTP/mtp:id27}{AIT23}{]}.

\item {} 
\sphinxAtStartPar
\sphinxstylestrong{Mobile tracking apps:} There are a variety of mobile tracking apps available that allow device owners to track the location of their device and remotely lock or wipe it if it is lost or stolen {[}\hyperlink{cite.MTP/mtp:id28}{Mar23}{]}.

\end{itemize}

\sphinxAtStartPar
In comparison, the model of using blockchain for mobile theft prevention offers several potential advantages over these existing methods. A decentralized and tamper\sphinxhyphen{}proof system for tracking and disabling stolen devices, and the smart contract can be programmed to automatically disable the device once the signal is sent, reducing human error and increasing the efficiency. Additionally, the proposed model can potentially work in cross\sphinxhyphen{}border cases, which is not possible with IMEI and SIM card blocking, and also can be integrated with other theft prevention methods.


\subsection{Methodology}
\label{\detokenize{MTP/mtp:methodology}}
\sphinxAtStartPar
The smart contract enables the registration of new mobile devices and maps them to their respective phone numbers. This allows users to update the status of their mobile devices on the blockchain, indicating whether they are lost or stolen. The smart contract also allows for changes to be made to the registered mobile devices’ information, such as their International Mobile Equipment Identity (IMEI) number, and to update the corresponding phone number. In this way, the smart contract provides a secure and tamper\sphinxhyphen{}proof solution for tracking the status of mobile devices on the blockchain.

\sphinxAtStartPar
The mobile application is designed to constantly monitor the state of the mobile device by making API calls to the blockchain. If the blockchain indicates that the device has been reported stolen, the application takes action by disabling the device’s Wi\sphinxhyphen{}Fi and network connections and forcing it into airplane mode. By doing so, the application prevents the thief from using any of the phone’s features, rendering it useless until it can be recovered by the rightful owner.

\sphinxAtStartPar
When a mobile phone is marked as stolen on the blockchain through the smart contract and later found, the owner can connect it to a computer via USB and use USB mode to provide data to the phone. This allows the owner to activate the phone again by providing the data through the USB based hotspot.

\begin{figure}[htbp]
\centering
\capstart

\noindent\sphinxincludegraphics[width=500\sphinxpxdimen,height=281\sphinxpxdimen]{{mtp}.png}
\caption{Working Mechanism of Mobile Theft Prevention using Blockchain}\label{\detokenize{MTP/mtp:mtp}}\end{figure}

\sphinxAtStartPar
The \sphinxhref{https://gist.github.com/yathin017/ec50576dcd706e5868f4d96edcd5a00c}{smart contract} is written in both Solidity and JavaScript programming languages that can be deployed on a blockchain network. It is designed to prevent mobile theft by using a mapping function to keep track of mobile devices using their IMEI numbers and phone numbers.

\sphinxAtStartPar
The smart contract consists of six functions that can be called by authorized users.
\begin{itemize}
\item {} 
\sphinxAtStartPar
\sphinxcode{\sphinxupquote{addIMEI()}}  allows users to add their mobile devices to the blockchain by passing in their IMEI and phone numbers. The function first checks if the IMEI and phone numbers already exist on the blockchain, and if not, it adds the device to the mapping function.

\item {} 
\sphinxAtStartPar
\sphinxcode{\sphinxupquote{activateLost()}}  is used to activate the lost mode of a mobile device. The function checks if the IMEI number of the device exists on the blockchain and if it does, it sets the value of  \sphinxcode{\sphinxupquote{isIMEIlost}} to true, indicating that the device is lost.

\item {} 
\sphinxAtStartPar
\sphinxcode{\sphinxupquote{deactivateLost()}} is used to deactivate the lost mode of a mobile device. The function checks if the IMEI number of the device exists on the blockchain and if it does, it sets the value of \sphinxcode{\sphinxupquote{isIMEIlost}} to false, indicating that the device is no longer lost.

\item {} 
\sphinxAtStartPar
\sphinxcode{\sphinxupquote{changeIMEI()}} allows users to change the IMEI number of their device. The function checks if the old IMEI and phone number exists on the blockchain and if it does, it replaces the old IMEI with the new one.

\item {} 
\sphinxAtStartPar
\sphinxcode{\sphinxupquote{changePhoneNumber()}} allows users to change the phone number associated with their device. The function checks if the old IMEI and phone number exists on the blockchain and if it does, it replaces the old phone number with the new one.

\item {} 
\sphinxAtStartPar
\sphinxcode{\sphinxupquote{checkIMEI()}} is a view function that allows anyone to check if a particular device is lost by passing in the IMEI number of the device. The function returns true if the device is lost, and false if it is not.

\end{itemize}


\subsection{Impact on Users and Mobile Manufacturers}
\label{\detokenize{MTP/mtp:impact-on-users-and-mobile-manufacturers}}
\sphinxAtStartPar
As the world continues to advance technologically, mobile phone theft has become a common issue that affects many people. However, with the implementation of a blockchain\sphinxhyphen{}based mobile theft prevention solution, it is possible to mitigate this problem.

\sphinxAtStartPar
For users, this solution provides an added layer of security, ensuring that their mobile devices cannot be easily used if they are lost or stolen. With the mobile application continuously reading the state of the mobile through API calls to the blockchain, it is possible to detect if the mobile is stolen, and take appropriate actions to disable the mobile network, Wi\sphinxhyphen{}Fi, and force activate airplane mode, preventing the thief from using any of the phone’s functionalities.

\sphinxAtStartPar
For mobile manufacturers, implementing blockchain\sphinxhyphen{}based mobile theft prevention solutions will increase customer satisfaction and retention as users are likely to be attracted by the added security feature. This, in turn, will lead to an increase in sales and profits.


\subsection{Economic and Social Benefits}
\label{\detokenize{MTP/mtp:economic-and-social-benefits}}
\sphinxAtStartPar
The implementation of blockchain\sphinxhyphen{}based mobile theft prevention solutions will lead to a reduction in mobile phone theft and related crimes. This will result in a decrease in the costs of replacing stolen or lost mobile phones, and a corresponding increase in the amount of money available for investment in other areas of the economy. Additionally, it can also help to reduce insurance premiums for mobile phone owners, leading to savings for consumers.

\sphinxAtStartPar
On a social level, it can help to reduce the fear of being robbed or mugged and reduce the potential for violent confrontations between victims and thieves. This can lead to an overall improvement in public safety and security.


\subsection{Future Possibilities and Extensions}
\label{\detokenize{MTP/mtp:future-possibilities-and-extensions}}
\sphinxAtStartPar
The implementation of this blockchain\sphinxhyphen{}based mobile theft prevention solution has future possibilities and extensions. It can be extended to other mobile devices like laptops, tablets, and smartwatches, further increasing the level of security for users. Additionally, it can be integrated with existing law enforcement agencies to enhance the tracking of lost or stolen mobile devices. This will make it easier for law enforcement to recover stolen mobile devices and increase the likelihood of criminals being brought to justice.

\sphinxAtStartPar
In conclusion, the implementation of blockchain\sphinxhyphen{}based mobile theft prevention solutions provides an added layer of security that can greatly benefit mobile phone users, manufacturers, and society at large. The potential for future extensions and possibilities only adds to its value, making it an ideal solution for improving the safety and security of mobile devices.




\subsection{References}
\label{\detokenize{MTP/mtp:references}}
\sphinxstepscope


\section{Financial Institutions and Crypto Customers}
\label{\detokenize{LEGACY/legacy:financial-institutions-and-crypto-customers}}\label{\detokenize{LEGACY/legacy::doc}}
\sphinxAtStartPar
\sphinxstylestrong{Industry Perspective}

\sphinxAtStartPar
\sphinxstylestrong{Disclaimer:} The views and opinions expressed in this article are solely those of the author

\begin{sphinxadmonition}{note}{Key Insights}
\begin{itemize}
\item {} 
\sphinxAtStartPar
Legacy banks and financial institutions face internal and external challenges in integrating cryptocurrencies into their existing business framework for generating new revenue streams.

\item {} 
\sphinxAtStartPar
The internal perspective involves leveraging distributed ledger technology for an overall better management.

\item {} 
\sphinxAtStartPar
The external perspective focuses on managing crypto customers, including regulatory compliance and environmental sustainability.

\item {} 
\sphinxAtStartPar
Legacy banks often miss out on business opportunities due to constraints related to regulation, organization, processes, delivery, and employee biases.

\item {} 
\sphinxAtStartPar
Innovative central organizational structures should establish clear provisions, monitor key performance indicators, and develop product management strategies.

\item {} 
\sphinxAtStartPar
Offering tailored financial products, such as crypto mining equipment insurance and crypto hedging insurance, can enhance profitability and help diversifying revenue streams.

\item {} 
\sphinxAtStartPar
Talent acquisition and skill development are crucial to improve crypto risk management within legacy banks.

\item {} 
\sphinxAtStartPar
Additional expertise is required in advanced programming languages, an end\sphinxhyphen{}to\sphinxhyphen{}end vision of innovation, and product design.

\item {} 
\sphinxAtStartPar
By addressing these challenges, legacy financial institutions can embrace the potential of cryptocurrencies and stay ahead in the evolving financial landscape.

\end{itemize}
\end{sphinxadmonition}


\subsection{A Background Story}
\label{\detokenize{LEGACY/legacy:a-background-story}}
\sphinxAtStartPar
Some months ago, a friend of mine working in the sales department of a major European bank told me about a chaotic experience he had with a potential client. The client was a cryptocurrency miner who had invested in a warehouse in Switzerland and filled it with state\sphinxhyphen{}of\sphinxhyphen{}the\sphinxhyphen{}art computers for mining. The business was thriving, and with the cryptocurrency boom, it was yielding good profits. However, some weather events, like a hailstorm, had caused heavy damage to his equipment.

\sphinxAtStartPar
The miner contacted my friend to inquire about insurance protection against such weather events. My friend saw the opportunity and approached the relevant departments in his bank, including bank insurance and product design. That’s where the story ended. My friend spent countless meetings and days explaining the basics to our colleagues: what a miner does, their work, and their business. Even escalations to top management were in vain. In the meantime, the miner found an alternative financial solution provided by a cutting\sphinxhyphen{}edge FinTech, and as a result, both he and his bank lost an outstanding prospective customer. Preventing this missed business opportunities from happening again in legacy banks is the subject of this article.

\begin{figure}[htbp]
\centering
\capstart

\noindent\sphinxincludegraphics[width=750\sphinxpxdimen,height=400\sphinxpxdimen]{{legacy}.png}
\caption{Functions of the Relationship Manager.}\label{\detokenize{LEGACY/legacy:legacy}}\end{figure}


\subsection{Introduction}
\label{\detokenize{LEGACY/legacy:introduction}}
\sphinxAtStartPar
Cryptocurrencies present unique challenges for legacy financial institutions from both internal and external viewpoints. Internally, these institutions must explore how to integrate cryptocurrencies and distributed ledger technologies into their existing systems, leveraging the technology to embrace a new era of financial management. This internal perspective significantly impacts processes and IT infrastructure and also has implications for costs and IT investments, from a financial statement standpoint.

\sphinxAtStartPar
Externally, the presence of institutional and retail cryptocurrency customers (e.g., cryptocurrency platforms or cryptocurrency holders) creates a conundrum. This demands meticulous attention to regulatory compliance, risk management, and product offerings within the existing framework, which aim to discover innovative revenue streams beyond traditional avenues. Indeed, from a financial statement perspective, this external view could have a significant impact on revenues, allowing a diversification strategy.

\begin{sphinxadmonition}{note}{Internal Perspective}

\sphinxAtStartPar
Legacy banks face the challenge of integrating cryptocurrencies into their traditional banking systems, which involves exploring ways to leverage blockchain technology, decentralization, and secure transactions. By effectively integrating these digital assets, legacy banks can tap into the potential of cryptocurrencies and offer innovative financial services to their customers.
\end{sphinxadmonition}


\subsection{Managing Crypto Customers}
\label{\detokenize{LEGACY/legacy:managing-crypto-customers}}
\sphinxAtStartPar
Legacy banks encounter a unique set of challenges when engaging with crypto customers {[}\hyperlink{cite.LEGACY/legacy:id93}{Ban23a}, \hyperlink{cite.LEGACY/legacy:id94}{Ban23b}{]}. Relationship managers and salespeople must navigate the complexities of regulatory compliance, such as Anti\sphinxhyphen{}Money Laundering (AML) and Know Your Customer (KYC) regulations, to ensure adherence while facilitating a seamless customer experience. Additionally, concerns related to environmental sustainability and energy consumption associated with cryptocurrencies need to be addressed, providing clarity on the bank’s stance regarding these issues.

\sphinxAtStartPar
Legacy banks often miss out on business opportunities due to the interplay of regulatory, organizational, process, delivery, and employee bias constraints. To overcome these constraints and expand their business opportunities, legacy banks must adopt a proactive and reactive approach by focusing on setting up new central organizational structures within their general management organization. These new structures have responsibilities to provide clear provisions (both from credit risk and reputation risk management), Key Performance Indicators (KPIs), and product management strategies exclusively dedicated to managing cryptocurrencies and serving crypto customers.

\sphinxAtStartPar
By proactively tackling both internal and external challenges, legacy financial institutions can begin to adeptly navigate the rapidly shifting landscape of cryptocurrency. In this article, we delve into a variety of innovative strategies and approaches that could be harnessed to tap into the vast potential of cryptocurrencies. A profound paradigm shift, altering how legacy banks engage with and manage cryptocurrencies, is indeed a pressing necessity in today’s digital era.


\subsection{Assessing Regulatory and Compliance Considerations}
\label{\detokenize{LEGACY/legacy:assessing-regulatory-and-compliance-considerations}}
\sphinxAtStartPar
Engaging with crypto customers entails navigating through stringent regulatory obligations, such as AML and KYC regulations. Relationship managers and salespersons must ensure effective compliance with these regulations while also addressing concerns regarding environmental sustainability, given the energy\sphinxhyphen{}intensive nature of certain cryptocurrencies.

\sphinxAtStartPar
Given that these professionals receive no support from other organizational structures within their banking or financial institution, the burden of responsibility rests squarely on their shoulders. The breadth of their tasks is considerable, presenting a complex landscape that must be navigated independently.

\sphinxAtStartPar
To illustrate this from an operational standpoint, these individuals must personally interpret and apply relevant legal constraints due to a lack of cryptocurrency regulatory knowledge amongst legal professionals in traditional banks. Beyond this, they must design and implement a profitable pricing strategy that gains management approval, a task often involving numerous meetings, complex analyses, and financial simulations.

\sphinxAtStartPar
Moreover, they must fulfill climate, anti\sphinxhyphen{}money laundering, and privacy assessments, which are lengthy, complex, and mandatory in the credit process {[}\hyperlink{cite.LEGACY/legacy:id95}{Uni18a}{]}. All these activities are required to be performed simultaneously and swiftly to maintain a positive commercial relationship with customers, and analyze competitors’ actions. A chaotic, complex, and overwhelming task without help!

\sphinxAtStartPar
A more constructive approach has been lacking mainly due to the two major reasons:
\begin{itemize}
\item {} 
\sphinxAtStartPar
Banks employees’ biases and a negative approach towards cryptos

\item {} 
\sphinxAtStartPar
Unclear regulation about cryptos.

\end{itemize}

\sphinxAtStartPar
On the latter, the EU has been working on regulating cryptocurrencies to address potential risks and ensure consumer protection. While there is no specific comprehensive regulation for crypto management in banks, the following regulations may be relevant:
\begin{itemize}
\item {} 
\sphinxAtStartPar
MiCAR (Market in Crypto\sphinxhyphen{}Assets Regulation) {[}\hyperlink{cite.LEGACY/legacy:id96}{Com20}{]}.

\item {} 
\sphinxAtStartPar
Anti\sphinxhyphen{}Money Laundering Directive (AMLD) {[}\hyperlink{cite.LEGACY/legacy:id98}{Uni15}{]}.

\item {} 
\sphinxAtStartPar
Markets in Financial Instruments Directive (MiFID II) {[}\hyperlink{cite.LEGACY/legacy:id99}{Uni14}{]}.

\item {} 
\sphinxAtStartPar
EU General Data Protection Regulation (2016/679, “GDPR”) {[}\hyperlink{cite.LEGACY/legacy:id97}{Uni18b}{]}.

\end{itemize}

\sphinxAtStartPar
From the perspective of a middle\sphinxhyphen{}aged EU bank employee, these regulations fail to provide a clear and straightforward framework. Instead, they offer only basic principles and contribute to the overwhelming amount of documentation that banks must manage within the European Union. Essentially, this situation becomes a burden, leading to additional costs without generating any significant increase in revenue. We will explore some potential solutions to these problems next.


\subsection{Expanding Business Opportunities}
\label{\detokenize{LEGACY/legacy:expanding-business-opportunities}}
\sphinxAtStartPar
Relationship managers and salespersons of legacy banks usually miss out on significant business opportunities due to the interplay of the regulatory, organizational, process, delivery, and employee biases constraints. To overcome these limitations, again, a paradigm shift is needed. Instead of treating crypto customers as a challenging prospect, the entire organization should embrace a proactive and reactive approach, which means setting up enabling factors for allowing the onboarding of crypto customers. These enabling factors entail creating central organizational structures within the bank that specialize in crypto management, encompassing risk management, AML, privacy, and product creation and offerings.

\begin{figure}[htbp]
\centering
\capstart

\noindent\sphinxincludegraphics[width=750\sphinxpxdimen,height=400\sphinxpxdimen]{{legacy2}.png}
\caption{Alternative Scenario with Relationship Manager.}\label{\detokenize{LEGACY/legacy:legacy2}}\end{figure}


\subsection{Establishing Centralized Organizational Structures}
\label{\detokenize{LEGACY/legacy:establishing-centralized-organizational-structures}}
\sphinxAtStartPar
Central organizational structures dedicated to managing cryptocurrencies need to be set up from scratch because they can provide the necessary focus and guidance, encompassing both business and technical perspectives. These entities should establish clear provisions, mapping opportunities and risks, setting up and monitoring KPIs and product management strategies.

\sphinxAtStartPar
By assigning specific responsibilities and clearly structured activities, legacy banks can effectively identify and serve the right crypto customers, while ensuring compliance and risk mitigation and disregarding crypto customers that do not respect the bank provisions (e. g. crypto customers that do not have data privacy IT servers within a certain list of countries, or crypto customers that have some previous criminal records in terms of money laundering). Furthermore, offering tailored financial products, such as financing and insurance solutions, can enhance the profitability of the legacy institution. These bespoke products can be imagined only from employees who are engaged and motivated in the crypto world. Here are some examples of innovative crypto products that legacy banks could sell to crypto customers:
\begin{enumerate}
\sphinxsetlistlabels{\arabic}{enumi}{enumii}{}{.}%
\item {} 
\sphinxAtStartPar
Crypto Mining Equipment Insurance: Develop insurance policies specifically designed to cover the risks associated with crypto mining equipment, such as physical risks (e.g., flood, hailstorm, theft, damage, or breakdown. This type of coverage would provide financial protection for miners who invest heavily in hardware

\item {} 
\sphinxAtStartPar
Crypto Hedging Insurance: Develop insurance policies specifically designed to cover the risks of extreme price volatility. This type of coverage would provide financial protection for crypto holders against price volatility

\item {} 
\sphinxAtStartPar
Operative Risks Insurance: Offer insurance policies that cover losses resulting from smart contract vulnerabilities, coding errors, fraudulent or failed transactions. This coverage could offer reimbursement for lost funds due to transaction errors, technical glitches, or fraudulent activities.

\end{enumerate}


\subsection{Talent Acquisition and Skill Development}
\label{\detokenize{LEGACY/legacy:talent-acquisition-and-skill-development}}
\sphinxAtStartPar
Overcoming biases and improving crypto risk management within legacy banks may require attracting talent from outside the financial industry. Re\sphinxhyphen{}skilling and up\sphinxhyphen{}skilling existing employees may not be sufficient; therefore, individuals with fresh perspectives and expertise, especially those from product design and development will be required. This transformation process should focus on developing a workforce with the necessary skills to navigate the complexities of cryptocurrencies and related financial services.

\sphinxAtStartPar
In the general context, and simplifying it to the utmost, within legacy banks there are now individuals who possess extensive expertise in accounting, credit risk and loan origination, as well as classical IT knowledge related to applications managing transactions and data. The skills that are lacking to integrate an active and proactive understanding of cryptocurrencies pertain to:
\begin{enumerate}
\sphinxsetlistlabels{\arabic}{enumi}{enumii}{}{.}%
\item {} 
\sphinxAtStartPar
Advanced and cutting\sphinxhyphen{}edge programming languages

\item {} 
\sphinxAtStartPar
An eclectic and end\sphinxhyphen{}to\sphinxhyphen{}end vision of innovation (currently, innovation departments in legacy banks only focus on processes and costs rather than revenues and products)

\item {} 
\sphinxAtStartPar
Proficiency in designing new products that are completely different from the traditional ones.

\end{enumerate}


\subsection{Conclusion}
\label{\detokenize{LEGACY/legacy:conclusion}}
\sphinxAtStartPar
Legacy financial institutions must proactively address the challenges associated with managing crypto customers. By navigating regulatory considerations, expanding business opportunities, establishing central organizations, and attracting specialized talent, these institutions can unlock the potential of cryptocurrencies within their existing frameworks. Embracing this transformation will not only help overcome the constraints imposed by regulation and biases but will also position the institutions at the forefront of the evolving financial landscape.




\subsection{References}
\label{\detokenize{LEGACY/legacy:references}}
\sphinxstepscope


\chapter{Innovation \& Ideation}
\label{\detokenize{STRUCTURE/innovation:innovation-ideation}}\label{\detokenize{STRUCTURE/innovation::doc}}
\begin{sphinxuseclass}{sd-container-fluid}
\begin{sphinxuseclass}{sd-sphinx-override}
\begin{sphinxuseclass}{sd-mb-4}
\begin{sphinxuseclass}{sd-row}
\begin{sphinxuseclass}{sd-row-cols-1}
\begin{sphinxuseclass}{sd-row-cols-xs-1}
\begin{sphinxuseclass}{sd-row-cols-sm-2}
\begin{sphinxuseclass}{sd-row-cols-md-2}
\begin{sphinxuseclass}{sd-row-cols-lg-2}
\begin{sphinxuseclass}{sd-g-2}
\begin{sphinxuseclass}{sd-g-xs-2}
\begin{sphinxuseclass}{sd-g-sm-2}
\begin{sphinxuseclass}{sd-g-md-2}
\begin{sphinxuseclass}{sd-g-lg-2}
\begin{sphinxuseclass}{sd-col}
\begin{sphinxuseclass}{sd-d-flex-row}
\begin{sphinxuseclass}{sd-mt-3}
\begin{sphinxuseclass}{sd-mb-0}
\begin{sphinxuseclass}{sd-ml-0}
\begin{sphinxuseclass}{sd-mr-0}
\begin{sphinxuseclass}{sd-card}
\begin{sphinxuseclass}{sd-sphinx-override}
\begin{sphinxuseclass}{sd-w-100}
\begin{sphinxuseclass}{sd-shadow-md}
\begin{sphinxuseclass}{sd-card-hover}
\begin{sphinxuseclass}{sd-text-center}
\begin{sphinxuseclass}{sd-card-body}
\begin{sphinxuseclass}{sd-card-title}
\begin{sphinxuseclass}{sd-font-weight-bold}Sharding: A Panacea for Blockchain Scalability Challenges
\end{sphinxuseclass}
\end{sphinxuseclass}




\end{sphinxuseclass}\sphinxhref{https://dlt-science.github.io/science-notes/SHARDING/sharding.html}{}
\end{sphinxuseclass}
\end{sphinxuseclass}
\end{sphinxuseclass}
\end{sphinxuseclass}
\end{sphinxuseclass}
\end{sphinxuseclass}
\end{sphinxuseclass}
\end{sphinxuseclass}
\end{sphinxuseclass}
\end{sphinxuseclass}
\end{sphinxuseclass}
\end{sphinxuseclass}
\begin{sphinxuseclass}{sd-col}
\begin{sphinxuseclass}{sd-d-flex-row}
\begin{sphinxuseclass}{sd-mt-3}
\begin{sphinxuseclass}{sd-mb-0}
\begin{sphinxuseclass}{sd-ml-0}
\begin{sphinxuseclass}{sd-mr-0}
\begin{sphinxuseclass}{sd-card}
\begin{sphinxuseclass}{sd-sphinx-override}
\begin{sphinxuseclass}{sd-w-100}
\begin{sphinxuseclass}{sd-shadow-md}
\begin{sphinxuseclass}{sd-card-hover}
\begin{sphinxuseclass}{sd-text-center}
\begin{sphinxuseclass}{sd-card-body}
\begin{sphinxuseclass}{sd-card-title}
\begin{sphinxuseclass}{sd-font-weight-bold}Self\sphinxhyphen{}Sovereign Identity: Technical Foundations and Applications
\end{sphinxuseclass}
\end{sphinxuseclass}




\end{sphinxuseclass}\sphinxhref{https://dlt-science.github.io/science-notes/SSI/ssi.html}{}
\end{sphinxuseclass}
\end{sphinxuseclass}
\end{sphinxuseclass}
\end{sphinxuseclass}
\end{sphinxuseclass}
\end{sphinxuseclass}
\end{sphinxuseclass}
\end{sphinxuseclass}
\end{sphinxuseclass}
\end{sphinxuseclass}
\end{sphinxuseclass}
\end{sphinxuseclass}
\end{sphinxuseclass}
\end{sphinxuseclass}
\end{sphinxuseclass}
\end{sphinxuseclass}
\end{sphinxuseclass}
\end{sphinxuseclass}
\end{sphinxuseclass}
\end{sphinxuseclass}
\end{sphinxuseclass}
\end{sphinxuseclass}
\end{sphinxuseclass}
\end{sphinxuseclass}
\end{sphinxuseclass}
\end{sphinxuseclass}
\sphinxstepscope


\section{Sharding: A Panacea for Blockchain Scalability Challenges?}
\label{\detokenize{SHARDING/sharding:sharding-a-panacea-for-blockchain-scalability-challenges}}\label{\detokenize{SHARDING/sharding::doc}}
\sphinxAtStartPar
\sphinxstylestrong{Innovation \& Ideation}

\begin{sphinxadmonition}{note}{Key Insights}
\begin{itemize}
\item {} 
\sphinxAtStartPar
Sharding is a promising scaling technique for blockchains, dividing the network into smaller partitions called shards to process transactions in parallel, thus increasing throughput.

\item {} 
\sphinxAtStartPar
Sharding approaches in blockchain systems vary, with solutions like Ethereum 2.0 using multiple shard chains coordinated by a beacon chain, and others, such as Near Protocol’s Nightshade, opting for processing data chunks in a single blockchain with different validator sets.

\item {} 
\sphinxAtStartPar
Sharding implementation faces challenges in security, cross\sphinxhyphen{}shard communication, and data availability. These require solutions like random validator assignment, transaction receipts, and erasure coding.

\item {} 
\sphinxAtStartPar
While sharding offers potential scalability improvements, layer 2 solutions like ZK\sphinxhyphen{}Rollups and Optimistic Rollups remain the preferred short\sphinxhyphen{}term scaling methods until sharding proves its ability to handle high transaction volumes.

\end{itemize}
\end{sphinxadmonition}

\sphinxAtStartPar
As the adoption of blockchain technology increases, scalability remains the central challenge and a major obstacle for blockchain to be adopted by mainstream industries. Bitcoin can only process 7 transactions per second (TPS), while the Ethereum blockchain can only process 15 TPS. Although after the Merge of Ethereum 1.0 into Ethereum 2.0, the TPS of Ethereum 2.0 is expected to reach 100,000 TPS, gas fees remain a major issue. Ethereum has been relying on ZK\sphinxhyphen{}rollups to scale the network, but rollups are only a short\sphinxhyphen{}term solution because of interoperability issues with other blockchains since they are mainly Ethereum\sphinxhyphen{}focussed. Therefore, the blockchain community is actively looking for a solution to the scalability problem.

\begin{sphinxShadowBox}
\sphinxstylesidebartitle{\sphinxstylestrong{ZK\sphinxhyphen{}Rollups}}

\sphinxAtStartPar
ZK\sphinxhyphen{}Rollups in Ethereum are a Layer 2 scaling solution that uses zero\sphinxhyphen{}knowledge proofs to bundle multiple transactions into a single proof on the main chain. This reduces on\sphinxhyphen{}chain data storage and gas costs while maintaining security. As a result, ZK\sphinxhyphen{}Rollups enable higher throughput, lower fees, and faster confirmations for Ethereum transactions while preserving privacy and decentralization.
\end{sphinxShadowBox}


\subsection{What is Sharding?}
\label{\detokenize{SHARDING/sharding:what-is-sharding}}
\sphinxAtStartPar
Sharding, originally a database design principle, is now being considered a promising solution to overcome the scalability challenges of blockchain systems. This scaling technique divides the blockchain network into smaller partitions called shards, each responsible for processing a subset of transactions. This allows the blockchain to process more transactions in parallel, thereby increasing the throughput of the system.

\sphinxAtStartPar
There are 2 common techniques blockchains implement to improve throughput:
\begin{itemize}
\item {} 
\sphinxAtStartPar
Delegate all the computation to a small set of powerful nodes; (e.g., Algorand, Solana)

\item {} 
\sphinxAtStartPar
Each node in the network only does a subset of the total work (Sharding). Ethereum, \sphinxhref{https://near.org/}{Near}, \sphinxhref{https://hedera.com/}{Hedera} use this technique.

\end{itemize}

\begin{sphinxadmonition}{note}{Note:}
\sphinxAtStartPar
\sphinxstylestrong{Sharding in Blockchains vs Traditional Databases}

\sphinxAtStartPar
The sharding techniques used in traditional databases cannot be directly applied to blockchains because of the following reasons:
\begin{itemize}
\item {} 
\sphinxAtStartPar
Blockchains rely on Byzantine Fault Tolerance (BFT) consensus protocols which have been shown to be a scalability bottleneck.

\item {} 
\sphinxAtStartPar
Distributed databases depend on highly available transaction coordinators for atomicity and isolation assurance; however, blockchain coordinators could exhibit malicious behaviour.

\item {} 
\sphinxAtStartPar
In a distributed database, any node can belong to any shard, but a blockchain must assign nodes to shards in a secure manner to ensure that no shard can be compromised by the attacker.

\end{itemize}
\end{sphinxadmonition}


\subsection{Different Sharding Approaches}
\label{\detokenize{SHARDING/sharding:different-sharding-approaches}}
\sphinxAtStartPar
Huang et al. {[}\hyperlink{cite.SHARDING/sharding:id71}{HPZ+22}{]} proposed a new cross\sphinxhyphen{}shard blockchain protocol called BrokerChain that aims to address the issue of hot shards and reduce the number of cross\sphinxhyphen{}shard transactions. They showed this protocol outperforms other state\sphinxhyphen{}of\sphinxhyphen{}the\sphinxhyphen{}art sharding methods in terms of transaction throughput, confirmation latency and queue size of transaction pool.
Tennakoon et al. {[}\hyperlink{cite.SHARDING/sharding:id72}{TG22}{]} propose a blockchain sharding protocol with dynamic sharding where
smart contract invocations stored in blocks reconfigure the sharding. This protocol is effective because it improves the efficiency of the blockchain, preventing resource wasting by closing the shards that are not processing as many transactions or are idle.
There have been a few proposed sharded blockchains such as Elastico {[}\hyperlink{cite.SHARDING/sharding:id64}{LNZ+16}{]}, OmniLedger {[}\hyperlink{cite.SHARDING/sharding:id62}{KKJG+18}{]} and RapidChain {[}\hyperlink{cite.SHARDING/sharding:id63}{ZMR18}{]}. Nonetheless, such systems are predominantly constrained to cryptocurrency use cases in open (or permissionless) environments. Due to their reliance on the unspent transaction output (UTXO) model—a simplistic data structure—these methods lack generalizability for applications beyond Bitcoin {[}\hyperlink{cite.SHARDING/sharding:id61}{DDL+19}{]}. So we will focus on more general\sphinxhyphen{}purpose blockchains such as Ethereum and Near Blockchain.

\begin{sphinxShadowBox}
\sphinxstylesidebartitle{\sphinxstylestrong{Hot Shards}}

\sphinxAtStartPar
Hot shards are shards that are experiencing a high volume of transactions, which can negatively impact the performance and security of the blockchain system.
\end{sphinxShadowBox}

\begin{figure}[htbp]
\centering
\capstart

\noindent\sphinxincludegraphics[width=720\sphinxpxdimen,height=330\sphinxpxdimen]{{sharding}.png}
\caption{Sharding in Ethereum vs Near Blockchain}\label{\detokenize{SHARDING/sharding:sharding-eth-near}}\end{figure}


\subsubsection{Sharding in Ethereum}
\label{\detokenize{SHARDING/sharding:sharding-in-ethereum}}
\sphinxAtStartPar
In Ethereum, data is distributed among several “shard chains” ({[}\hyperref[\detokenize{SHARDING/sharding:sharding-eth-near}]{Fig.\@ \ref{\detokenize{SHARDING/sharding:sharding-eth-near}}}{]}). Each of these shard chains submits a record of transactions to the “beacon chain” or “coordinating layer”, which coordinates and manages the shards by maintaining synchronization and ensuring a common ledger. The shards receive sets of transactions from the mempool. Under the Ethereum 2.0 proposal, these TX are split based on their transaction types: Token transfers and Smart contract interactions. Validators then use an EVM to process shards’ data into a block and update the Merkle tree’s state on the beacon chain {[}\hyperlink{cite.SHARDING/sharding:id67}{KTTI22}{]}.


\subsubsection{Sharding in Near Blockchain}
\label{\detokenize{SHARDING/sharding:sharding-in-near-blockchain}}
\sphinxAtStartPar
Near’s sharding technique is called “Nightshade” {[}\hyperlink{cite.SHARDING/sharding:id65}{Nea20a}{]}. Although the full implementation is still in progress, the idea is that instead of having multiple subchains with a single beacon chain, the data is divided into smaller partitions called chunks. Each chunk is processed by a different set of validators. The validators are randomly assigned to chunks, and the assignment is done in a way that the same validator is not assigned to multiple chunks, as shown in {[}\hyperref[\detokenize{SHARDING/sharding:sharding-eth-near}]{Fig.\@ \ref{\detokenize{SHARDING/sharding:sharding-eth-near}}}{]}. At present, the Near blockchain has 4 shards, and the eventual plan is to have 100 shards \{cite\}`near roadmap.




\subsection{Sharding Challenges}
\label{\detokenize{SHARDING/sharding:sharding-challenges}}
\sphinxAtStartPar
The main issue with sharding is that it is extremely complicated to implement, as it opens up possibilities of new attack vectors and security challenges. The following are some of the challenges that need to be addressed before sharding can be implemented in a blockchain system.


\subsubsection{Security}
\label{\detokenize{SHARDING/sharding:security}}
\sphinxAtStartPar
In a 10\sphinxhyphen{}shard system, each shard’s security is reduced by a factor of 10 due to separate validator sets. Upon hard\sphinxhyphen{}forking a non\sphinxhyphen{}sharded chain with X validators into a sharded chain, each shard has X/10 validators. Consequently, compromising one shard necessitates corrupting only 5.1\% (51\% / 10) of the total validators. This is a significant reduction in security. To overcome this challenge, Ethereum uses a beacon chain to randomly assign validators to shards. Blockchains like Near and Algorand use Verifiable Random Functions (VRFs) to assign validators to shards. This ensures that the validators are randomly assigned to shards and the same validator is not assigned to multiple shards.

\sphinxAtStartPar
Hafid et al. {[}\hyperlink{cite.SHARDING/sharding:id70}{HHS22}{]} propose a Probabilistic Generating Function Analysis (PGFA) approach as an effective and tractable method to analyze the security of sharding\sphinxhyphen{}based blockchain protocols. They conclude that an increase in the number of Sybil IDs (unique nodes), network size, and ID Selection Pool (random pool from which nodes are randomly selected to be assigned to shards) size results in a higher failure probability, compromises network security and can lead to shard takeover attacks.

\begin{sphinxShadowBox}
\sphinxstylesidebartitle{\sphinxstylestrong{VRF}}

\sphinxAtStartPar
Verifiable Random Functions (VRFs) are cryptographic primitive that allows a user to generate a random number that can be verified by anyone.
\end{sphinxShadowBox}


\subsubsection{Cross\sphinxhyphen{}Shard Communication}
\label{\detokenize{SHARDING/sharding:cross-shard-communication}}
\sphinxAtStartPar
As the network gets divided into multiple shards, it is important to ensure that the shards can communicate with each other to maintain consistency and interoperability. As seen in {[}\hyperref[\detokenize{SHARDING/sharding:cross-sharding}]{Fig.\@ \ref{\detokenize{SHARDING/sharding:cross-sharding}}}{]}, this can be problematic if there is forking within the shards and the block issuing the transaction is not included in the canonical chain. Both Near and Ethereum overcome this challenge by exchanging receipts between the shards. The receipts are used to prove that a transaction has been executed on a shard {[}\hyperlink{cite.SHARDING/sharding:id68}{Nea20b}{]} and the corresponding transaction can be executed on the other shard. In Hedera Hashgraph, which uses a gossip protocol to exchange information between shards, each shard maintains a queue of outgoing messages for other shards. Messages are sent from one shard to another through nodes randomly contacting each other, along with proof of consensus. The process continues until the receiving shard confirms message processing with an updated sequence number in its shared state {[}\hyperlink{cite.SHARDING/sharding:id69}{Hed20}{]}. Instead of receipts, Hedera uses sequence numbers which are maintained by a shard for each other shard as a proof of latest execution message.

\begin{sphinxShadowBox}
\sphinxstylesidebartitle{\sphinxstylestrong{Sequence Numbers}}

\sphinxAtStartPar
In the context of Hedera’s multi\sphinxhyphen{}shard system, sequence numbers are 64\sphinxhyphen{}bit identifiers assigned to inter\sphinxhyphen{}shard messages to keep track of their order. When a transaction involves resources from different shards, it triggers inter\sphinxhyphen{}shard messages. Each shard maintains a queue of outgoing messages to be sent to other shards, and each message within a specific queue is assigned a unique sequence number.
\end{sphinxShadowBox}

\begin{figure}[htbp]
\centering
\capstart

\noindent\sphinxincludegraphics[width=350\sphinxpxdimen,height=230\sphinxpxdimen]{{cross-shard}.png}
\caption{Cross\sphinxhyphen{}Shard Communication}\label{\detokenize{SHARDING/sharding:cross-sharding}}\end{figure}


\subsubsection{Data Availability}
\label{\detokenize{SHARDING/sharding:data-availability}}
\sphinxAtStartPar
The data availability problem relates to the difficulty of ensuring that all necessary data for verifying a block’s validity is accessible to all participants in the network. For instance, a light client cannot access complete block data and thus cannot
verify the validity of data. To overcome this problem, erasure coding is used. If the light client can retrieve a sufficient number of chunks of data, it can reconstruct the original data and verify the block’s validity. Ethereum and Near are currently using this approach.

\begin{sphinxShadowBox}
\sphinxstylesidebartitle{\sphinxstylestrong{Erasure Codes}}

\sphinxAtStartPar
Erasure codes allow a piece of data M chunks long to be expanded into a piece of data N chunks (“chunks” can be of arbitrary size), such that any M of the N chunks can be used to recover the original data.
\end{sphinxShadowBox}


\subsection{Sharding in Hedera}
\label{\detokenize{SHARDING/sharding:sharding-in-hedera}}
\sphinxAtStartPar
As per Hedera network’s whitepaper {[}\hyperlink{cite.SHARDING/sharding:id69}{Hed20}{]}, it starts as a single shard composed of nodes managed by Governing Council Members. As the council grows, the network will transition to a multi\sphinxhyphen{}shard system to enhance performance, enable parallel consensus, and maintain asynchronous Byzantine fault tolerance. Nodes will be randomly assigned to shards by a master shard, balancing hbar distribution and minimizing centralization risks. Shards will trust and collaborate, allowing seamless cross\sphinxhyphen{}shard transactions. Nodes will communicate via push messages, maintaining queues for inter\sphinxhyphen{}shard messaging. Transactions involving multiple shards will be consistently recorded in each shard’s state, ensuring ledger\sphinxhyphen{}wide coherence and integrity. The master shard will be responsible for maintaining the overall state of the network, including the hbar supply and the hbar distribution across shards.


\subsection{Conclusion}
\label{\detokenize{SHARDING/sharding:conclusion}}
\sphinxAtStartPar
Sharding is the most promising solution to overcome the scalability challenges of blockchain systems. However, although Ethereum and Near have made significant progress in implementing sharding, it is still not time\sphinxhyphen{}tested and it remains to be seen whether these blockchains will be able to bear a load of transactions volume when scenarios such as \sphinxhref{https://www.forbes.com/sites/tatianakoffman/2020/08/31/defi-the-hot-new-crypto-trend-of-2020/?sh=5576a8a05bce}{DeFi boom} or \sphinxhref{https://qz.com/1145833/cryptokitties-is-causing-ethereum-network-congestion}{NFT craze} happen again. Until then, layer 2 solutions such as ZK\sphinxhyphen{}Rollups and Optimistic Rollups will continue to be the preferred scaling solutions for blockchain systems.




\subsection{References}
\label{\detokenize{SHARDING/sharding:references}}
\sphinxstepscope


\section{Self\sphinxhyphen{}Sovereign Identity: Technical Foundations and Applications}
\label{\detokenize{SSI/ssi:self-sovereign-identity-technical-foundations-and-applications}}\label{\detokenize{SSI/ssi::doc}}
\sphinxAtStartPar
\sphinxstylestrong{Innovation \& Ideation}

\begin{sphinxadmonition}{note}{Key Insights}
\begin{itemize}
\item {} 
\sphinxAtStartPar
SSI systems leverage Decentralised Identifiers (DIDs) and Verifiable Credentials (VCs) to enable secure and trustworthy data sharing between issuers, holders, and verifiers, without relying on a centralised authority.

\item {} 
\sphinxAtStartPar
Privacy\sphinxhyphen{}preserving techniques, such as zero\sphinxhyphen{}knowledge proofs and selective disclosure, allow SSI users to maintain control over their digital identities and securely share credentials without exposing unnecessary information.

\item {} 
\sphinxAtStartPar
The implementation of SSI in various industries, including healthcare, land registration, and e\sphinxhyphen{}voting, demonstrates the potential for SSI to revolutionise identity management and enhance security, privacy, and trust in these systems.

\item {} 
\sphinxAtStartPar
While blockchain is not mandatory for SSI systems, its use as a decentralised data registry ensures secure, tamper\sphinxhyphen{}evident, and verifiable storage of credentials, contributing to the trustworthiness and reliability of identity management processes.

\end{itemize}
\end{sphinxadmonition}


\subsection{Introduction}
\label{\detokenize{SSI/ssi:introduction}}
\sphinxAtStartPar
According to World Bank estimates, nearly 850 million people lack an official identity {[}\hyperlink{cite.SSI/ssi:id58}{JC23}{]}, and the proliferation of digital devices has made it increasingly essential to possess a verifiable digital identity. This has led to a rise in digital transactions and the need for a secure and reliable identity management system. SSI is emerging as a decentralised alternative to traditional centralised identity management systems, in which identities are cryptographically verifiable. It allows individuals to control their digital identities and share them with trusted parties. Each entity in the SSI system is identified by a unique DID (Decentralised Identifier) as shown below, which can be resolved to reveal information such as the entity’s public key and other metadata.
\begin{equation*}
\begin{split}
{\tt \underbrace{DID}_{Scheme}: \overbrace{example}^{DID \, Method}:\underbrace{BzCbsNYhMrjHiqZDTUASHg}_{Method \, Specific \, Identifier}} \label{eq1}
\end{split}
\end{equation*}



\sphinxstrong{See also:}
\nopagebreak


\sphinxAtStartPar
Find out more about some of the most commonly used DID methods:
\begin{itemize}
\item {} 
\sphinxAtStartPar
\sphinxhref{https://hyperledger.github.io/indy-did-method/}{DID:INDY}

\item {} 
\sphinxAtStartPar
\sphinxhref{https://developer.uport.me/ethr-did/docs/index}{DID:UPORT}

\item {} 
\sphinxAtStartPar
\sphinxhref{https://sovrin.org/}{DID:SOV}

\end{itemize}



\sphinxAtStartPar
While centralised identities and federated identities offer convenience, control remains with the identity provider {[}\hyperlink{cite.SSI/ssi:id49}{LB15}{]}. User\sphinxhyphen{}centric identities such as OpenID {[}\hyperlink{cite.SSI/ssi:id50}{RR06}{]} and OAuth {[}\hyperlink{cite.SSI/ssi:id51}{FKustersS16}{]} improve portability but do not give complete control to the users. SSI is designed to give users full control over their digital identities, and involves guiding principles around security, controllability, and portability. In addition to providing total control, Bernabe et al. {[}\hyperlink{cite.SSI/ssi:id52}{BCHR+19}{]} presented a classification of techniques for maintaining privacy in SSI, which included Secure Multiparty Computation and Zero\sphinxhyphen{}Knowledge Proofs, among others.

\sphinxAtStartPar
The three main parties involved in SSI systems are the issuer, holder and verifier, as shown in {[}\hyperref[\detokenize{SSI/ssi:ssi-fig}]{Fig.\@ \ref{\detokenize{SSI/ssi:ssi-fig}}}{]}. The issuer issues a cryptographically signed credential to the holder, and the verifier is the entity that confirms the credential’s authenticity using a decentralised data registry such as a Blockchain. Holders store their credentials in secure digital wallets and can share them with other parties as needed. The holder can also create a presentation and share it with the verifier on request.

\begin{sphinxShadowBox}
\sphinxstylesidebartitle{\sphinxstylestrong{SSI}}

\sphinxAtStartPar
Self\sphinxhyphen{}Sovereign Identity (SSI) is a decentralised digital identity management system which leverages blockchain technology as a data registry, allowing individuals to create, control, and share their identities securely.
\end{sphinxShadowBox}

\begin{sphinxShadowBox}
\sphinxstylesidebartitle{\sphinxstylestrong{Verifiable Credential}}

\sphinxAtStartPar
A verifiable credential is a digital artefact that provides tamper\sphinxhyphen{}evident, cryptographically verifiable proof of an individual’s personal information or attributes.
\end{sphinxShadowBox}

\begin{figure}[htbp]
\centering
\capstart

\noindent\sphinxincludegraphics[width=550\sphinxpxdimen,height=300\sphinxpxdimen]{{SSI.drawio}.png}
\caption{SSI entities and their relations}\label{\detokenize{SSI/ssi:ssi-fig}}\end{figure}


\sphinxstrong{See also:}
\nopagebreak


\sphinxAtStartPar
This is a verifiable credential issued using the javascript didkit\sphinxhyphen{}wasm library.
\\
\sphinxhref{https://gist.github.com/singhparshant/d2157ad4b48555c155d0f807d37ec3f1}{Click here for full credential}

\begin{sphinxVerbatim}[commandchars=\\\{\}]
\PYG{p}{\PYGZob{}}
\PYG{p}{...}\PYG{p}{.}
\PYG{+w}{ }\PYG{l+s+s2}{\PYGZdq{}id\PYGZdq{}}\PYG{o}{:}\PYG{l+s+s2}{\PYGZdq{}urn:uuid:7041d211\PYGZhy{}72c9\PYGZhy{}49fe\PYGZhy{}b6d1\PYGZhy{}d8b6b94abfe3\PYGZdq{}}\PYG{p}{,}
\PYG{+w}{      }\PYG{l+s+s2}{\PYGZdq{}type\PYGZdq{}}\PYG{o}{:}\PYG{p}{[}
\PYG{+w}{         }\PYG{l+s+s2}{\PYGZdq{}VerifiableCredential\PYGZdq{}}\PYG{p}{,}
\PYG{+w}{         }\PYG{l+s+s2}{\PYGZdq{}BasicProfile\PYGZdq{}}
\PYG{+w}{      }\PYG{p}{]}\PYG{p}{,}
\PYG{+w}{      }\PYG{l+s+s2}{\PYGZdq{}credentialSubject\PYGZdq{}}\PYG{o}{:}\PYG{p}{\PYGZob{}}
\PYG{+w}{         }\PYG{l+s+s2}{\PYGZdq{}id\PYGZdq{}}\PYG{o}{:}\PYG{l+s+s2}{\PYGZdq{}did:pkh:tz:tz1N699qJqMVbMDan2r6R3QYFw42J5ydReh6\PYGZdq{}}\PYG{p}{,}
\PYG{+w}{         }\PYG{l+s+s2}{\PYGZdq{}alias\PYGZdq{}}\PYG{o}{:}\PYG{l+s+s2}{\PYGZdq{}TU Munich\PYGZdq{}}\PYG{p}{,}
\PYG{+w}{         }\PYG{l+s+s2}{\PYGZdq{}website\PYGZdq{}}\PYG{o}{:}\PYG{l+s+s2}{\PYGZdq{}Germany\PYGZdq{}}\PYG{p}{,}
\PYG{+w}{         }\PYG{l+s+s2}{\PYGZdq{}description\PYGZdq{}}\PYG{o}{:}\PYG{l+s+s2}{\PYGZdq{}My name\PYGZdq{}}\PYG{p}{,}
\PYG{+w}{         }\PYG{l+s+s2}{\PYGZdq{}logo\PYGZdq{}}\PYG{o}{:}\PYG{l+s+s2}{\PYGZdq{}Helene\PYGZhy{}Mayer\PYGZhy{}Ring 7B\PYGZdq{}}
\PYG{+w}{      }\PYG{p}{\PYGZcb{}}\PYG{p}{,}
\PYG{+w}{      }\PYG{l+s+s2}{\PYGZdq{}issuer\PYGZdq{}}\PYG{o}{:}\PYG{l+s+s2}{\PYGZdq{}did:pkh:tz:tz1QRuc9BkvsBfeSGr6kJ5GCzBsrDjMedvA7\PYGZdq{}}\PYG{p}{,}
\PYG{+w}{      }\PYG{l+s+s2}{\PYGZdq{}issuanceDate\PYGZdq{}}\PYG{o}{:}\PYG{l+s+s2}{\PYGZdq{}2023\PYGZhy{}01\PYGZhy{}13T12:24:52.630Z\PYGZdq{}}\PYG{p}{,}
\PYG{p}{...}\PYG{p}{.}
\PYG{p}{\PYGZcb{}}
\end{sphinxVerbatim}



\begin{sphinxadmonition}{note}{Nitty Gritty of SSI}
\begin{itemize}
\item {} 
\sphinxAtStartPar
SSI solutions are designed to be blockchain\sphinxhyphen{}agnostic and adhere to \sphinxhref{https://www.w3.org/TR/did-core/}{W3C’s specifications}.

\item {} 
\sphinxAtStartPar
The identity wallets (e.g., uPort, Trinsic, \sphinxhref{http://Connect.Me}{Connect.Me}) are different from the digital wallets (e.g., Coinbase, Ledger, Trezor) that store cryptocurrencies in the sense that they store and manage DIDs and VCs instead of cryptocurrencies.

\item {} 
\sphinxAtStartPar
To protect privacy, SSI solutions (e.g. \sphinxhyphen{} \sphinxhref{https://medium.com/stm-blockchain/how-zero-knowledge-proofs-work-on-indy-network-241d6da112bc}{Hyperledger Indy} and Aries) are increasingly using Zero\sphinxhyphen{}Knowledge Proofs (ZKPs) to prove the authenticity of credentials without revealing the actual data.

\item {} 
\sphinxAtStartPar
To facilitate secure communication between different SSI components (issuer\sphinxhyphen{}holder\sphinxhyphen{}verifier), \sphinxhref{https://medium.com/decentralized-identity/understanding-didcomm-14da547ca36b}{DIDComm} and \sphinxhref{https://iiw.idcommons.net/101\_Session:\_Verifiable\_Credential\_Handler\_(CHAPI)\_and\_DIDComm}{CHAPI} protocols have been developed and are heavily used.

\end{itemize}
\end{sphinxadmonition}


\subsection{Applications for SSI}
\label{\detokenize{SSI/ssi:applications-for-ssi}}
\begin{sphinxShadowBox}
\sphinxstylesidebartitle{\sphinxstylestrong{Zero\sphinxhyphen{}Knowledge Proofs}}

\sphinxAtStartPar
A zero\sphinxhyphen{}knowledge proof (ZKP) is a cryptographic technique that enables one party, the prover, to convince another party, the verifier, of the validity of a statement or the possession of a secret without revealing any additional information about the underlying secret or data.
\end{sphinxShadowBox}


\subsubsection{SSI in healthcare}
\label{\detokenize{SSI/ssi:ssi-in-healthcare}}
\sphinxAtStartPar
Recent studies have demonstrated the feasibility of using zero\sphinxhyphen{}knowledge proofs to disclose information selectively, such as proof of vaccination status, without revealing users’ identities. These studies have employed interoperable open\sphinxhyphen{}source tools to implement these systems globally at a minimal cost. Schlatt et al. {[}\hyperlink{cite.SSI/ssi:id45}{SSFU22}{]} illustrates how a customer can leverage a Zero\sphinxhyphen{}knowledge Proof concept called ‘blinded link secret’ to disclose information selectively. Similarly, Barros et al. {[}\hyperlink{cite.SSI/ssi:id44}{dVBSFCustodio22}{]} implemented a prototype of an application for presenting proof of vaccination without revealing users’ identities. Furthermore, it uses interoperable open\sphinxhyphen{}source tools across countries to implement this system globally at a minimal cost for each country’s government. The NHS Digital Staff Passport solution {[}\hyperlink{cite.SSI/ssi:id48}{LC22}{]} employs the Sovrin Network as a public key infrastructure (PKI) to manage verifiable credentials for staff onboarding. Hospitals register on the network and use their private keys to sign credentials, while staff members utilise Evernym’s \sphinxhref{https://www.connect.me/}{Connect.Me} SSI digital wallet app to store and share credentials.


\subsubsection{SSI in land registration}
\label{\detokenize{SSI/ssi:ssi-in-land-registration}}
\sphinxAtStartPar
Shuaib et al. {[}\hyperlink{cite.SSI/ssi:id57}{SHU+22}{]} suggest that a blockchain\sphinxhyphen{}based land registry system can be combined with a self\sphinxhyphen{}sovereign identity (SSI) solution to provide a secure and efficient identity management system for landowners. Three existing SSI solutions, Everest, Evernym, and uPort {[}\hyperlink{cite.SSI/ssi:id46}{Ame22}{]}, were evaluated based on SSI principles {[}\hyperlink{cite.SSI/ssi:id61}{All16}{]} to determine their compliance and effectiveness in addressing identity problems in land registry systems. The Everest platform was found to be the most compliant with the SSI principles, whereas Evernym and uPort had some limitations in terms of interoperability and user control.


\subsubsection{SSI in e\sphinxhyphen{}voting}
\label{\detokenize{SSI/ssi:ssi-in-e-voting}}
\sphinxAtStartPar
Estonia is one of the few countries in the world that have managed to make e\sphinxhyphen{}voting a reality {[}\hyperlink{cite.SSI/ssi:id59}{SS22}{]}. Sertkaya et al. {[}\hyperlink{cite.SSI/ssi:id60}{SRR22}{]} proposed an EIV\sphinxhyphen{}AC scheme that integrates the Estonian Internet voting (EIV) scheme with anonymous credentials (AC) based on self\sphinxhyphen{}sovereign identity (SSI). The use of SSI\sphinxhyphen{}based anonymous credentials enables voters to prove their eligibility to vote without revealing their identity. The zero\sphinxhyphen{}knowledge proof of identity is used to prove that the voter has the right to vote without revealing any additional information. The EIV\sphinxhyphen{}AC scheme enhances the security and privacy of the EIV scheme, making it more compliant with privacy\sphinxhyphen{}enhancing and data minimisation regulations.


\subsubsection{SSI in finance and identity management}
\label{\detokenize{SSI/ssi:ssi-in-finance-and-identity-management}}
\sphinxAtStartPar
Innovative proposals surrounding digital identity management systems, such as \sphinxhref{http://www.kiva.org/protocol/}{Kiva’s architecture}, suggest the development of an insurance marketplace for consequential damages related to identity claims. This marketplace could offer a market mechanism for evaluating the accuracy, trustworthiness, and usefulness of various identity claims, subsequently allowing lenders to confidently underwrite loans, even to individuals lacking formal credit history. Furthermore, by leveraging blockchain technology in a semi\sphinxhyphen{}decentralised identity management system, banks and microfinance lenders could underwrite the risk associated with issuing identity credentials, facilitating de\sphinxhyphen{}risking for subsequent lenders.

\sphinxAtStartPar
Ferdous et al. {[}\hyperlink{cite.SSI/ssi:id47}{FIP23}{]} introduce a SSI4Web framework and demonstrate how an SSI\sphinxhyphen{}based framework can be designed for web services and offer a secure and passwordless user authentication mechanism, which eliminates the need for users to remember passwords and reduces the risk of password breaches.


\subsection{Can SSI work without Blockchain?}
\label{\detokenize{SSI/ssi:can-ssi-work-without-blockchain}}
\sphinxAtStartPar
Blockchain is one of many options when implementing a Self\sphinxhyphen{}sovereign Identity system. Alternatives like IPFS, Public\sphinxhyphen{}key cryptography and even traditional Certificate Authorities can be used to implement SSI. However, the main advantage of using Blockchain is that it provides a decentralised and immutable ledger that can be used to store and verify credentials.


\subsection{Conclusion}
\label{\detokenize{SSI/ssi:conclusion}}
\sphinxAtStartPar
Self\sphinxhyphen{}sovereign identity can potentially revolutionise various industries, including healthcare, voting systems and many more. However, as research and development in SSI progress, it will be
crucial to address interoperability, scalability, and usability challenges to realise SSI’s potential in a global context fully.




\subsection{References}
\label{\detokenize{SSI/ssi:references}}
\sphinxstepscope


\part{Q3 2023}

\sphinxstepscope


\chapter{Academic Insights}
\label{\detokenize{STRUCTURE/academic_2:academic-insights}}\label{\detokenize{STRUCTURE/academic_2::doc}}
\begin{sphinxuseclass}{sd-container-fluid}
\begin{sphinxuseclass}{sd-sphinx-override}
\begin{sphinxuseclass}{sd-mb-4}
\begin{sphinxuseclass}{sd-row}
\begin{sphinxuseclass}{sd-row-cols-1}
\begin{sphinxuseclass}{sd-row-cols-xs-1}
\begin{sphinxuseclass}{sd-row-cols-sm-2}
\begin{sphinxuseclass}{sd-row-cols-md-2}
\begin{sphinxuseclass}{sd-row-cols-lg-2}
\begin{sphinxuseclass}{sd-g-2}
\begin{sphinxuseclass}{sd-g-xs-2}
\begin{sphinxuseclass}{sd-g-sm-2}
\begin{sphinxuseclass}{sd-g-md-2}
\begin{sphinxuseclass}{sd-g-lg-2}
\begin{sphinxuseclass}{sd-col}
\begin{sphinxuseclass}{sd-d-flex-row}
\begin{sphinxuseclass}{sd-mt-3}
\begin{sphinxuseclass}{sd-mb-0}
\begin{sphinxuseclass}{sd-ml-0}
\begin{sphinxuseclass}{sd-mr-0}
\begin{sphinxuseclass}{sd-card}
\begin{sphinxuseclass}{sd-sphinx-override}
\begin{sphinxuseclass}{sd-w-100}
\begin{sphinxuseclass}{sd-shadow-md}
\begin{sphinxuseclass}{sd-card-hover}
\begin{sphinxuseclass}{sd-text-center}
\begin{sphinxuseclass}{sd-card-body}
\begin{sphinxuseclass}{sd-card-title}
\begin{sphinxuseclass}{sd-font-weight-bold}Scam Detection on Ethereum
\end{sphinxuseclass}
\end{sphinxuseclass}


\end{sphinxuseclass}\sphinxhref{https://dlt-science.github.io/science-notes/SDE/ScamDetec.html}{}
\end{sphinxuseclass}
\end{sphinxuseclass}
\end{sphinxuseclass}
\end{sphinxuseclass}
\end{sphinxuseclass}
\end{sphinxuseclass}
\end{sphinxuseclass}
\end{sphinxuseclass}
\end{sphinxuseclass}
\end{sphinxuseclass}
\end{sphinxuseclass}
\end{sphinxuseclass}
\end{sphinxuseclass}
\end{sphinxuseclass}
\end{sphinxuseclass}
\end{sphinxuseclass}
\end{sphinxuseclass}
\end{sphinxuseclass}
\end{sphinxuseclass}
\end{sphinxuseclass}
\end{sphinxuseclass}
\end{sphinxuseclass}
\end{sphinxuseclass}
\end{sphinxuseclass}
\end{sphinxuseclass}
\end{sphinxuseclass}
\sphinxstepscope


\section{Cross\sphinxhyphen{}Chain Interoperability: A Comprehensive Overview}
\label{\detokenize{Interoperability/Cross-Chain Interoperability:cross-chain-interoperability-a-comprehensive-overview}}\label{\detokenize{Interoperability/Cross-Chain Interoperability::doc}}
\sphinxAtStartPar
\sphinxstylestrong{Innovation \& Ideation}

\begin{sphinxadmonition}{note}{Key Insights}
\begin{itemize}
\item {} 
\sphinxAtStartPar
The blockchain landscape has evolved rapidly, leading to a plethora of isolated networks.

\item {} 
\sphinxAtStartPar
Cross\sphinxhyphen{}chain interoperability emerges as the solution to bridge these divides, promising seamless communication between blockchains.

\item {} 
\sphinxAtStartPar
Major blockchain networks like Ethereum, Binance Smart Chain, and Cardano each bring unique strengths, but also highlight the challenges of a fragmented ecosystem.

\end{itemize}
\end{sphinxadmonition}


\subsection{1. \sphinxstylestrong{Introduction}}
\label{\detokenize{Interoperability/Cross-Chain Interoperability:introduction}}
\sphinxAtStartPar
The blockchain revolution, which began with the inception of Bitcoin, has given rise to a myriad of networks, each promising transformative solutions for various sectors. From finance to supply chain, the potential applications of blockchain are vast. However, as these networks proliferated, so did the realization of their isolated nature. This siloed landscape, while rich in diversity, often lacks the ability to communicate and collaborate, leading to fragmented ecosystems. Enter the concept of cross\sphinxhyphen{}chain interoperability, a beacon of hope promising a future where blockchains can seamlessly interact, unlocking the full potential of a decentralized digital world.

\begin{figure}[htbp]
\centering

\noindent\sphinxincludegraphics{{p1}.jpg}
\end{figure}


\subsection{2. \sphinxstylestrong{Definition and Importance of Cross\sphinxhyphen{}Chain Interoperability}}
\label{\detokenize{Interoperability/Cross-Chain Interoperability:definition-and-importance-of-cross-chain-interoperability}}
\sphinxAtStartPar
\sphinxstylestrong{Cross\sphinxhyphen{}chain interoperability} can be defined as the ability of different blockchain networks to communicate, share data, and transact with one another without relying on centralized intermediaries. In simpler terms, it’s the bridge that allows one blockchain to “talk” to another.

\sphinxAtStartPar
But why is this so crucial?
\begin{itemize}
\item {} 
\sphinxAtStartPar
\sphinxstylestrong{Maximized Efficiency}: Interoperability allows for the pooling of resources and functionalities from various chains, leading to more efficient and powerful applications.

\item {} 
\sphinxAtStartPar
\sphinxstylestrong{Enhanced Liquidity}: Assets can move freely across chains, creating a more fluid and interconnected financial ecosystem.

\item {} 
\sphinxAtStartPar
\sphinxstylestrong{Broader Adoption}: A unified blockchain ecosystem can be more easily adopted by traditional sectors, as it offers a comprehensive set of solutions without the limitations of individual chains.

\end{itemize}


\subsection{3. \sphinxstylestrong{The Landscape of Blockchain Networks}}
\label{\detokenize{Interoperability/Cross-Chain Interoperability:the-landscape-of-blockchain-networks}}
\sphinxAtStartPar
The current blockchain landscape is a testament to the innovative spirit of the crypto community. Let’s delve into some major players:
\begin{itemize}
\item {} 
\sphinxAtStartPar
\sphinxstylestrong{Ethereum}: Often dubbed the “world computer”, Ethereum introduced the world to smart contracts, self\sphinxhyphen{}executing contracts with the terms of the agreement directly written into code. It has since become the go\sphinxhyphen{}to platform for decentralized applications (dApps) and Decentralized Finance (DeFi) projects.

\item {} 
\sphinxAtStartPar
\sphinxstylestrong{Binance Smart Chain (BSC)}: BSC offers a faster and more scalable alternative to Ethereum, with shorter block times and lower transaction fees. It has attracted a significant number of DeFi projects and users, especially during periods of network congestion on Ethereum.

\item {} 
\sphinxAtStartPar
\sphinxstylestrong{Cardano}: With a research\sphinxhyphen{}driven approach, Cardano emphasizes sustainability, scalability, and transparency. It’s built on a unique “Ouroboros” proof\sphinxhyphen{}of\sphinxhyphen{}stake consensus mechanism and promises more efficient smart contract execution.

\end{itemize}

\sphinxAtStartPar
While each of these networks (and many others) offers unique features and strengths, they operate in isolation. The lack of a common language or protocol for these chains to communicate poses challenges, especially for applications that could benefit from features spread across multiple chains.


\subsection{4. \sphinxstylestrong{Mechanisms and Methods for Interoperability}}
\label{\detokenize{Interoperability/Cross-Chain Interoperability:mechanisms-and-methods-for-interoperability}}
\sphinxAtStartPar
The quest for seamless communication between blockchains has led to the development of various mechanisms and methods. Here’s a closer look:
\begin{itemize}
\item {} 
\sphinxAtStartPar
\sphinxstylestrong{Bridges}: These are protocols designed to facilitate communication between two blockchains, allowing for the transfer of tokens and data. For instance, the “Wrapped Bitcoin” (wBTC) on Ethereum is a representation of Bitcoin, made possible through a bridge.

\item {} 
\sphinxAtStartPar
\sphinxstylestrong{Relay Chains and Parachains}: Popularized by Polkadot, this model uses a central relay chain to which multiple parachains (individual blockchains) connect. The relay chain is responsible for the network’s shared security, consensus, and cross\sphinxhyphen{}chain interoperability.

\item {} 
\sphinxAtStartPar
\sphinxstylestrong{Hub\sphinxhyphen{}and\sphinxhyphen{}Spoke}: Cosmos employs this model, where the central Cosmos Hub connects to various blockchains (termed zones). The Inter\sphinxhyphen{}Blockchain Communication (IBC) protocol facilitates the communication between the hub and its zones.

\item {} 
\sphinxAtStartPar
\sphinxstylestrong{Oracles}: While primarily known for feeding external data to blockchains, oracles like Chainlink can also serve as data bridges between chains, allowing smart contracts on one blockchain to receive data from another.

\item {} 
\sphinxAtStartPar
\sphinxstylestrong{Wrapped Tokens}: These are tokens from one blockchain represented on another. For example, wBTC on Ethereum represents Bitcoin, allowing BTC to be used within Ethereum’s DeFi ecosystem.

\item {} 
\sphinxAtStartPar
\sphinxstylestrong{Interledger Protocols}: These aim to create a standardized protocol for payments across chains, ensuring seamless asset transfers.

\end{itemize}


\subsection{5. \sphinxstylestrong{Leading Innovators in Interoperability}}
\label{\detokenize{Interoperability/Cross-Chain Interoperability:leading-innovators-in-interoperability}}
\sphinxAtStartPar
Several trailblazing projects are at the forefront of cross\sphinxhyphen{}chain solutions:
\begin{itemize}
\item {} 
\sphinxAtStartPar
\sphinxstylestrong{Polkadot}: Spearheaded by Dr. Gavin Wood, one of Ethereum’s co\sphinxhyphen{}founders, Polkadot focuses on enabling different blockchains to transfer messages and value in a trust\sphinxhyphen{}free fashion. Its relay chain and parachain model is a novel approach to scalability and interoperability.

\item {} 
\sphinxAtStartPar
\sphinxstylestrong{Cosmos}: With its vision of an “Internet of Blockchains”, Cosmos aims to create a network of blockchains able to communicate with one another. Its IBC protocol is a significant step towards achieving this vision.

\item {} 
\sphinxAtStartPar
\sphinxstylestrong{Chainlink}: Beyond its primary role as a decentralized oracle network, Chainlink is vital for cross\sphinxhyphen{}chain data sharing, ensuring that smart contracts on one chain can access data from another.

\end{itemize}


\subsection{6. \sphinxstylestrong{Challenges in Achieving Interoperability}}
\label{\detokenize{Interoperability/Cross-Chain Interoperability:challenges-in-achieving-interoperability}}
\sphinxAtStartPar
While the promise of cross\sphinxhyphen{}chain interoperability is tantalizing, the journey is riddled with challenges:
\begin{itemize}
\item {} 
\sphinxAtStartPar
\sphinxstylestrong{Security Concerns}: As assets move between chains, ensuring their security becomes paramount. The bridge or protocol facilitating the transfer must be foolproof to prevent potential exploits.

\item {} 
\sphinxAtStartPar
\sphinxstylestrong{Standardization}: With a plethora of blockchains, each with its architecture and consensus mechanism, developing a universal standard for communication is daunting.

\item {} 
\sphinxAtStartPar
\sphinxstylestrong{Scalability}: As interoperability solutions gain traction, the volume of cross\sphinxhyphen{}chain transactions will surge. Handling this increased traffic without causing congestion is a significant challenge.

\item {} 
\sphinxAtStartPar
\sphinxstylestrong{Economic Models}: Harmonizing the varying fee structures, staking mechanisms, and incentive models of different blockchains is a complex task.

\end{itemize}


\subsection{7. \sphinxstylestrong{Potential Solutions to Challenges}}
\label{\detokenize{Interoperability/Cross-Chain Interoperability:potential-solutions-to-challenges}}
\sphinxAtStartPar
The challenges of cross\sphinxhyphen{}chain interoperability, while significant, are not insurmountable. Innovators in the space are already devising solutions:
\begin{itemize}
\item {} 
\sphinxAtStartPar
\sphinxstylestrong{Layer 2 Solutions}: By moving some operations off the main blockchain (Layer 1) and onto a secondary layer, Layer 2 solutions can significantly increase transaction throughput and reduce fees. Examples include Ethereum’s Rollups and Bitcoin’s Lightning Network.

\item {} 
\sphinxAtStartPar
\sphinxstylestrong{Off\sphinxhyphen{}chain Computations}: By performing certain operations off\sphinxhyphen{}chain and only recording the final state on\sphinxhyphen{}chain, it’s possible to achieve higher efficiency without compromising security.

\item {} 
\sphinxAtStartPar
\sphinxstylestrong{Sharding}: This involves splitting a blockchain into multiple smaller chains (shards) that run concurrently. Each shard handles a fraction of the network’s transactions, leading to increased scalability.

\item {} 
\sphinxAtStartPar
\sphinxstylestrong{Witnesses and Validators}: To ensure security during cross\sphinxhyphen{}chain operations, a set of nodes (witnesses or validators) can be used to monitor and validate transactions. Their consensus can help prevent malicious activities.

\item {} 
\sphinxAtStartPar
\sphinxstylestrong{Protocols for Standardization}: Initiatives like the Interledger Protocol aim to create a standardized protocol for payments across chains, ensuring seamless asset transfers.

\end{itemize}


\subsection{8. \sphinxstylestrong{Implications of a Unified Blockchain Ecosystem}}
\label{\detokenize{Interoperability/Cross-Chain Interoperability:implications-of-a-unified-blockchain-ecosystem}}
\sphinxAtStartPar
The potential of a fully interoperable blockchain ecosystem is vast, with implications spanning various sectors:
\begin{itemize}
\item {} 
\sphinxAtStartPar
\sphinxstylestrong{Supply Chain}: Interoperability can enhance transparency and traceability across the supply chain, ensuring that goods are sourced ethically and reach their destination efficiently.

\item {} 
\sphinxAtStartPar
\sphinxstylestrong{Finance}: In the realm of DeFi, interoperability can lead to more integrated financial systems, where assets and data flow seamlessly across platforms, enhancing liquidity and financial products’ versatility.

\item {} 
\sphinxAtStartPar
\sphinxstylestrong{Identity Verification}: A unified blockchain ecosystem can streamline digital identity verification processes, ensuring that a person’s identity is consistent and verifiable across platforms.

\item {} 
\sphinxAtStartPar
\sphinxstylestrong{Gaming}: Imagine interconnected gaming universes where assets and characters from one game can be used in another. Interoperability can make this a reality, leading to more immersive gaming experiences.

\end{itemize}


\subsection{9. \sphinxstylestrong{Future Prospects and Developments}}
\label{\detokenize{Interoperability/Cross-Chain Interoperability:future-prospects-and-developments}}
\sphinxAtStartPar
The realm of cross\sphinxhyphen{}chain interoperability is still in its nascent stages, but the trajectory is promising. Here’s a glimpse into the future:
\begin{itemize}
\item {} 
\sphinxAtStartPar
\sphinxstylestrong{Emerging Projects}: As the need for interoperability becomes more pronounced, we can expect a surge in projects focusing on innovative solutions. These projects will not only address the current challenges but also anticipate future obstacles.

\item {} 
\sphinxAtStartPar
\sphinxstylestrong{Interconnected dApps}: Decentralized applications will no longer be confined to a single blockchain. We’ll see dApps that leverage features from multiple chains, offering users a richer and more seamless experience.

\item {} 
\sphinxAtStartPar
\sphinxstylestrong{Mainstream Adoption}: As interoperability solutions mature, traditional sectors that have been hesitant to adopt blockchain technology might find it more palatable. The ability to communicate with multiple chains can make integration smoother for these sectors.

\item {} 
\sphinxAtStartPar
\sphinxstylestrong{Regulatory Evolution}: With the growth of cross\sphinxhyphen{}chain operations, regulatory frameworks will need to evolve. We might see global standards and protocols being developed to ensure that cross\sphinxhyphen{}chain activities are secure, transparent, and compliant.

\end{itemize}


\subsection{10. \sphinxstylestrong{Conclusion}}
\label{\detokenize{Interoperability/Cross-Chain Interoperability:conclusion}}
\sphinxAtStartPar
Cross\sphinxhyphen{}chain interoperability is more than just a technical challenge; it’s a vision of a unified, decentralized digital world. A world where blockchains, despite their individual strengths and features, come together to create a cohesive ecosystem. The journey towards this vision is filled with challenges, but with the relentless spirit of innovation in the blockchain community, the future is bright.

\sphinxAtStartPar
The potential of a world where blockchains can “talk” to each other is vast. From finance to gaming, the implications are transformative. As we stand on the cusp of this revolution, it’s an exciting time to be part of the blockchain community.



\sphinxstepscope


\section{Scam Detection on Ethereum}
\label{\detokenize{SDE/ScamDetec:scam-detection-on-ethereum}}\label{\detokenize{SDE/ScamDetec::doc}}


\sphinxAtStartPar
\sphinxstylestrong{Academic Insight}



\begin{sphinxadmonition}{note}{Key Insights}
\begin{itemize}
\item {} 
\sphinxAtStartPar
Ethereum harbors an array of scams, spanning phishing, ponzi schemes, and pump\sphinxhyphen{}and\sphinxhyphen{}dumps.

\item {} 
\sphinxAtStartPar
Identifying and countering these scams pose intricate challenges, encompassing systematic analysis, accurate feature extraction, and prompt detection.

\item {} 
\sphinxAtStartPar
Novel strategies emerge: trans2vec leveraging transaction to tackle phishing, SADPonzi deciphering bytecode for Ponzi schemes, and LightGBM incorporating N\sphinxhyphen{}gram features to foresee early\sphinxhyphen{}stage honeypots.

\item {} 
\sphinxAtStartPar
Predicting rug pulls involves assessing pool state, token distribution of users, and forecasting coin being pumped relies on sophisticated market movement information.

\end{itemize}
\end{sphinxadmonition}


\subsection{Introduction}
\label{\detokenize{SDE/ScamDetec:introduction}}
\sphinxAtStartPar
Ethereum is famous as the largest blockchain platform that supports smart contracts, which has become increasingly prosperous and has attracted investors from all over the world. However, due to its anonymity, Ethereum has become a hotbed for various kinds of fraudulent activities, such as phishing scams, Ponzi schemes, honeypot schemes, rug pull scams, pump\sphinxhyphen{}and\sphinxhyphen{}dump schemes, and so on, which pose a serious threat to trading security on Ethereum. From Chainalysis 2022 Crypto Crime Report {[}\hyperlink{cite.SDE/ScamDetec:id144}{Tea22}{]}, scams have been the largest form of cryptocurrency\sphinxhyphen{}based crime since 2017, leading to significant losses, which is shown in {[}\hyperref[\detokenize{SDE/ScamDetec:crypto-value}]{Fig.\@ \ref{\detokenize{SDE/ScamDetec:crypto-value}}}{]}. Therefore, it is imperative to protect investors from scams and create a secure trading ecosystem on Ethereum. In this science note, we delve into the current academic landscape surrounding scam detection on Ethereum and summarise detection techniques of five kinds of common scams.

\begin{figure}[htbp]
\centering
\capstart

\noindent\sphinxincludegraphics{{crypto_value}.png}
\caption{Total cryptocurrency value received by illicit address from 2017 to 2021.}\label{\detokenize{SDE/ScamDetec:crypto-value}}\end{figure}


\subsection{Challenges in Scam Detection on Ethereum}
\label{\detokenize{SDE/ScamDetec:challenges-in-scam-detection-on-ethereum}}
\sphinxAtStartPar
There are three main challenges to be addressed in scam detection on Ethereum :
\begin{itemize}
\item {} 
\sphinxAtStartPar
\sphinxstylestrong{How to systematically analyze scams:} Different types of scams may employ different methods and strategies, targeting different victims. Therefore, researchers need to collect and analyze a large amount of fraud case data to gain a deeper understanding of the characteristics and patterns of fraudulent behavior.

\item {} 
\sphinxAtStartPar
\sphinxstylestrong{How to extract effective features:} The performance of scam detection is closely related to the choice of extracted features. Since fraudulent behavior may exhibit subtle differences from normal behavior, it is necessary to select discriminative features to distinguish between the two.

\item {} 
\sphinxAtStartPar
\sphinxstylestrong{How to timely detect scams:} Detecting scams timely is crucial to minimize losses and prevent more ordinary investors from falling victim to fraud. When scams are identified or predicted early, authorities and exchanges can take appropriate actions to freeze suspicious accounts and block fraudulent transactions.

\end{itemize}


\subsection{Phishing Scam Detection}
\label{\detokenize{SDE/ScamDetec:phishing-scam-detection}}
\begin{sphinxShadowBox}
\sphinxstylesidebartitle{\sphinxstylestrong{Phishing Scam}}

\sphinxAtStartPar
Phishing scam is a common kind of scam where phishers attempt to obtain the sensitive information and money from accounts by disguising as a trustworthy entity.
\end{sphinxShadowBox}

\sphinxAtStartPar
Wu et al. {[}\hyperlink{cite.SDE/ScamDetec:id134}{WYL+22}{]} conducted the first investigation on phishing identification on Ethereum. Transaction information is very critical but cannot be captured by general random walk\sphinxhyphen{}based network embedding methods. Therefore, they proposed a novel network embedding algorithm called trans2vec to extract the features for subsequent phishing identification by taking the transaction amount and timestamp into consideration. They also assumed that a larger amount of value of the transaction implies a closer relationship between accounts and the later the transaction is, the greater the impact on the current relationship of the accounts.

\sphinxAtStartPar
New means of Non\sphinxhyphen{}Fungible Tokens (NFTs) phishing scams have emerged in the Ethereum ecosystem with the popularity of NFTs. Previous research lacks a systematic review and retrospective analysis of NFT phishing scams. Yang et al. {[}\hyperlink{cite.SDE/ScamDetec:id135}{YLW23}{]} collected 469 NFT phishing accounts and transactions and systematically summarized different patterns of NFT phishing scams, measuring the economic impacts and preferences of scammers. Interestingly, NFT phishers chose to transfer 57.5\% of NFTs to their accomplices for further operations, accompanied by signs of gang theft. Detecting NFT phishing gangs and exploring withdrawal methods could be a potential research direction in the future.


\subsection{Ponzi Scheme Detection}
\label{\detokenize{SDE/ScamDetec:ponzi-scheme-detection}}
\begin{sphinxShadowBox}
\sphinxstylesidebartitle{\sphinxstylestrong{Ponzi Scheme}}

\sphinxAtStartPar
Ponzi scheme is an investment fraud in which so\sphinxhyphen{}called returns are paid to existing investors through funds provided by new investors.
\end{sphinxShadowBox}

\sphinxAtStartPar
Existing methods to identify Ponzi smart contracts can be classified into two categories: transaction behavior\sphinxhyphen{}based detection {[}\hyperlink{cite.SDE/ScamDetec:id137}{JLTGG19}{]} and opcodes\sphinxhyphen{}based detection {[}\hyperlink{cite.SDE/ScamDetec:id138}{CZC+18}{]}. The former requires a considerable number of transactions to learn the behaviors, and the latter lacks interpretability. Chen et al. {[}\hyperlink{cite.SDE/ScamDetec:id136}{CLS+21}{]} proposed SADPonzi, a semantic\sphinxhyphen{}aware detection approach, which utilizes the symbolic execution technique to extract semantic information from contract bytecode and match it with four semantic patterns of Ponzi contracts, ultimately identifying Ponzi contracts. Experimental results indicate that SADPonzi outperforms all the existing techniques in terms of accuracy and robustness. However, the symbolic execution technique has a limitation in handling evasion methods which can lead to serious path explosion.


\subsection{Honeypot Scheme Detection}
\label{\detokenize{SDE/ScamDetec:honeypot-scheme-detection}}
\begin{sphinxShadowBox}
\sphinxstylesidebartitle{\sphinxstylestrong{Honeypot Scam}}

\sphinxAtStartPar
Honeypot Scheme is smart contracts intentionally designed with a flaw to attract victim attackers, who attempt to exploit it by sending funds. But the contracts fail to operate as expected, resulting in the loss of the investment.
\end{sphinxShadowBox}

\sphinxAtStartPar
To detect honeypot contracts early in their creation, Chen et al. {[}\hyperlink{cite.SDE/ScamDetec:id139}{CGC+20}{]} put forward a machine learning model for honeypot contracts detection based on N\sphinxhyphen{}gram features and LightGBM. They construct a series of N\sphinxhyphen{}Gram\sphinxhyphen{}based features and use a feature selection method to drop out those useless features. The model performs well in different imbalances of the data set. In the future, it is a potential way to combine the behavior of contracts’ creators and features of contracts to get a more accurate classification model for detecting honeypot contracts.


\subsection{Rug Pull Scam Detection}
\label{\detokenize{SDE/ScamDetec:rug-pull-scam-detection}}
\begin{sphinxShadowBox}
\sphinxstylesidebartitle{\sphinxstylestrong{Rug Pull Scam}}

\sphinxAtStartPar
Rug Pull Scam is a scam where developers abandon a project and take their investors’ money when enough investors rush into the project and exchange for the worthless tokens.
\end{sphinxShadowBox}

\sphinxAtStartPar
Xia et al. {[}\hyperlink{cite.SDE/ScamDetec:id140}{XWG+21}{]} are the first one to propose an accurate approach for flagging rug pull scams and the scam tokens on Uniswap based on a guilt\sphinxhyphen{}by\sphinxhyphen{}association heuristic and a machine\sphinxhyphen{}learning powered technique. The guilt\sphinxhyphen{}by\sphinxhyphen{}association heuristic technique helps to identify and expand
obvious scam tokens and scammers. Machine learning\sphinxhyphen{}based detection helps to identify more scammers and scam tokens based on transactions on Uniswap. Interestingly, they found thousands of collusion addresses to help carry out the scams in league with the scam token/pool creators. Four kinds of collusion addresses can be seen in {[}\hyperref[\detokenize{SDE/ScamDetec:collusion-address}]{Fig.\@ \ref{\detokenize{SDE/ScamDetec:collusion-address}}}{]}.

\sphinxAtStartPar
However, the method proposed by Xia et al. {[}\hyperlink{cite.SDE/ScamDetec:id140}{XWG+21}{]} is only effective for detecting scams accurately after they have been executed. Mazorra et al. {[}\hyperlink{cite.SDE/ScamDetec:id141}{MAD22}{]} designed an accurate automated rug pull detection to predict future rug pulls and scams using relevant features of the pool’s state and the token distribution among the users. They use the Herfindahl–Hirschman Index and clustering transaction coefficient as heuristics to measure the distribution of the token among the investors. Additionally, they feed these features to train XGBoost and FT\sphinxhyphen{}Transformer models, respectively, and predict tokens before the malicious maneuver.

\begin{figure}[htbp]
\centering
\capstart

\noindent\sphinxincludegraphics{{collusion_address}.png}
\caption{Four kinds of collusion addresses categorized based on their Uniswap transaction behaviors.}\label{\detokenize{SDE/ScamDetec:collusion-address}}\end{figure}


\subsection{Pump\sphinxhyphen{}and\sphinxhyphen{}Dump Scheme Detection}
\label{\detokenize{SDE/ScamDetec:pump-and-dump-scheme-detection}}
\begin{sphinxShadowBox}
\sphinxstylesidebartitle{\sphinxstylestrong{Pump\sphinxhyphen{}and\sphinxhyphen{}Dump Scheme}}

\sphinxAtStartPar
Pump\sphinxhyphen{}and\sphinxhyphen{}Dump Scheme is a form of price manipulation that involves artificially inflating an asset’s price before selling the cheaply purchased asset at a higher price. Once the assets are dumped, the price falls and investors lose money.
\end{sphinxShadowBox}

\sphinxAtStartPar
Telegram, with its relative anonymity, has fostered the organization of pump\sphinxhyphen{}and\sphinxhyphen{}dump activities by many people in channels. Xu et al. {[}\hyperlink{cite.SDE/ScamDetec:id143}{XL19}{]} analyzed features of pumped coins and market movements of coins before, during, and after pump and dump. They also built a predictive random forest model and a generalized linear model able to predict the coin being pumped before the actual pump event by Telegram channels using the information of market movements. In addition, they proposed a simple but effective trading strategy that can be used in combination with the prediction models, leading to fewer people falling victim to market manipulation and more people trading strategically. Different from the work in {[}\hyperlink{cite.SDE/ScamDetec:id143}{XL19}{]}, La et al. {[}\hyperlink{cite.SDE/ScamDetec:id142}{LMMSS23}{]} built a machine learning model able to detect pump\sphinxhyphen{}and\sphinxhyphen{}dump schemes using the information of rush orders within 25 seconds from the moment it starts, instead of predicting it before it happens.


\subsection{Conclusion}
\label{\detokenize{SDE/ScamDetec:conclusion}}
\sphinxAtStartPar
The popularity of Ethereum has attracted a surge of fraudulent activities, posing serious risks to users. Detecting and preventing scams on Ethereum presents several challenges, ongoing research and innovative approaches are making significant progress in scam detection. In this science note, we delve into the current academic landscape surrounding scam detection on Ethereum and summarise detection techniques of five common types of scams. Scam detection on Ethereum remains a worthwhile and pressing challenge in the field. Through ongoing exploration and innovation, we can collectively strive to build a more secure and trustworthy cryptocurrency trading ecosystem.




\subsection{References}
\label{\detokenize{SDE/ScamDetec:references}}
\sphinxstepscope


\chapter{Industry Perspective}
\label{\detokenize{STRUCTURE/industry_2:industry-perspective}}\label{\detokenize{STRUCTURE/industry_2::doc}}
\begin{sphinxuseclass}{sd-container-fluid}
\begin{sphinxuseclass}{sd-sphinx-override}
\begin{sphinxuseclass}{sd-mb-4}
\begin{sphinxuseclass}{sd-row}
\begin{sphinxuseclass}{sd-row-cols-1}
\begin{sphinxuseclass}{sd-row-cols-xs-1}
\begin{sphinxuseclass}{sd-row-cols-sm-2}
\begin{sphinxuseclass}{sd-row-cols-md-2}
\begin{sphinxuseclass}{sd-row-cols-lg-2}
\begin{sphinxuseclass}{sd-g-2}
\begin{sphinxuseclass}{sd-g-xs-2}
\begin{sphinxuseclass}{sd-g-sm-2}
\begin{sphinxuseclass}{sd-g-md-2}
\begin{sphinxuseclass}{sd-g-lg-2}
\begin{sphinxuseclass}{sd-col}
\begin{sphinxuseclass}{sd-d-flex-row}
\begin{sphinxuseclass}{sd-mt-3}
\begin{sphinxuseclass}{sd-mb-0}
\begin{sphinxuseclass}{sd-ml-0}
\begin{sphinxuseclass}{sd-mr-0}
\begin{sphinxuseclass}{sd-card}
\begin{sphinxuseclass}{sd-sphinx-override}
\begin{sphinxuseclass}{sd-w-100}
\begin{sphinxuseclass}{sd-shadow-md}
\begin{sphinxuseclass}{sd-card-hover}
\begin{sphinxuseclass}{sd-text-center}
\begin{sphinxuseclass}{sd-card-body}
\begin{sphinxuseclass}{sd-card-title}
\begin{sphinxuseclass}{sd-font-weight-bold}Understanding Hedera Services: An Overview of Transaction Types on the Hashgraph Network
\end{sphinxuseclass}
\end{sphinxuseclass}


\end{sphinxuseclass}\sphinxhref{https://dlt-science.github.io/science-notes/HED/hed.html}{}
\end{sphinxuseclass}
\end{sphinxuseclass}
\end{sphinxuseclass}
\end{sphinxuseclass}
\end{sphinxuseclass}
\end{sphinxuseclass}
\end{sphinxuseclass}
\end{sphinxuseclass}
\end{sphinxuseclass}
\end{sphinxuseclass}
\end{sphinxuseclass}
\end{sphinxuseclass}
\begin{sphinxuseclass}{sd-col}
\begin{sphinxuseclass}{sd-d-flex-row}
\begin{sphinxuseclass}{sd-mt-3}
\begin{sphinxuseclass}{sd-mb-0}
\begin{sphinxuseclass}{sd-ml-0}
\begin{sphinxuseclass}{sd-mr-0}
\begin{sphinxuseclass}{sd-card}
\begin{sphinxuseclass}{sd-sphinx-override}
\begin{sphinxuseclass}{sd-w-100}
\begin{sphinxuseclass}{sd-shadow-md}
\begin{sphinxuseclass}{sd-card-hover}
\begin{sphinxuseclass}{sd-text-center}
\begin{sphinxuseclass}{sd-card-body}
\begin{sphinxuseclass}{sd-card-title}
\begin{sphinxuseclass}{sd-font-weight-bold}Opportunities and Risks for Traditional Insurance Companies and Banks with DeFi Business Models
\end{sphinxuseclass}
\end{sphinxuseclass}




\end{sphinxuseclass}\sphinxhref{https://dlt-science.github.io/science-notes/BOER/boer.html}{}
\end{sphinxuseclass}
\end{sphinxuseclass}
\end{sphinxuseclass}
\end{sphinxuseclass}
\end{sphinxuseclass}
\end{sphinxuseclass}
\end{sphinxuseclass}
\end{sphinxuseclass}
\end{sphinxuseclass}
\end{sphinxuseclass}
\end{sphinxuseclass}
\end{sphinxuseclass}
\begin{sphinxuseclass}{sd-col}
\begin{sphinxuseclass}{sd-d-flex-row}
\begin{sphinxuseclass}{sd-mt-3}
\begin{sphinxuseclass}{sd-mb-0}
\begin{sphinxuseclass}{sd-ml-0}
\begin{sphinxuseclass}{sd-mr-0}
\begin{sphinxuseclass}{sd-card}
\begin{sphinxuseclass}{sd-sphinx-override}
\begin{sphinxuseclass}{sd-w-100}
\begin{sphinxuseclass}{sd-shadow-md}
\begin{sphinxuseclass}{sd-card-hover}
\begin{sphinxuseclass}{sd-text-center}
\begin{sphinxuseclass}{sd-card-body}
\begin{sphinxuseclass}{sd-card-title}
\begin{sphinxuseclass}{sd-font-weight-bold}Crypto and Digital Assets Risk Management: Navigating Opportunities and Challenges
\end{sphinxuseclass}
\end{sphinxuseclass}




\end{sphinxuseclass}\sphinxhref{https://dlt-science.github.io/science-notes/ARM/arm.html}{}
\end{sphinxuseclass}
\end{sphinxuseclass}
\end{sphinxuseclass}
\end{sphinxuseclass}
\end{sphinxuseclass}
\end{sphinxuseclass}
\end{sphinxuseclass}
\end{sphinxuseclass}
\end{sphinxuseclass}
\end{sphinxuseclass}
\end{sphinxuseclass}
\end{sphinxuseclass}
\end{sphinxuseclass}
\end{sphinxuseclass}
\end{sphinxuseclass}
\end{sphinxuseclass}
\end{sphinxuseclass}
\end{sphinxuseclass}
\end{sphinxuseclass}
\end{sphinxuseclass}
\end{sphinxuseclass}
\end{sphinxuseclass}
\end{sphinxuseclass}
\end{sphinxuseclass}
\end{sphinxuseclass}
\end{sphinxuseclass}
\sphinxstepscope


\section{Understanding Hedera Services: An Overview of Transaction Types on the Hashgraph Network}
\label{\detokenize{HED/hed:understanding-hedera-services-an-overview-of-transaction-types-on-the-hashgraph-network}}\label{\detokenize{HED/hed::doc}}


\sphinxAtStartPar
\sphinxstylestrong{Industry Perspective}

\begin{sphinxadmonition}{note}{Key Insights}
\begin{itemize}
\item {} 
\sphinxAtStartPar
Hedera’s unique protocol and algorithm ensure a fast, secure, and fair platform for real\sphinxhyphen{}time applications and services.

\item {} 
\sphinxAtStartPar
Hedera’s native cryptocurrency, HBAR, bolsters network security and powers transactions at low, stable fees.

\item {} 
\sphinxAtStartPar
Hedera’s USD\sphinxhyphen{}fixed transaction fees, tailored for network operations, counteract HBAR price fluctuations to provide a stable and predictable cost framework for users.

\item {} 
\sphinxAtStartPar
The Hedera Consensus Service streamlines consensus processes, fostering trust and decentralisation for various applications.

\item {} 
\sphinxAtStartPar
The Hedera Token Service streamlines tokenisation, supporting secure asset conversion, and promoting interoperability.

\end{itemize}
\end{sphinxadmonition}


\subsection{Introduction}
\label{\detokenize{HED/hed:introduction}}
\begin{sphinxShadowBox}
\sphinxstylesidebartitle{\sphinxstylestrong{Gossip about gossip}}

\sphinxAtStartPar
The “Gossip about Gossip” protocol is a fundamental part of the consensus algorithm, which disseminates transactions across network nodes through random information exchange, similar to social gossip. Beyond transaction data, this protocol also conveys the timeline and pathway of data propagation, enabling an efficient consensus on transaction order without a laborious proof\sphinxhyphen{}of\sphinxhyphen{}work process.
\end{sphinxShadowBox}



\sphinxAtStartPar
Hedera is a distributed ledger technology designed to offer a secure, fair, and fast platform for a new generation of real\sphinxhyphen{}time applications and services {[}\hyperlink{cite.HED/hed:id122}{HH23}{]}. In the Hedera network, a user initiates a transaction, which is then quickly disseminated among nodes through an efficient “gossip” protocol. Nodes gossip messages to each other about transactions randomly. Consensus on transactions is achieved independently by nodes using a virtual voting algorithm, which calculates a consensus timestamp based on the median timestamp when the majority of nodes received the transaction. This mechanism ensures transaction fairness and security, as no single node can significantly manipulate the order, thereby providing resilience against malicious activities.

\sphinxAtStartPar
The transaction recording process using Hedera and its benefits can be seen in \hyperref[\detokenize{HED/hed:hed-diagram}]{Fig.\@ \ref{\detokenize{HED/hed:hed-diagram}}}.

\begin{figure}[htbp]
\centering
\capstart

\noindent\sphinxincludegraphics[width=733\sphinxpxdimen,height=400\sphinxpxdimen]{{hashgraph}.png}
\caption{Hedera Transaction Recording Process \& Its Key Benefits.}\label{\detokenize{HED/hed:hed-diagram}}\end{figure}

\sphinxAtStartPar
Since Hedera launched in August 2018, it offers unique transactional capabilities through its native HBAR cryptocurrency, used both for network security and as fuel for network services. As a proof\sphinxhyphen{}of\sphinxhyphen{}stake network, HBARs help safeguard the network by representing voting power; thus, wider distribution of HBARs prevents potential attacks by making it prohibitively expensive for a malicious entity to control one\sphinxhyphen{}third of the coins. In addition, HBARs serve as fuel for network services, compensating nodes for providing computing resources, and ensuring low, stable transaction fees. For example, a cryptocurrency transfer currently costs \$0.0001 (paid in HBARs) {[}\hyperlink{cite.HED/hed:id123}{BGT19}{]}.


\subsection{Hedera Cryptocurrency Service}
\label{\detokenize{HED/hed:hedera-cryptocurrency-service}}
\sphinxAtStartPar
Hedera provides two distinct services related to digital assets: the Hedera Cryptocurrency Service and Token Service. The Cryptocurrency Service pertains specifically to the use of HBAR for transactions and fees on the network, while the Token Service provides a platform for users to create and manage their own custom tokens. Hedera’s cryptocurrency is engineered for speed, resulting in minimal network fees and enabling feasible micro\sphinxhyphen{}transactions. When Hedera is fully operational, every user will have the capability to manage a network node and receive cryptocurrency payments for this contribution. To create an account, all that is needed is to generate a key pair; there is no need for a linked name or address. However, users are given the flexibility to link hashes of identity credentials from any selected third\sphinxhyphen{}party certificate or identity authority. This function is purposefully constructed to aid in adhering to legal standards in regions enforcing Know Your Customer (KYC) or Anti\sphinxhyphen{}Money Laundering (AML) regulations. Further details can be found in the Regulatory Compliance section {[}\hyperlink{cite.HED/hed:id128}{BHM20}{]}.


\subsection{Fees Associated with Hedera Transactions}
\label{\detokenize{HED/hed:fees-associated-with-hedera-transactions}}
\sphinxAtStartPar
Hedera’s network fees are designed for specific network operations. The fees are payable in HBAR, but are also fixed in USD for stability.

\begin{sphinxadmonition}{note}{Note:}
\sphinxAtStartPar
\sphinxstylestrong{HBAR Denominations and Abbreviations}

\sphinxAtStartPar
Hedera denominates HBAR into various units:

\begin{sphinxVerbatim}[commandchars=\\\{\}]
1 gigabar (Gℏ) = 1 billion HBAR
1 megabar (Mℏ) = 1 million HBAR 
1 kilobar (Kℏ) = 1,000 HBAR 
1 hbar (ℏ) = 1 HBAR 
1 millibar (mℏ) = 0.001 HBAR 
1 microbar (μℏ) = 0.000001 HBAR
1 tinybar (tℏ) = 0.00000001 HBAR.
\end{sphinxVerbatim}
\end{sphinxadmonition}

\sphinxAtStartPar
Fee structures for various operations are as follows:
\begin{itemize}
\item {} 
\sphinxAtStartPar
Cryptocurrency Service: The cost for creating a crypto account is \$0.05, for auto renewing an account is \$0.00022, and for transferring cryptocurrency is \$0.0001, amongst others.

\item {} 
\sphinxAtStartPar
Consensus Service: Fees for creating a topic on the Consensus Service is \$0.01, for updating a topic is \$0.00022, and for submitting a message is \$0.0001, etc.

\item {} 
\sphinxAtStartPar
Token Service: The cost to create a token is \$1.00, to update a token is \$0.001, and to associate a token with an account is \$0.05, amongst others.

\item {} 
\sphinxAtStartPar
File Service: The fee for creating a file is \$0.05, updating a file is \$0.05, and deleting a file is \$0.007, etc.

\item {} 
\sphinxAtStartPar
Smart Contract Service: Fees for creating a contract are \$1.0, for updating a contract are \$0.026, and for making a contract call are \$0.05, etc.

\end{itemize}

\sphinxAtStartPar
Exact service fees will be visible once finalized through the \sphinxhref{https://docs.hedera.com/hedera/networks/mainnet/fees}{pricing calculator}.


\subsection{Hedera Consensus Service}
\label{\detokenize{HED/hed:hedera-consensus-service}}
\sphinxAtStartPar
The Hedera Consensus Service (HCS) is an essential component of the Hedera network that provides a decentralised, secure, and verifiable log of events. It facilitates agreement on transaction order and timing across diverse applications, ranging from supply chains to multiplayer gaming. More than just a transactional ledger, HCS revolutionises the blockchain ecosystem by offering swift, fair, and decentralised consensus. Utilising the hashgraph consensus algorithm, HCS expedites transaction settlements, fostering transparency with a timestamped process. It not only bolsters efficiency and the trust model of private networks but also significantly reduces operational costs. Moreover, it promotes a collaborative environment for interconnected applications, paving the way for the next wave of decentralised applications{[}\hyperlink{cite.HED/hed:id123}{BGT19}{]}.


\subsubsection{HCS Architecture and Configuration}
\label{\detokenize{HED/hed:hcs-architecture-and-configuration}}
\sphinxAtStartPar
HCS is made accessible through various SDKs and the Hedera API (HAPI). It processes byte string messages from client applications tied to unique topics. These messages, carrying transactional details, are processed against a fee in HBARs, Hedera’s native currency. Upon successful processing, the Hedera ledger returns a record with consensus details, timestamps, sequence numbers, and running hashes reflecting previous messages. To configure this service, organisations set up mirror nodes, program applications, and define unique keys and topics for transactions to configure HCS. After verification and confirmation of the transaction fee payment, the transaction information is disseminated across the network to establish a consensus timestamp. Mirror nodes receive this information, facilitating the creation of state proofs and further transaction processing. This configuration ensures robust record distribution and real\sphinxhyphen{}time auditing, fostering transparency and immediate validation of transaction order and accuracy.


\subsection{Hedera Token Service}
\label{\detokenize{HED/hed:hedera-token-service}}
\sphinxAtStartPar
The Hedera Token Service (HTS) enables native token creation on the Hedera platform, storing information on the public Hedera ledger and offering pseudonymous privacy. This model is governed by the Hedera Governing Council and allows for limited customisation {[}\hyperlink{cite.HED/hed:id124}{HH20}{]}. HTS makes token deployment straightforward and cost\sphinxhyphen{}effective, without the need for additional infrastructure. It can support high throughput applications with thousands of transactions per second, achieving transaction finality in 3\sphinxhyphen{}5 seconds. The interoperability of tokens across the Hedera ecosystem and decentralised trust model ensures transparent, verifiable transactions.


\subsubsection{Applications of Hedera’s Tokenisation Model}
\label{\detokenize{HED/hed:applications-of-hedera-s-tokenisation-model}}
\sphinxAtStartPar
Hedera’s tokenisation model can support various token use cases {[}\hyperlink{cite.HED/hed:id124}{HH20}{]}. In financial services, it can facilitate efficient trading and settlement of assets like bonds, stocks, or commodities. Tokens can also track physical goods in supply chains, enable fractional ownership in real estate, represent unique art pieces as Non\sphinxhyphen{}fungible tokens, and form the backbone of Decentralised Finance. Other applications include tokenising in\sphinxhyphen{}game assets, loyalty rewards, and personal identities for enhanced security and user privacy.


\subsection{Hedera Smart Contract Service}
\label{\detokenize{HED/hed:hedera-smart-contract-service}}
\sphinxAtStartPar
Hedera’s Smart Contract Service revolutionises the world of blockchain programming by introducing exceptional features that enhance performance, reduce costs, ensure security and fairness, and promote interoperability with Ethereum. The service allows the development of smart contracts using Solidity, a common language in Ethereum, simplifying the transition for developers familiar with Ethereum’s ecosystem {[}\hyperlink{cite.HED/hed:id125}{Cla22}{]}. The Besu EVM, tailored for the Hedera network and hashgraph consensus, enables high\sphinxhyphen{}speed transactions, predictable low fees, a negative carbon footprint, and exceptional performance with 15 million gas per second {[}\hyperlink{cite.HED/hed:id126}{Hed}{]}. By leveraging the hashgraph consensus algorithm, the service offers rapid transaction finality and optimises contract execution, surpassing the capabilities of traditional block\sphinxhyphen{}based systems. It is designed to maintain low and predictable costs, significantly benefiting developers compared to Ethereum’s fluctuating fees. The platform’s robust security features and the governance of an esteemed council of industry leaders assure platform stability and continuous improvements. Furthermore, the service supports easy migration of existing Ethereum contracts to Hedera, showcasing its adaptability and convenience for developers.


\sphinxstrong{See also:}
\nopagebreak


\sphinxAtStartPar
\\
The full documentation for the Smart Contract Service and a “Deploy Your First Smart Contract” tutorial \sphinxhref{https://docs.hedera.com/hedera/tutorials/smart-contracts/deploy-a-contract-using-the-hedera-token-service}{here}.
Additionally, a JavaScript code snippet is provided below for creating a very first smart contract transaction on Hedera.

\begin{sphinxVerbatim}[commandchars=\\\{\}]
\PYG{p}{\PYGZob{}}
\PYG{p}{...}\PYG{p}{.}
\PYG{c+c1}{//Create the transaction}

\PYG{k+kd}{const}\PYG{+w}{ }\PYG{n+nx}{transaction}\PYG{+w}{ }\PYG{o}{=}\PYG{+w}{ }\PYG{o+ow}{new}\PYG{+w}{ }\PYG{n+nx}{ContractCreateTransaction}\PYG{p}{(}\PYG{p}{)}

\PYG{+w}{    }\PYG{p}{.}\PYG{n+nx}{setGas}\PYG{p}{(}\PYG{l+m+mf}{100}\PYG{n+nx}{\PYGZus{}000\PYGZus{}000}\PYG{p}{)}
\PYG{+w}{    }\PYG{p}{.}\PYG{n+nx}{setBytecodeFileId}\PYG{p}{(}\PYG{n+nx}{bytecodeFileId}\PYG{p}{)}
\PYG{+w}{    }\PYG{p}{.}\PYG{n+nx}{setAdminKey}\PYG{p}{(}\PYG{n+nx}{adminKey}\PYG{p}{)}\PYG{p}{;}

\PYG{c+c1}{//Modify the default max transaction fee (default: 1 hbar)}

\PYG{k+kd}{const}\PYG{+w}{ }\PYG{n+nx}{modifyTransactionFee}\PYG{+w}{ }\PYG{o}{=}\PYG{+w}{ }\PYG{n+nx}{transaction}\PYG{p}{.}\PYG{n+nx}{setMaxTransactionFee}\PYG{p}{(}\PYG{o+ow}{new}\PYG{+w}{ }\PYG{n+nx}{Hbar}\PYG{p}{(}\PYG{l+m+mf}{16}\PYG{p}{)}\PYG{p}{)}\PYG{p}{;}

\PYG{c+c1}{//Sign the transaction with the client operator key and submit to a Hedera network}

\PYG{k+kd}{const}\PYG{+w}{ }\PYG{n+nx}{txResponse}\PYG{+w}{ }\PYG{o}{=}\PYG{+w}{ }\PYG{k}{await}\PYG{+w}{ }\PYG{n+nx}{modifyTransactionFee}\PYG{p}{.}\PYG{n+nx}{execute}\PYG{p}{(}\PYG{n+nx}{client}\PYG{p}{)}\PYG{p}{;}

\PYG{c+c1}{//Get the receipt of the transaction}

\PYG{k+kd}{const}\PYG{+w}{ }\PYG{n+nx}{receipt}\PYG{+w}{ }\PYG{o}{=}\PYG{+w}{ }\PYG{k}{await}\PYG{+w}{ }\PYG{n+nx}{txResponse}\PYG{p}{.}\PYG{n+nx}{getReceipt}\PYG{p}{(}\PYG{n+nx}{client}\PYG{p}{)}\PYG{p}{;}

\PYG{c+c1}{//Get the new contract ID}

\PYG{k+kd}{const}\PYG{+w}{ }\PYG{n+nx}{newContractId}\PYG{+w}{ }\PYG{o}{=}\PYG{+w}{ }\PYG{n+nx}{receipt}\PYG{p}{.}\PYG{n+nx}{contractId}\PYG{p}{;}

\PYG{n+nx}{console}\PYG{p}{.}\PYG{n+nx}{log}\PYG{p}{(}\PYG{l+s+s2}{\PYGZdq{}The new contract ID is \PYGZdq{}}\PYG{+w}{ }\PYG{o}{+}\PYG{+w}{ }\PYG{n+nx}{newContractId}\PYG{p}{)}\PYG{p}{;}
\PYG{p}{...}\PYG{p}{.}
\PYG{p}{\PYGZcb{}}
\end{sphinxVerbatim}




\subsection{Hedera File Service}
\label{\detokenize{HED/hed:hedera-file-service}}
\begin{sphinxShadowBox}
\sphinxstylesidebartitle{\sphinxstylestrong{Merkle Directed Acyclic Graph}}

\sphinxAtStartPar
Merkle DAG is a unique form of DAG where each node is identified by a cryptographic hash, composed of the node’s content and the identifiers of its children nodes. They are self\sphinxhyphen{}verifying structures, immutable by nature, and constructed from the leaves up, meaning children nodes are added before their parents, with each node essentially serving as the root of its own sub\sphinxhyphen{}Merkle DAG.
\end{sphinxShadowBox}

\sphinxAtStartPar
Hedera’s File Service provides a resilient, secure, and efficient system for data storage in a decentralised environment. It functions like a transaction graph, processing data in parallel and storing files across all network nodes in Merkle Trees and Merkle Directed Acyclic Graphs, ensuring tamper\sphinxhyphen{}proof, regionally accessible, and 100\% available data. A unique feature is the provision of proof\sphinxhyphen{}of\sphinxhyphen{}deletion, allowing businesses to comply with General Data Protection Regulations (GDPR) requirements. Files in the system have a set expiration date and are deleted automatically, while the storage service costs are based on the file size and the desired storage duration. Furthermore, Hedera offers controlled mutability via WACL (WriteAccess Control) keys, providing flexible data management and ensuring consensus for any changes. Transactions on Hedera are limited to 4KB, ensuring efficiency, although larger files can be accommodated through the appending of additional data. The service supports various transactions including creating, appending, deleting, and updating files, offering comprehensive and flexible options for developers and users. In essence, Hedera’s File Service is a robust, secure, and efficient solution for decentralised data storage, embodying the low\sphinxhyphen{}cost and high\sphinxhyphen{}performance advantages of the platform {[}\hyperlink{cite.HED/hed:id127}{Won19}{]}.

\sphinxAtStartPar
The architecture of Hedera’s core services can be seen in \hyperref[\detokenize{HED/hed:hed-service-diagram}]{Fig.\@ \ref{\detokenize{HED/hed:hed-service-diagram}}}.

\begin{figure}[htbp]
\centering
\capstart

\noindent\sphinxincludegraphics[width=700\sphinxpxdimen,height=450\sphinxpxdimen]{{servicesarchitecture}.png}
\caption{Architecture of Hedera’s core service}\label{\detokenize{HED/hed:hed-service-diagram}}\end{figure}


\subsection{Conclusion}
\label{\detokenize{HED/hed:conclusion}}


\sphinxAtStartPar
Hedera has emerged as a pioneer in the realm of blockchain and Distributed Ledger Technology (DLT), leveraging unique transactional capabilities. It has significantly transformed the DLT landscape through its core services such as the Hedera Cryptocurrency Service, Hedera Consensus Service, Hedera Token Service, Smart Contract Service, and File Service, altering the handling of consensus and tokenisation. The advanced transactional capabilities of Hedera, coupled with its suite of innovative services, position it as a substantial disruptor in the blockchain and DLT arena. Its commitment to cost\sphinxhyphen{}effectiveness, high performance, secure operations, and transparency has the potential to redefine how businesses and individuals interact with distributed ledger technology, thus paving the way for the next wave of decentralised applications.




\subsection{References}
\label{\detokenize{HED/hed:references}}
\sphinxstepscope


\section{Opportunities and Risks for Traditional Insurance Companies and Banks with DeFi Business Models}
\label{\detokenize{BOER/boer:opportunities-and-risks-for-traditional-insurance-companies-and-banks-with-defi-business-models}}\label{\detokenize{BOER/boer::doc}}
\sphinxAtStartPar
\sphinxstylestrong{Industry Perspective}

\sphinxAtStartPar
\sphinxstylestrong{Disclaimer:} The views and opinions expressed in this article are solely those of the author




\subsection{Introduction}
\label{\detokenize{BOER/boer:introduction}}
\sphinxAtStartPar
The decentralized finance (DeFi) ecosystem, with its innovative models and frameworks, has introduced disruptive business models that challenge the traditional financial landscape. This article aims to analyze this challenge in detail, exploring how traditional players in finance can leverage the hidden and visible value within DeFi business models. To enhance your understanding of this topic and to provide you with a better understanding of this article, I recommend reviewing the following articles “\sphinxstyleemphasis{Short Survey on Business Models of Decentralized Finance}” by Jiahua Xu and Teng Andrea Xu and “\sphinxstyleemphasis{DeFi vs TradFi: Valuation Using Multiples and Discounted Cash Flow}” by Teng Andrea Xu, Jiahua Xu, Kristof Lommers.


\subsection{Business Ideas}
\label{\detokenize{BOER/boer:business-ideas}}
\sphinxAtStartPar
Let’s begin with a foundational overview, drawing insights from the aforementioned articles.

\sphinxAtStartPar
Within the context of Decentralized Finance (DeFi) Business Models among the prominent DeFi business models are Protocols for Loanable Funds (PLFs), Decentralized Exchanges (DEXs), and Yield Aggregators. This section will delve into these DeFi models and discuss their potential adoption by traditional insurance companies and banks. Any additional details not covered in this article can be further explored in the aforementioned articles.

\sphinxAtStartPar
\sphinxstylestrong{a) Protocols for Loanable Funds (PLFs)}:

\sphinxAtStartPar
PLFs have revolutionized lending and borrowing within the financial sector. These protocols facilitate access to loans and cryptocurrency lending without intermediaries. Revenues are generated through interest rates, shared between the protocols and lenders.

\sphinxAtStartPar
\sphinxstylestrong{b) Decentralized Exchanges (DEXs)}:

\sphinxAtStartPar
DEXs enable peer\sphinxhyphen{}to\sphinxhyphen{}peer cryptocurrency trading, eliminating the need for intermediaries. These platforms earn revenue by charging trading fees, contributing to the protocol’s treasury.

\sphinxAtStartPar
\sphinxstylestrong{c) Yield Aggregators}:

\sphinxAtStartPar
Yield Aggregators merge strategies to optimize investor returns. Investors deposit funds into “Vaults,” implementing diverse strategies for profitability. Yield Aggregators levy commission fees on strategy profits.


\subsection{Resolving the Paradox}
\label{\detokenize{BOER/boer:resolving-the-paradox}}
\sphinxAtStartPar
The question arises: Is it paradoxical for traditional insurance companies and banks, champions of centralized finance, to embrace DeFi models to extract additional value while upholding their established positions? However, as the financial landscape evolves, exploring innovative opportunities becomes essential to maintain profitability and stay ahead in the highly competitive and regulated financial sector. To be noted that profits in the financial industry are significantly influenced by Central Bank decisions, particularly interest rate fluctuations. \sphinxstylestrong{By adopting decentralized finance business models, though seemingly paradoxical at first glance, financial institutions can stabilize revenues and partially free themselves from Central Bank decisions.} With this context in mind, let’s explore potential business opportunities and risks for traditional banks and insurance companies in adopting each DeFi business model.

\sphinxAtStartPar
\sphinxstylestrong{a) PLFs}:
\begin{itemize}
\item {} 
\sphinxAtStartPar
\sphinxstylestrong{Business Opportunities for Traditional Banks}:
\begin{enumerate}
\sphinxsetlistlabels{\arabic}{enumi}{enumii}{}{.}%
\item {} 
\sphinxAtStartPar
Penetrating a New Market: Through PLFs, banks can tap into the burgeoning cryptocurrency lending market, attracting a fresh customer base. The emphasis, in reality, must be on cross\sphinxhyphen{}selling and upselling commercial opportunities, rather than immediate profits.

\item {} 
\sphinxAtStartPar
Diversified Revenues: PLFs provide an additional revenue stream, complementing traditional lending practices.

\item {} 
\sphinxAtStartPar
Technological Innovation: Legacy IT systems in banks are challenging to overhaul. PLFs, with their innovative technology, could serve as pilot systems, gradually expanding across the organization.

\end{enumerate}

\item {} 
\sphinxAtStartPar
\sphinxstylestrong{Risks for Traditional Banks}:
\begin{enumerate}
\sphinxsetlistlabels{\arabic}{enumi}{enumii}{}{.}%
\item {} 
\sphinxAtStartPar
Regulatory Uncertainty: DeFi lacks comprehensive regulations, exposing banks to potential legal and regulatory complexities.

\item {} 
\sphinxAtStartPar
Smart Contract Vulnerabilities: Relying on smart contracts exposes banks to cybersecurity risks and financial losses.

\end{enumerate}

\item {} 
\sphinxAtStartPar
\sphinxstylestrong{Business Opportunities for Insurance Companies}:
\begin{enumerate}
\sphinxsetlistlabels{\arabic}{enumi}{enumii}{}{.}%
\item {} 
\sphinxAtStartPar
Smart Contract Insurance: Insurance firms can offer coverage against smart contract vulnerabilities in PLFs, safeguarding users and generating premiums.

\item {} 
\sphinxAtStartPar
Risk Assessment Services: Insurance companies can develop risk assessment services for PLF participants, aiding informed decision\sphinxhyphen{}making.

\end{enumerate}

\item {} 
\sphinxAtStartPar
\sphinxstylestrong{Risks for Insurance Companies}:
\begin{enumerate}
\sphinxsetlistlabels{\arabic}{enumi}{enumii}{}{.}%
\item {} 
\sphinxAtStartPar
Limited Historical Data: Insufficient data on DeFi transactions could hinder accurate risk assessment and policy pricing.

\item {} 
\sphinxAtStartPar
Market Volatility: Cryptocurrency’s volatile nature could pose challenges in pricing insurance products, necessitating reinsurance policies. In the article “DeFi vs TradFi: Valuation Using Multiples and Discounted Cash Flow” we can learn how to value digital assets. Furthermore, it is paramount to note that, so far, this value is extremely volatile, and traditional players need to surf on this volatility, need to learn how to govern this volatility.

\end{enumerate}

\item {} 
\sphinxAtStartPar
\sphinxstylestrong{Risk Management Practices}:
\begin{enumerate}
\sphinxsetlistlabels{\arabic}{enumi}{enumii}{}{.}%
\item {} 
\sphinxAtStartPar
Tech Collaboration: Insurers can partner with tech experts to bolster smart contract security and mitigate risks.

\item {} 
\sphinxAtStartPar
Robust Risk Models: Creating advanced risk models for DeFi\sphinxhyphen{}related insurance products can enhance underwriting precision.

\end{enumerate}

\end{itemize}

\sphinxAtStartPar
\sphinxstylestrong{b) DEXs}:
\begin{itemize}
\item {} 
\sphinxAtStartPar
\sphinxstylestrong{Business Opportunities for Traditional Banks}:
\begin{enumerate}
\sphinxsetlistlabels{\arabic}{enumi}{enumii}{}{.}%
\item {} 
\sphinxAtStartPar
Liquidity Provision: Banks can serve as liquidity providers on DEX platforms, earning fees and interest on deposited assets.

\item {} 
\sphinxAtStartPar
Market Expansion: DEX partnerships enable banks to broaden their market reach beyond conventional borders.

\end{enumerate}

\item {} 
\sphinxAtStartPar
\sphinxstylestrong{Emerging Risks for Traditional Banks}:
\begin{enumerate}
\sphinxsetlistlabels{\arabic}{enumi}{enumii}{}{.}%
\item {} 
\sphinxAtStartPar
Counterparty Risks: Engaging with anonymous DEX users exposes banks to counterparty risks and potential fraud, although modern risk management practices mitigate these concerns.

\item {} 
\sphinxAtStartPar
Reputation Risks: Involvement in unregulated DEX spaces could tarnish a bank’s reputation if associated with illicit activities.

\end{enumerate}

\item {} 
\sphinxAtStartPar
\sphinxstylestrong{Business Opportunities for Insurance Companies}:
\begin{enumerate}
\sphinxsetlistlabels{\arabic}{enumi}{enumii}{}{.}%
\item {} 
\sphinxAtStartPar
Secure Custody Solutions: Insurance firms can offer secure custody solutions for DEX users, reducing asset loss risk.

\item {} 
\sphinxAtStartPar
Smart Contract Risk Coverage: Insurance products can be designed to cover smart contract vulnerabilities on DEXs.

\end{enumerate}

\item {} 
\sphinxAtStartPar
\sphinxstylestrong{Risks for Insurance Companies}:
\begin{enumerate}
\sphinxsetlistlabels{\arabic}{enumi}{enumii}{}{.}%
\item {} 
\sphinxAtStartPar
DEX Risk Understanding: Insurance firms might lack comprehensive knowledge of DEX\sphinxhyphen{}specific risks, impacting policy underwriting.

\item {} 
\sphinxAtStartPar
Pricing Data Limitation: The absence of pricing data for DEX\sphinxhyphen{}related risks could hinder accurate insurance product pricing.

\end{enumerate}

\item {} 
\sphinxAtStartPar
\sphinxstylestrong{Risk Management Practices}:
\begin{enumerate}
\sphinxsetlistlabels{\arabic}{enumi}{enumii}{}{.}%
\item {} 
\sphinxAtStartPar
Security Audits: Insurers can conduct security audits of DEX platforms to ensure adherence to high\sphinxhyphen{}security standards.

\item {} 
\sphinxAtStartPar
Blockchain Expert Collaboration: Collaboration with blockchain experts aids insurers in comprehending and assessing DEX\sphinxhyphen{}related risks.

\end{enumerate}

\end{itemize}

\sphinxAtStartPar
\sphinxstylestrong{c) Yield Aggregators}:
\begin{itemize}
\item {} 
\sphinxAtStartPar
\sphinxstylestrong{Business Opportunities for Traditional Banks}:
\begin{enumerate}
\sphinxsetlistlabels{\arabic}{enumi}{enumii}{}{.}%
\item {} 
\sphinxAtStartPar
Investment Ventures: Banks can partner with Yield Aggregators to offer innovative investment opportunities to clients, potentially yielding higher returns.

\item {} 
\sphinxAtStartPar
Fee\sphinxhyphen{}Based Income: Banks can receive fees for directing client funds toward specific Yield Aggregator strategies, reducing reliance on central bank interest rate changes.

\end{enumerate}

\item {} 
\sphinxAtStartPar
\sphinxstylestrong{Risks for Traditional Banks}:
\begin{enumerate}
\sphinxsetlistlabels{\arabic}{enumi}{enumii}{}{.}%
\item {} 
\sphinxAtStartPar
Investment Volatility: Involvement with Yield Aggregators exposes banks to market volatility and possible investment losses.

\item {} 
\sphinxAtStartPar
Reputation Risks: Poor strategy performance may harm a bank’s reputation.

\end{enumerate}

\item {} 
\sphinxAtStartPar
\sphinxstylestrong{Business Opportunities for Insurance Companies}:
\begin{enumerate}
\sphinxsetlistlabels{\arabic}{enumi}{enumii}{}{.}%
\item {} 
\sphinxAtStartPar
Tailored Investments: Insurance firms can craft investment products incorporating Yield Aggregator strategies.

\item {} 
\sphinxAtStartPar
Risk Mitigation Services: Insurance companies can provide risk mitigation services to Yield Aggregator users, reducing potential losses.

\end{enumerate}

\item {} 
\sphinxAtStartPar
\sphinxstylestrong{Risks for Insurance Companies}:
\begin{enumerate}
\sphinxsetlistlabels{\arabic}{enumi}{enumii}{}{.}%
\item {} 
\sphinxAtStartPar
Complex Risk Evaluation: Grasping and evaluating diverse Yield Aggregator strategies might pose challenges for insurers.

\item {} 
\sphinxAtStartPar
Strategy Performance Uncertainty: Insurers face uncertainty in predicting the performance of selected Yield Aggregator strategies.

\end{enumerate}

\item {} 
\sphinxAtStartPar
\sphinxstylestrong{Risk Management Practices}:
\begin{enumerate}
\sphinxsetlistlabels{\arabic}{enumi}{enumii}{}{.}%
\item {} 
\sphinxAtStartPar
Diversification: Encouraging investors to diversify across different Yield Aggregators can mitigate risks tied to individual strategies.

\item {} 
\sphinxAtStartPar
Transparent Reporting: Insurers can demand transparent reporting from Yield Aggregators, ensuring clients comprehend their investments.

\end{enumerate}

\end{itemize}


\subsection{Conclusion}
\label{\detokenize{BOER/boer:conclusion}}
\sphinxAtStartPar
Although the adoption of DeFi business models by traditional insurance companies and banks may seem paradoxical, exploring these opportunities is essential to thrive in a competitive financial landscape. Understanding unique business opportunities and potential risks associated with each DeFi model is crucial for informed decision\sphinxhyphen{}making. By implementing robust risk management practices and collaborating with blockchain experts, traditional financial institutions can harness DeFi’s potential while mitigating inherent risks. This shift is strategic, enabling institutions to capitalize on emerging opportunities and ensure long\sphinxhyphen{}term sustainability within the financial industry mitigating existing and emerging risks.




\subsection{References}
\label{\detokenize{BOER/boer:references}}\phantomsection\label{\detokenize{BOER/boer:id1}}
\sphinxstepscope


\section{Crypto and Digital Assets Risk Management: Navigating Opportunities and Challenges}
\label{\detokenize{ARM/arm:crypto-and-digital-assets-risk-management-navigating-opportunities-and-challenges}}\label{\detokenize{ARM/arm::doc}}
\sphinxAtStartPar
\sphinxstylestrong{Industry Perspective}

\sphinxAtStartPar
\sphinxstylestrong{Disclaimer:} The views and opinions expressed in this article are solely those of the author

\begin{sphinxadmonition}{note}{Key Insights}
\begin{itemize}
\item {} 
\sphinxAtStartPar
Cryptocurrencies and digital assets offer legacy banks opportunities for diversification and portfolio expansion beyond traditional financial instruments, leading to potential risk\sphinxhyphen{}adjusted returns and exposure to new markets. Final goal is to gain a rare and hard\sphinxhyphen{}to\sphinxhyphen{}replicate competitive advantage that will lead to extra\sphinxhyphen{}profits.

\item {} 
\sphinxAtStartPar
Traditional banks that offer cryptocurrency and digital assets services can attract and retain more clients, leading to increased customer engagement and new revenues streams.

\item {} 
\sphinxAtStartPar
Embracing cryptocurrencies allows banks to position themselves as innovators in the financial industry, differentiating themselves from competitors and gaining a competitive advantage.

\item {} 
\sphinxAtStartPar
Banks need to consider the inter\sphinxhyphen{}relationships between arising risks deriving from crypto\sphinxhyphen{}currencies and traditional risks such as interest rate risk, currency risk, and liquidity risks when managing their exposure to these assets.

\item {} 
\sphinxAtStartPar
Interest rate fluctuations can impact the valuation and attractiveness of cryptocurrencies, influencing banks’ investment strategies and portfolio allocation.

\item {} 
\sphinxAtStartPar
Cryptocurrencies do not affect currency risk due to their lack of ties to specific fiat currencies.

\item {} 
\sphinxAtStartPar
Banks have opportunities to hedge against volatility in customer digital asset portfolios and their own exposure to price volatility through emerging cryptocurrency derivatives and hedging instruments.

\item {} 
\sphinxAtStartPar
Liquidity risks in cryptocurrency markets need to be carefully assessed and managed, particularly during periods of market stress.

\item {} 
\sphinxAtStartPar
Banks, especially in the European Union, need to consider climate and environmental risks associated with financing crypto customers, given the energy\sphinxhyphen{}intensive nature of cryptocurrency assets.

\item {} 
\sphinxAtStartPar
Information and Communication Technology (ICT) risks and Cyber Risks are critical considerations for banks offering cryptocurrency services.

\item {} 
\sphinxAtStartPar
Effective risk management practices, proactive monitoring, robust internal controls, and compliance with regulations are crucial for banks to navigate the opportunities and challenges presented by cryptocurrencies and digital assets. By adopting a proactive approach to risk management, banks can position themselves as industry leaders, enhance customer relationships, and drive sustainable growth in the digital era.

\end{itemize}
\end{sphinxadmonition}


\subsection{Introduction}
\label{\detokenize{ARM/arm:introduction}}
\sphinxAtStartPar
Cryptocurrencies and digital assets have emerged as disruptive forces in the financial industry, presenting both opportunities and risks for banks {[}\hyperlink{cite.ARM/arm:id144}{NBMG16}, \hyperlink{cite.ARM/arm:id145}{Pan23}{]}.
As technology advances and the digital economy expands, it is crucial for banks (to be more specific: legacy and traditional banks) to understand and effectively manage the risks associated with these assets. These risks are generated and linked from the following drivers:
\begin{itemize}
\item {} 
\sphinxAtStartPar
the volatile nature and intrinsic features of digital assets and cryptos

\item {} 
\sphinxAtStartPar
customers that trade and hold these specific assets

\item {} 
\sphinxAtStartPar
as\sphinxhyphen{}is organizational frameworks related to risk management and best practices that did not consider cryptocurrencies and digital assets

\end{itemize}

\sphinxAtStartPar
The objective of this article is twofold: on one hand to provide a review of the main financial and non\sphinxhyphen{}financial risks arising from the management of non\sphinxhyphen{}standard products for traditional banks; on the other hand, to identify the key value drivers within risk management to uncover new best practices and new business opportunities.

\sphinxAtStartPar
To achieve these two objectives, the pros and cons of each risk will be emphasized.

\sphinxAtStartPar
At the conclusion of the article, the risks that can raise greater concern and those that can be harnessed as catalysts to amplify innovation, both technologically and financially, will become clear.


\subsection{Managing Risks}
\label{\detokenize{ARM/arm:managing-risks}}
\sphinxAtStartPar
Let us begin with an examination of the various pros related to holding and trading cryptocurrencies for legacy Banks:
\begin{enumerate}
\sphinxsetlistlabels{\arabic}{enumi}{enumii}{}{.}%
\item {} 
\sphinxAtStartPar
\sphinxstylestrong{Diversification and Portfolio Expansion}:
Cryptocurrencies and digital assets offer banks the potential for diversifying their portfolios beyond traditional financial instruments. This can enhance risk\sphinxhyphen{}adjusted returns and provide exposure to new markets and investment opportunities.

\item {} 
\sphinxAtStartPar
\sphinxstylestrong{Client Demand and Engagement}:\\
Banks that offer cryptocurrency services can attract and retain more clients seeking exposure to digital assets. Meeting this demand can lead to increased customer engagement and revenue generation (i.e., new revenue streams). Let’s look for example at Revolut Ltd. case {[}\hyperlink{cite.ARM/arm:id150}{Ltd21}{]}. As widely acknowledged, Revolut stands as one of the most accomplished digital banks, providing customers with an array of services encompassing crypto services, traditional banking features, and various other financial offerings. The Financial Statements of Revolut Ltd. communicate that the company succeeded in drawing a substantial customer base by utilizing a comprehensive financial package, with its cryptocurrency services taking the spotlight. Moreover, Revolut’s strong suit lies in its ability to enable many individuals to manage cryptocurrencies without encountering technicalities or complexities associated with handling the private keys of digital assets. All these concepts are derived from Revolut’s main financials {[}\hyperlink{cite.ARM/arm:id150}{Ltd21}{]} and Revolut case, as depicted in the table below, Revolut’s growth rate is remarkable, encompassing both its customer base and revenues. \sphinxstylestrong{This underscores the hidden value that legacy banks can uncover and harness by integrating cryptocurrency and digital asset services into their existing offerings.}

\end{enumerate}




\begin{enumerate}
\sphinxsetlistlabels{\arabic}{enumi}{enumii}{}{.}%
\setcounter{enumi}{2}
\item {} 
\sphinxAtStartPar
\sphinxstylestrong{Innovation and Competitive Advantage} {[}\hyperlink{cite.ARM/arm:id144}{NBMG16}{]}:\\
Embracing cryptocurrencies allows banks to position themselves as innovators in the financial industry. It demonstrates adaptability and can help differentiate them from competitors enabling the creation of a different, cutting\sphinxhyphen{}edge competitive advantage, something that is rare in the financial industry, and hard to copy. It is very important to highlight that, within the legacy banks market, the first mover advantage is paramount – guaranteeing the capture of extra profits that later movers cannot exploit.

\end{enumerate}

\sphinxAtStartPar
At this stage, it becomes crucial to emphasize the interrelations between cryptocurrencies and traditional risks. Below, you will discover the key insights pertaining to this subject.
\begin{enumerate}
\sphinxsetlistlabels{\arabic}{enumi}{enumii}{}{.}%
\item {} 
\sphinxAtStartPar
\sphinxstylestrong{Interest Rate Risk} {[}\hyperlink{cite.ARM/arm:id146}{DugganWaynePowellFarran23}, \hyperlink{cite.ARM/arm:id149}{Tan23}{]}:
Banks need to consider the potential impact of interest rate fluctuations on the valuation of cryptocurrencies, especially if they hold these assets on their balance sheets. It’s important to recognize that the relationship between interest rates and cryptocurrencies is complex and influenced by numerous factors, including absolute value of interests, market sentiment, regulatory developments, and technological advancements {[}\hyperlink{cite.ARM/arm:id146}{DugganWaynePowellFarran23}, \hyperlink{cite.ARM/arm:id151}{Pec23}, \hyperlink{cite.ARM/arm:id149}{Tan23}{]}. Banks need to carefully assess and monitor these dynamics to effectively manage the interest rate risk associated with cryptocurrencies.

\sphinxAtStartPar
Let’s look at the \sphinxstylestrong{Interest Rate Increase} scenario:

\sphinxAtStartPar
If interest rates rise, they can have several implications for cryptocurrencies held by banks:
\begin{itemize}
\item {} 
\sphinxAtStartPar
\sphinxstylestrong{Valuation Impact}: cryptocurrencies, like other investment assets, may experience a decline in value when interest rates increase. This valuation impact can affect the overall profitability of the bank’s investment portfolio.

\item {} 
\sphinxAtStartPar
\sphinxstylestrong{Opportunity Cost}: Rising interest rates can make traditional fixed\sphinxhyphen{}income investments more attractive, potentially diverting investment capital away from cryptocurrencies. This could lead to a reduced allocation to cryptocurrencies within the bank’s portfolio.

\end{itemize}

\sphinxAtStartPar
Conversely, if \sphinxstylestrong{interest rates decrease}, they can also affect cryptocurrencies in the following ways:
\begin{itemize}
\item {} 
\sphinxAtStartPar
\sphinxstylestrong{Valuation Impact}: cryptocurrencies may experience increased demand when interest rates decline, leading to potential price appreciation. However, it is important to note that cryptocurrencies are influenced by various other factors, and interest rates alone may not be the sole driver of their valuation.

\item {} 
\sphinxAtStartPar
\sphinxstylestrong{Investment Attractiveness}: lower interest rates can make cryptocurrencies relatively more attractive compared to traditional fixed\sphinxhyphen{}income investments that provide lower yields. Banks may consider increasing their exposure to cryptocurrencies as part of their investment strategy during periods of low interest rates.

\end{itemize}

\sphinxAtStartPar
Moreover, it’s worth noting that interest rate risk related to cryptocurrencies primarily arises when they are held as investment assets on a bank’s balance sheet. If cryptocurrencies are held for other purposes, such as providing custody {[}\hyperlink{cite.ARM/arm:id147}{20223}{]}or trading services, the interest rate risk may be less relevant. As with any investment asset, banks should have appropriate risk management policies and frameworks in place to assess and monitor interest rate risk associated with cryptocurrencies. This may involve stress testing, scenario analysis, and regular reviews to ensure the bank’s exposure to interest rate fluctuations aligns with its risk appetite and strategic objectives.

\item {} 
\sphinxAtStartPar
\sphinxstylestrong{Currency Risk}: Cryptocurrencies are not tied to any specific fiat currency, so it is possible to infer that no idiosyncratic currency risk is in place.

\item {} 
\sphinxAtStartPar
\sphinxstylestrong{Hedging Risks} {[}\hyperlink{cite.ARM/arm:id149}{Tan23}{]}: here, hedging risks have two financial legs:
\begin{itemize}
\item {} 
\sphinxAtStartPar
The first one is related to the protection against volatility related to customer digital assets portfolios (that can be seen as a commercial opportunity for legacy banks)

\item {} 
\sphinxAtStartPar
The second one is related to the protection of the legacy bank itself against volatility. Please note that cryptocurrency derivatives (or more in general, digital assets derivatives) and hedging instruments are emerging, providing opportunities, and related risks, for banks to hedge their exposure to price volatility and mitigate associated risks.

\end{itemize}

\item {} 
\sphinxAtStartPar
\sphinxstylestrong{Liquidity Risks} {[}\hyperlink{cite.ARM/arm:id149}{Tan23}{]}: cryptocurrency markets can experience liquidity challenges, especially during periods of market stress. Banks should carefully assess liquidity ratios and availability (to be more specific also in the interbank and monetary market) when holding and trading these assets.

\item {} 
\sphinxAtStartPar
\sphinxstylestrong{Climate \& Environmental Risks}: banks, especially in the European Union, have targets (such as Net Zero Banking Alliance) on their own greenhouse gases (GHG) emissions and financed emissions (GHG emissions produced by banks’ customers). This topic is very relevant because financing crypto customers could result in increasing GHG financed emissions. Since some cryptocurrency assets/digital assets are very energy\sphinxhyphen{}intensive assets (e.g., due to mining), the main challenge is how to make a fruitful cherry\sphinxhyphen{}picking of digital and crypto assets that at the same time:
\begin{itemize}
\item {} 
\sphinxAtStartPar
rely on the most efficient technology (hence, they emit less with respect to the market)

\item {} 
\sphinxAtStartPar
guarantee the most profitable income for legacy banks and customers, given the GHG emissions target

\end{itemize}

\item {} 
\sphinxAtStartPar
\sphinxstylestrong{Reputational Risks}: Reputational risks can emerge while managing cryptocurrencies and crypto customers, stemming from regulatory scrutiny as well as media and public perception surrounding these assets.

\item {} 
\sphinxAtStartPar
\sphinxstylestrong{ICT Risks} {[}\hyperlink{cite.ARM/arm:id148}{Ser23}, \hyperlink{cite.ARM/arm:id149}{Tan23}{]}: Information and Communication Technology (ICT) risks refer to the potential risks associated with the use of technology, information systems, and infrastructure within an organization. In the context of cryptocurrencies and digital assets, ICT risk could be the paramount one, and can manifest in the following ways:
\begin{itemize}
\item {} 
\sphinxAtStartPar
\sphinxstylestrong{Security of Digital Wallets and Exchanges}:
\begin{itemize}
\item {} 
\sphinxAtStartPar
Unauthorized Access: Cryptocurrency wallets (provided by legacy banks) can be vulnerable to unauthorized access, leading to (personal) data theft or loss of digital assets. Risks include hacking attempts, phishing attacks, or malware targeting the storage or transfer mechanisms of cryptocurrencies.

\item {} 
\sphinxAtStartPar
Infrastructure Vulnerabilities: Weaknesses in ICT infrastructure, such as servers, networks, or software, can expose digital wallets to potential security breaches. Vulnerabilities in infrastructure may arise due to outdated systems, inadequate security measures, or insufficient monitoring and patching practices.

\end{itemize}

\item {} 
\sphinxAtStartPar
\sphinxstylestrong{Data Integrity and Availability}: No specific issues coming from these topics.

\end{itemize}

\item {} 
\sphinxAtStartPar
\sphinxstylestrong{Cyber Risks} {[}\hyperlink{cite.ARM/arm:id148}{Ser23}, \hyperlink{cite.ARM/arm:id149}{Tan23}{]}: Cyber risks refer to the potential risks arising from malicious activities. In the context of cryptocurrencies and digital assets, cyber risks can include, inter alia:
\begin{itemize}
\item {} 
\sphinxAtStartPar
\sphinxstylestrong{Phishing and Social Engineering}: Cybercriminals may attempt to deceive individuals within the bank or its clients into revealing sensitive information, such as private keys or login credentials, through phishing emails, fake websites, or social engineering techniques.

\item {} 
\sphinxAtStartPar
\sphinxstylestrong{Malware and Ransomware Attacks}: Malicious software can be used to target digital wallets, exchanges, or users, aiming to gain unauthorized access, steal funds, or encrypt data for ransom. Ransomware attacks can disrupt operations and cause financial losses if not adequately mitigated.

\item {} 
\sphinxAtStartPar
\sphinxstylestrong{Insider Threats}: Internal personnel with access to sensitive information or systems can intentionally or unintentionally misuse or compromise digital assets.

\end{itemize}

\item {} 
\sphinxAtStartPar
\sphinxstylestrong{Operational Risks}: There are no specific or innovative aspects in terms of Operational Risks. All Operational Risks are interconnected and integrated within the Cyber Risk realm. Therefore, there’s no necessity to extensively explore this topic.

\end{enumerate}

\sphinxAtStartPar
As evident from the text above, ICT Risks and Cyber Risks pose the most significant threats that legacy banks must face, and these could even lead to potential reputational risks in the event of their occurrence. These two areas require the most substantial efforts in terms of financial IT investment, skilled human resources, and new organization framework by the legacy banks. They must undertake considerable technological advancements to enter the realm of cryptocurrency business.






\subsection{Main results}
\label{\detokenize{ARM/arm:main-results}}
\sphinxAtStartPar
The objective of this article, as previously mentioned, was two\sphinxhyphen{}fold: firstly, to provide an overview of the main financial and non\sphinxhyphen{}financial risks arising from managing non\sphinxhyphen{}standard products for traditional banks, such as cryptocurrencies and digital assets. Secondly, to pinpoint the key value drivers within risk management to uncover novel best practices and innovative business opportunities. The primary advantage of the guidelines outlined in the main section of this article is the following: to manage cryptocurrencies and digital assets, banks must introduce an additional variable, in other words, an additional complexity layer, into their organizational framework, and to be more specific within their risk management models. This complexity translates into both a negative aspect, which can be represented as increased costs to manage this new layer of complexity, and a positive aspect: skillful management of this complexity could transform legacy institutions into innovative ones, securing additional profits and competitive advantages that are rare and hard to replicate within the financial industry {[}\hyperlink{cite.ARM/arm:id147}{20223}, \hyperlink{cite.ARM/arm:id146}{DugganWaynePowellFarran23}, \hyperlink{cite.ARM/arm:id144}{NBMG16}, \hyperlink{cite.ARM/arm:id149}{Tan23}{]}.

\sphinxAtStartPar
Furthermore, the legacy bank that first adopts this new analytical variable into its risk management model (and therefore, also into its business approach) can enjoy the benefits of being a first mover.

\sphinxAtStartPar
Eventually, the following key takeaways emerge:
1. Banks can benefit from engaging with cryptocurrencies through diversification, attracting clients, and fostering innovation (let’s look for example at the Revolut case, highlighted in this article).
2. Effective, proactive, and reactive risk management practices are crucial to mitigate traditional risks (e.g., interest rate risk) in the context of cryptocurrencies.
3. Liquidity risks in cryptocurrency markets require careful consideration and contingency planning.
4. Proactive monitoring, robust internal controls, and compliance with regulations are essential for banks to manage risks effectively.


\subsection{Conclusion}
\label{\detokenize{ARM/arm:conclusion}}
\sphinxAtStartPar
Cryptocurrencies and digital assets represent a significant paradigm shift in the financial industry.

\sphinxAtStartPar
\sphinxstylestrong{It’s important for a legacy bank to:}
\begin{itemize}
\item {} 
\sphinxAtStartPar
gain a rare and hard\sphinxhyphen{}to\sphinxhyphen{}replicate competitive advantage, leveraging also on first\sphinxhyphen{}mover advantage. The ultimate objective is to secure additional profits compared to the current state.

\item {} 
\sphinxAtStartPar
tailor risk management practices specific to your organization, regulatory requirements, and commercial opportunities. Banks that recognize and embrace these changes can harness new opportunities while managing the associated risks. By adopting a proactive and reactive approach to risk management, banks can position themselves at the forefront of innovation, enhance customer relationships, and drive sustainable growth in the digital era.

\end{itemize}




\subsection{References}
\label{\detokenize{ARM/arm:references}}
\sphinxstepscope


\chapter{Innovation \& Ideation}
\label{\detokenize{STRUCTURE/innovation_2:innovation-ideation}}\label{\detokenize{STRUCTURE/innovation_2::doc}}
\begin{sphinxuseclass}{sd-container-fluid}
\begin{sphinxuseclass}{sd-sphinx-override}
\begin{sphinxuseclass}{sd-mb-4}
\begin{sphinxuseclass}{sd-row}
\begin{sphinxuseclass}{sd-row-cols-1}
\begin{sphinxuseclass}{sd-row-cols-xs-1}
\begin{sphinxuseclass}{sd-row-cols-sm-2}
\begin{sphinxuseclass}{sd-row-cols-md-2}
\begin{sphinxuseclass}{sd-row-cols-lg-2}
\begin{sphinxuseclass}{sd-g-2}
\begin{sphinxuseclass}{sd-g-xs-2}
\begin{sphinxuseclass}{sd-g-sm-2}
\begin{sphinxuseclass}{sd-g-md-2}
\begin{sphinxuseclass}{sd-g-lg-2}
\begin{sphinxuseclass}{sd-col}
\begin{sphinxuseclass}{sd-d-flex-row}
\begin{sphinxuseclass}{sd-mt-3}
\begin{sphinxuseclass}{sd-mb-0}
\begin{sphinxuseclass}{sd-ml-0}
\begin{sphinxuseclass}{sd-mr-0}
\begin{sphinxuseclass}{sd-card}
\begin{sphinxuseclass}{sd-sphinx-override}
\begin{sphinxuseclass}{sd-w-100}
\begin{sphinxuseclass}{sd-shadow-md}
\begin{sphinxuseclass}{sd-card-hover}
\begin{sphinxuseclass}{sd-text-center}
\begin{sphinxuseclass}{sd-card-body}
\begin{sphinxuseclass}{sd-card-title}
\begin{sphinxuseclass}{sd-font-weight-bold}Soulbound Tokens (SBTs)
\end{sphinxuseclass}
\end{sphinxuseclass}


\end{sphinxuseclass}\sphinxhref{https://dlt-science.github.io/science-notes/SBT/SBT.html}{}
\end{sphinxuseclass}
\end{sphinxuseclass}
\end{sphinxuseclass}
\end{sphinxuseclass}
\end{sphinxuseclass}
\end{sphinxuseclass}
\end{sphinxuseclass}
\end{sphinxuseclass}
\end{sphinxuseclass}
\end{sphinxuseclass}
\end{sphinxuseclass}
\end{sphinxuseclass}
\end{sphinxuseclass}
\end{sphinxuseclass}
\end{sphinxuseclass}
\end{sphinxuseclass}
\end{sphinxuseclass}
\end{sphinxuseclass}
\end{sphinxuseclass}
\end{sphinxuseclass}
\end{sphinxuseclass}
\end{sphinxuseclass}
\end{sphinxuseclass}
\end{sphinxuseclass}
\end{sphinxuseclass}
\end{sphinxuseclass}
\sphinxstepscope


\section{Soulbound Tokens (SBTs)}
\label{\detokenize{SBT/SBT:soulbound-tokens-sbts}}\label{\detokenize{SBT/SBT::doc}}
\sphinxAtStartPar
\sphinxstylestrong{Innovation \& Ideation}

\begin{sphinxadmonition}{note}{Key Insights}
\begin{itemize}
\item {} 
\sphinxAtStartPar
Decentralised Society (DeSoc) serves as an innovative solution that encourages a trust\sphinxhyphen{}based, cooperative, bottom\sphinxhyphen{}up strategy in constructing resilient networks, thereby enhancing the potential of Web3.

\item {} 
\sphinxAtStartPar
Soulbound tokens (SBTs), as non\sphinxhyphen{}transferable assets, improve the provenance and reputation in the Decentralised Society (DeSoc) and provide a versatile representation of digital identities.

\item {} 
\sphinxAtStartPar
SBTs have the potential to redefine digital identity verification due to their non\sphinxhyphen{}transferable nature, allowing them to authenticate factual records, establish digital inheritance plans, and prevent Sybil attacks in Decentralised Autonomous Organisations.

\item {} 
\sphinxAtStartPar
SBTs offer functional solutions in various sectors such as finance, real estate, and healthcare, promoting transparency, security, and innovation across these industries.

\item {} 
\sphinxAtStartPar
Despite the significant advancements that SBTs bring to digital identity systems, they also face obstacles concerning privacy, security, and interoperability.

\end{itemize}
\end{sphinxadmonition}


\subsection{Introduction}
\label{\detokenize{SBT/SBT:introduction}}
\begin{sphinxShadowBox}
\sphinxstylesidebartitle{\sphinxstylestrong{Web3}}

\sphinxAtStartPar
Web3, short for Web 3.0, is the third generation of internet services for websites and applications that incorporate blockchain\sphinxhyphen{}based and decentralised processes. It emphasises a user\sphinxhyphen{}centric online experience where data ownership and control is returned to the individual, as opposed to being centralised in large tech companies.
\end{sphinxShadowBox}

\begin{sphinxShadowBox}
\sphinxstylesidebartitle{\sphinxstylestrong{DAOs}}

\sphinxAtStartPar
A Decentralised Autonomous Organization (DAO) is a blockchain\sphinxhyphen{}based system governed by rules encoded as computer programmes known as smart contracts, with decision\sphinxhyphen{}making authority distributed among its members.
\end{sphinxShadowBox}

\sphinxAtStartPar
Web3 has largely been anonymous for its users, due to its founding principles, which are deeply rooted in privacy and decentralisation. However, the lack of ability to confirm individual identities, their properties, and affiliations has posed a challenge for blockchain adoption in some industries. Soulbound tokens (SBTs) are set to bridge this identity gap inherent in Web3, facilitating the formation of trusted relationships. Soulbound tokens can be issued by any entity, be it DAOs, academic institutions, DeFi firms, or employers, to denote membership, authentication or certification, or event participation. Moreover, soulbound tokens can utilise these links between individuals and various entities to provide a more comprehensive picture of distinct user identities and their roots. The reputation of the issuing entity is transferred via SBTs to the wallets or individuals who hold them. The more prestigious the issuing body, the higher the standing of the individual in possession of a soulbound token.

\sphinxAtStartPar
As an individual’s connections with different entities grow, so does their unique identity and reputation. Soulbound tokens capitalise on this network of connections to construct verifiable identities for souls. NFTs will serve as proof of ownership, and SBTs as proof of character {[}\hyperlink{cite.SBT/SBT:id80}{CG22}{]}.

\begin{figure}[htbp]
\centering
\capstart

\noindent\sphinxincludegraphics[width=1055\sphinxpxdimen,height=785\sphinxpxdimen]{{USBTDiagram3.drawio}.png}
\caption{Soulbound Token Issuance in DeSoc.}\label{\detokenize{SBT/SBT:sbt-diagram}}\end{figure}


\subsection{Moving towards a Decentralised Society (DeSoc)}
\label{\detokenize{SBT/SBT:moving-towards-a-decentralised-society-desoc}}
\begin{sphinxShadowBox}
\sphinxstylesidebartitle{\sphinxstylestrong{Sybil Attack}}

\sphinxAtStartPar
A Sybil attack is a type of security threat in decentralised networks where a single entity creates multiple fake identities (Sybils) to gain undue influence or control. These attacks can disrupt the functioning of the network by undermining consensus mechanisms.
\end{sphinxShadowBox}

\begin{sphinxShadowBox}
\sphinxstylesidebartitle{\sphinxstylestrong{Collusion Attack}}

\sphinxAtStartPar
A collusion attack in blockchain refers to a scenario where a group of participants in a network conspire together to manipulate the system for their own advantage. This can happen in proof\sphinxhyphen{}of\sphinxhyphen{}stake or proof\sphinxhyphen{}of\sphinxhyphen{}work blockchain systems where enough nodes, typically over 50\%, are controlled by a colluding group. This allows them to control the validation of transactions, potentially allowing them to double\sphinxhyphen{}spend, block transactions, or manipulate the blockchain in other ways. This is often referred to as a 51\% attack.
\end{sphinxShadowBox}

\begin{sphinxShadowBox}
\sphinxstylesidebartitle{\sphinxstylestrong{Hyper\sphinxhyphen{}financialization}}

\sphinxAtStartPar
Hyper\sphinxhyphen{}financialization refers to the dominance of financial markets, institutions, and elites in the economy.
\end{sphinxShadowBox}

\sphinxAtStartPar
Web3 aims to revolutionise society beyond just financial systems, but its current lack of mechanisms to represent human identities and relationships in virtual worlds leads to issues like Sybil attacks, collusion, and an inclination towards hyper\sphinxhyphen{}financialization {[}\hyperlink{cite.SBT/SBT:id86}{WOB22}{]}. To counter this, Weyl et al. {[}\hyperlink{cite.SBT/SBT:id86}{WOB22}{]} proposed the concept of a Decentralised Society (DeSoc), an approach fostering complex and diverse relationships across digital and physical realities. It is built on trust and cooperation while also correcting for biases and tendencies to overcoordinate.

\begin{sphinxShadowBox}
\sphinxstylesidebartitle{\sphinxstylestrong{Vampire Attack}}

\sphinxAtStartPar
In the context of decentralised finance (DeFi) and blockchain, a vampire attack is a strategy where a new protocol aims to drain liquidity and users from an existing one. This is often done by offering higher rewards or better incentives on the new platform, incentivizing users of the old platform to migrate their assets.
\end{sphinxShadowBox}

\sphinxAtStartPar
Economic growth is primarily driven by networks yielding increasing returns, but the current private property paradigm of DeFi can limit such growth. DeSoc recommends treating networks as partially and collectively shared goods, applying governance mechanisms that balance trust and cooperation, while checking for collusion and capture. This model supports a bottom\sphinxhyphen{}up approach in building, participating in, and governing networks. Consequently, it creates structures resilient to Sybil and vampire attacks and collusion, and promotes plural networks that provide widespread benefits, agreed upon by diverse members.

\sphinxAtStartPar
The strength of DeSoc lies in fostering broader cooperation by encouraging the creation and intersection of nested networks across the physical and digital realms. Building on trust, it allows for the establishment of resilient plural network goods. This approach enables Web3 to resist short\sphinxhyphen{}term financialization and cultivate a future with increasing returns across diverse social connections.

\begin{sphinxadmonition}{note}{Note:}
\sphinxAtStartPar
\sphinxstylestrong{Differences between soulbound tokens (SBTs) and regular non\sphinxhyphen{}fungible tokens (NFTs)}
\begin{itemize}
\item {} 
\sphinxAtStartPar
Soulbound Tokens (SBTs) are unique in that they are non\sphinxhyphen{}transferable and exhibit public transparency.

\item {} 
\sphinxAtStartPar
Regular Non\sphinxhyphen{}Fungible Tokens (NFTs), on the other hand, are unique, non\sphinxhyphen{}interchangeable tokens that can represent digital assets, such as digital art.

\item {} 
\sphinxAtStartPar
Both SBTs and NFTs can serve as a means to authenticate and identify products or records.

\item {} 
\sphinxAtStartPar
SBTs stand out in their ability to serve as a form of permission, authorisation, and access to legal documents, binding the token to a specific identity.

\item {} 
\sphinxAtStartPar
Conversely, NFTs can be utilised as tickets granting access to exclusive events, without requiring identity verification, as they can be freely transferred between parties.

\end{itemize}
\end{sphinxadmonition}


\subsection{Functionality of SBTs}
\label{\detokenize{SBT/SBT:functionality-of-sbts}}
\sphinxAtStartPar
A distinct and pivotal characteristic of Soulbound tokens (SBTs) is their inherent non\sphinxhyphen{}transferability. Unlike existing NFTs and token standards like the fungible ERC\sphinxhyphen{}20 or non\sphinxhyphen{}fungible ERC\sphinxhyphen{}721, which are built to hold market value and can be sold or transferred between wallets, SBTs are uniquely tied to Souls and therefore are not designed for selling or transferring {[}\hyperlink{cite.SBT/SBT:id90}{Tak23}{]}.

\sphinxAtStartPar
SBTs are issued and held within unique accounts known as Souls, which serve as a vessel for these tokens and play a crucial role in establishing provenance and reputation. Souls can denote various entities, ranging from individuals to organisations, companies, and more. It’s noteworthy that in a decentralised society (DeSoc), Souls are not required to have a direct human equivalence, meaning a single person can be associated with multiple Souls. Unlike regular NFTs, soulbound tokens (SBTs) are a concept of non\sphinxhyphen{}transferable assets {[}\hyperlink{cite.SBT/SBT:id91}{Tit22}{]}. Once issued, they belong to a specific identity {[}\hyperlink{cite.SBT/SBT:id82}{Hil22}{]}.

\sphinxAtStartPar
This flexibility can manifest in a multitude of ways. For example, an individual could possess an array of ‘Souls’, each symbolizing different facets of their identity, such as their professional credentials, medical histories, among other elements {[}\hyperlink{cite.SBT/SBT:id90}{Tak23}{]}.


\subsection{Potential Applications of SBTs}
\label{\detokenize{SBT/SBT:potential-applications-of-sbts}}
\sphinxAtStartPar
Soulbound Tokens (SBTs) are a revolutionary concept in the realm of blockchain technology, enabling the creation of verifiable, non\sphinxhyphen{}transferable digital records tied to an individual’s identity or “soul”. With their immutable and decentralised characteristics, these tokens offer several potential applications that span numerous industries and societal structures. From authenticating factual records, devising digital inheritance plans, and facilitating alternative credit systems to preventing Sybil attacks in Decentralised Autonomous Organisations (DAOs), enhancing trust in online property rentals, and securing the management of healthcare records, SBTs are primed to reshape the digital world. The following sections detail some of the most promising applications of Soulbound Tokens in diverse fields.


\subsubsection{Verifying Authenticity of Factual Records}
\label{\detokenize{SBT/SBT:verifying-authenticity-of-factual-records}}
\sphinxAtStartPar
Soulbound Tokens can be used to confirm the authenticity of supposed factual records, such as photos and videos. As deep fake technology continues to advance, it’s becoming increasingly difficult for both humans and algorithms to determine the truth through direct examination. While blockchain inclusion enables us to trace the time a particular work was made, SBTs would enable us to trace the social provenance, giving us rich social context to the Soul that issued the work, their constellation of memberships, aliations, credentials and their social distance to the subject {[}\hyperlink{cite.SBT/SBT:id86}{WOB22}{]}.


\subsubsection{Digital Inheritance Planning}
\label{\detokenize{SBT/SBT:digital-inheritance-planning}}
\sphinxAtStartPar
Soulbound Tokens (SBTs) can be employed as a mechanism to confirm a user’s existence. Considering SBT use cases, a digital inheritance plan could be created where the testator generates SBTs for executors, guardians, and beneficiaries, transferring these tokens to their respective wallets. This process not only verifies the existence of all involved parties but also bolsters the security of the testator’s digital assets {[}\hyperlink{cite.SBT/SBT:id81}{GCOGI23}{]}.


\subsubsection{Alternative Credit Systems}
\label{\detokenize{SBT/SBT:alternative-credit-systems}}
\sphinxAtStartPar
An ecosystem of Soulbound Tokens (SBTs) could provide an alternative to traditional credit systems, using education credentials, work history, and rental contracts to build a credit history. Non\sphinxhyphen{}transferable SBTs representing loans could be used as non\sphinxhyphen{}seizable reputation\sphinxhyphen{}based collateral. The system would prevent loan evasion and promote transparency in lending markets, reducing reliance on centralised credit\sphinxhyphen{}scoring. Ultimately, this could enhance lending algorithms and facilitate lending within social networks {[}\hyperlink{cite.SBT/SBT:id86}{WOB22}{]}.


\subsubsection{Preventing Sybil Attacks in DAOs}
\label{\detokenize{SBT/SBT:preventing-sybil-attacks-in-daos}}
\sphinxAtStartPar
Soulbound Tokens (SBTs) can also be used to prevent Sybil attacks in Decentralised Autonomous Organisations (DAOs) by differentiating between unique users and potential bots based on their SBTs. More reputable SBT holders can be given more voting power. Specific “proof\sphinxhyphen{}of\sphinxhyphen{}personhood” SBTs can be issued to assist other DAOs in Sybil resistance. Additionally, vote weight can be adjusted based on correlations among SBTs held by voting participants {[}\hyperlink{cite.SBT/SBT:id86}{WOB22}{]}.


\subsubsection{Enhancing Trust in Online Property Rental}
\label{\detokenize{SBT/SBT:enhancing-trust-in-online-property-rental}}
\sphinxAtStartPar
The economic growth potential of the real estate sector is significant, encompassing diverse industries from retail to housing services. The digitization of real estate, however, has invited several challenges, notably in the form of scams targeting landlords and tenants. To address this issue, a blockchain\sphinxhyphen{}based property rental platform is proposed. This platform will utilise Soulbound Tokens (SBTs) to verify the credibility and reputation of users, providing security against online rental fraud. A non\sphinxhyphen{}transferable, non\sphinxhyphen{}fungible token is provided to the new user that records their reputation across their time on the website. Property listings will be structured as smart contracts on the platform, ensuring secure and immutable transaction terms between landlords and tenants. This could drastically reduce fraud, enhance trust, and potentially transform the online rental industry {[}\hyperlink{cite.SBT/SBT:id83}{SKSK23}{]}.


\subsubsection{Decentralised Dispute Resolution}
\label{\detokenize{SBT/SBT:decentralised-dispute-resolution}}
\sphinxAtStartPar
Decentralised dispute resolution platforms could use soulbound tokens, tied to an arbitrator’s real identity, as a mechanism to enhance system integrity. These tokens, earned through completing tasks, safeguard against system manipulation such as whitewashing or Sybil attacks. Additionally, the tokens represent an arbitrator’s decision\sphinxhyphen{}making accuracy, not their financial capacity. Arbitrators may need to provide credentials like licences or certificates, along with proof of identity. This data would be presented to a decentralised committee, which upon verification, associates a long\sphinxhyphen{}term secret key with the arbitrator’s identity and the soulbound token, ensuring transparency and confidentiality {[}\hyperlink{cite.SBT/SBT:id85}{UY22}{]}.


\subsubsection{Recording Employment History and Professional Qualifications}
\label{\detokenize{SBT/SBT:recording-employment-history-and-professional-qualifications}}
\sphinxAtStartPar
Soulbound Tokens (SBTs) can be utilised to record employment history and professional qualifications. Employers can distribute these tokens to reflect an employee’s work experience, project involvement, accomplishments, and other pertinent details. When seeking new employment opportunities or during job interviews, employees can present these SBTs. Thus, SBTs function as tangible evidence of professional skills and achievements {[}\hyperlink{cite.SBT/SBT:id90}{Tak23}{]}.


\subsubsection{Authenticating Academic Credentials}
\label{\detokenize{SBT/SBT:authenticating-academic-credentials}}
\sphinxAtStartPar
Linking Soulbound Tokens (SBTs) to detailed resumes, university degrees, certificates, and transcripts could be another practical use of this technology. Considering that such credentials are non\sphinxhyphen{}transferable in the traditional Web2 world, employers and educational institutions could utilise SBTs to authenticate the details provided by an applicant in their resume. Moreover, for reference verification, the addresses of the references could be incorporated into the SBT, facilitating on\sphinxhyphen{}chain attestations, and thus further streamlining the verification process {[}\hyperlink{cite.SBT/SBT:id81}{GCOGI23}{]}.


\subsubsection{Secure Management of Healthcare Records}
\label{\detokenize{SBT/SBT:secure-management-of-healthcare-records}}
\sphinxAtStartPar
In a patient\sphinxhyphen{}centric soulbound NFT framework for electronic health records (EHRs) to prevent the unauthorised trading of important medical documents, Soulbound Tokens (SBTs) can be employed. These tokens can’t be bought or transferred; once assigned, they remain tethered to your private wallet and identity. As such, they’re ideal for digitising non\sphinxhyphen{}transferable aspects like qualifications, reputation, and healthcare records. The ownership of the token bestows the right to control access to the information it contains, including the ability to revoke that access when required. Instead of being stored in a centralised database, personal information is managed in a blockchain\sphinxhyphen{}enabled format, providing enhanced access and control to the token’s owner {[}\hyperlink{cite.SBT/SBT:id84}{TT23}{]}. The ability to manage personal information in a blockchain\sphinxhyphen{}enabled form rather than having it stored in a central database makes SBTs an option for people who want the most access to their information {[}\hyperlink{cite.SBT/SBT:id89}{Mor23}{]}.


\subsection{Challenges and Concerns}
\label{\detokenize{SBT/SBT:challenges-and-concerns}}
\sphinxAtStartPar
Soulbound Tokens (SBTs), as an emerging concept, come with several challenges. Some of the notable concerns include {[}\hyperlink{cite.SBT/SBT:id88}{Lea22}{]}:
\begin{itemize}
\item {} 
\sphinxAtStartPar
\sphinxstylestrong{Privacy:} As SBTs are linked to a specific individual, they could potentially be used for tracking and monitoring that individual’s online activities. Technological advancements like zero\sphinxhyphen{}knowledge proofs on the blockchain could help address these privacy concerns by providing improved anonymity.

\item {} 
\sphinxAtStartPar
\sphinxstylestrong{Security:} If a user’s non\sphinxhyphen{}custodial wallet is compromised, malicious entities could misuse the SBTs, particularly those providing exclusive access or governance rights. This could harm the user and the communities they’re associated with. This issue can be mitigated by storing assets in secure custodial wallets or vaults.

\item {} 
\sphinxAtStartPar
\sphinxstylestrong{Interoperability:} Like many NFTs, SBTs are often minted on specific blockchains, which can restrict their versatility and applicability beyond their native chain. This limitation can be partially addressed by integrating EVM\sphinxhyphen{}compatible chains into prevalent Web3 applications and ensuring most users stay within a single blockchain ecosystem.

\item {} 
\sphinxAtStartPar
\sphinxstylestrong{Non\sphinxhyphen{}transferability:} The non\sphinxhyphen{}transferable nature of SBTs, while offering numerous benefits, can also pose challenges. If a token is unwillingly assigned to someone, it may lead to issues. This can be resolved by developing more robust permissioned interfaces on top of the blockchain, allowing users to enjoy the benefits of SBTs while also having the option to conceal or remove SBTs from their profile.

\end{itemize}

\sphinxAtStartPar
To ensure wider adoption and success, these issues associated with SBTs need to be ironed out. Although souls can choose to hide what SBTs reveal, in a way, they could also foster discrimination by revealing too many details in specific situations or contexts. This is particularly true for marginalized social groups who are more likely to experience disfavor {[}\hyperlink{cite.SBT/SBT:id87}{ShrishtiEth22}{]}(HackerNoon, CBDCs and soulbound token explained 2022). With the right solutions, non\sphinxhyphen{}transferable NFTs like SBTs have the potential to contribute to a more equitable and privacy\sphinxhyphen{}focused digital society.


\subsection{Conclusion}
\label{\detokenize{SBT/SBT:conclusion}}
\sphinxAtStartPar
In the quest to build a decentralised society, or DeSoc, Soulbound tokens (SBTs) serve as fundamental components. By creating a solid digital identity and provenance, they play an instrumental role in the growth of this new societal structure. The idea of a decentralised society might seem theoretical or abstract, yet it has numerous practical implications that are worth contemplating.

\sphinxAtStartPar
Soulbound tokens, in their diverse and wide\sphinxhyphen{}ranging applications, span the spectrum from web3 and DeFi to facets of everyday life. They are a rapidly emerging trend, destined not only to significantly influence the Web3 ecosystem but also to elevate the perception of NFTs. Rather than being seen merely as a means of owning artwork or symbols of prestige, NFTs can function as pivotal tools in the creation and confirmation of digital identities and connections in a decentralised world.




\subsection{References}
\label{\detokenize{SBT/SBT:references}}
\begin{sphinxthebibliography}{DugganWa}
\bibitem[BSP+22a]{GOV/gov:id31}
\sphinxAtStartPar
Tom Barbereau, Reilly Smethurst, Orestis Papageorgiou, Alexander Rieger, and Gilbert Fridgen. Defi, not so decentralized: the measured distribution of voting rights. \sphinxstyleemphasis{Hawaii International Conference on System Sciences (HICSS)}, 2022.
\bibitem[BSP+22b]{GOV/gov:id28}
\sphinxAtStartPar
Tom Barbereau, Reilly Smethurst, Orestis Papageorgiou, Johannes Sedlmeir, and Gilbert Fridgen. Decentralised finance's unregulated governance: minority rule in the digital wild west. \sphinxstyleemphasis{Available at SSRN}, 2022.
\bibitem[EAw22]{GOV/gov:id29}
\sphinxAtStartPar
Hassan Hamid Ekal and Shams N Abdul\sphinxhyphen{}wahab. Defi governance and decision\sphinxhyphen{}making on blockchain. \sphinxstyleemphasis{Mesopotamian Journal of Computer Science}, 2022:9–16, 2022.
\bibitem[JvWR21]{GOV/gov:id30}
\sphinxAtStartPar
Johannes Rude Jensen, Victor von Wachter, and Omri Ross. How decentralized is the governance of blockchain\sphinxhyphen{}based finance: empirical evidence from four governance token distributions. \sphinxstyleemphasis{arXiv preprint arXiv:2102.10096}, 2021.
\bibitem[KL22]{GOV/gov:id36}
\sphinxAtStartPar
Aggelos Kiayias and Philip Lazos. Sok: blockchain governance. \sphinxstyleemphasis{arXiv preprint arXiv:2201.07188}, 2022.
\bibitem[MKB22]{GOV/gov:id26}
\sphinxAtStartPar
Vijay Mohan, Peyman Khezr, and Chris Berg. Voting with time commitment for decentralized governance: bond voting as a sybil\sphinxhyphen{}resistant mechanism. \sphinxstyleemphasis{Available at SSRN}, 2022.
\bibitem[Nab23]{GOV/gov:id25}
\sphinxAtStartPar
Kelsie Nabben. Governance by algorithms, governance of algorithms: human\sphinxhyphen{}machine politics in decentralised autonomous organisations (daos). \sphinxstyleemphasis{puntOorg International Journal}, 8(1):36–54, 2023.
\bibitem[S+]{GOV/gov:id27}
\sphinxAtStartPar
K Stroponiati and others. Decentralized governance in defi: examples and pitfalls. squarespace. retrieved december 30, 2022.
\bibitem[XPFL23]{GOV/gov:id38}
\sphinxAtStartPar
Jiahua Xu, Daniel Perez, Yebo Feng, and Benjamin Livshits. Auto. gov: learning\sphinxhyphen{}based on\sphinxhyphen{}chain governance for decentralized finance (defi). \sphinxstyleemphasis{arXiv preprint arXiv:2302.09551}, 2023.
\bibitem[Bro22]{BBSecurity/bbsecurity:id100}
\sphinxAtStartPar
Ryan Browne. Hackers have stolen \$1.4 billion this year using crypto bridges. here’s why it’s happening, cnbc. \sphinxstyleemphasis{CNBC}, 2022. URL: \sphinxurl{https://www.cnbc.com/2022/08/10/hackers-have-stolen-1point4-billion-this-year-using-crypto-bridges.html}.
\bibitem[Cha22]{BBSecurity/bbsecurity:id88}
\sphinxAtStartPar
ChainAnalysis. Cross\sphinxhyphen{}chain bridge hacks emerge as top security risk, chainalysis. \sphinxstyleemphasis{ChainAnalysis}, 2022. URL: \sphinxurl{https://blog.chainalysis.com/reports/cross-chain-bridge-hacks-2022/}.
\bibitem[DDJ+18]{BBSecurity/bbsecurity:id89}
\sphinxAtStartPar
Donghui Ding, Tiantian Duan, Linpeng Jia, Kang Li, Zhongcheng Li, and Yi Sun. Interchain: a framework to support blockchain interoperability. \sphinxstyleemphasis{Second Asia\sphinxhyphen{}Pacific Work. Netw}, 2018.
\bibitem[F+20]{BBSecurity/bbsecurity:id93}
\sphinxAtStartPar
P Frauenthaler and others. Leveraging blockchain relays for cross\sphinxhyphen{}chain token transfers. 2020. \sphinxstyleemphasis{URL: https://www. dsg. tuwien. ac. at/projects/tast/pub/tast\sphinxhyphen{}white\sphinxhyphen{}paper\sphinxhyphen{}8. pdf. White Paper, Technische Universität Wien. Version}, 2020.
\bibitem[HLP19]{BBSecurity/bbsecurity:id95}
\sphinxAtStartPar
Thomas Hardjono, Alexander Lipton, and Alex Pentland. Toward an interoperability architecture for blockchain autonomous systems. \sphinxstyleemphasis{IEEE Transactions on Engineering Management}, 67(4):1298–1309, 2019.
\bibitem[KY22]{BBSecurity/bbsecurity:id101}
\sphinxAtStartPar
Sam Kessler and Sage D. Young. Ronin attack shows cross\sphinxhyphen{}chain crypto is a bridge too far, coindesk latest headlines. \sphinxstyleemphasis{CoinDesk}, 2022. URL: \sphinxurl{https://www.coindesk.com/layer2/2022/04/05/ronin-attack-shows-cross-chain-crypto-is-a-bridge-too-far/}.
\bibitem[KRDO17]{BBSecurity/bbsecurity:id91}
\sphinxAtStartPar
Aggelos Kiayias, Alexander Russell, Bernardo David, and Roman Oliynykov. Ouroboros: a provably secure proof\sphinxhyphen{}of\sphinxhyphen{}stake blockchain protocol. In \sphinxstyleemphasis{Advances in Cryptology–CRYPTO 2017: 37th Annual International Cryptology Conference, Santa Barbara, CA, USA, August 20–24, 2017, Proceedings, Part I}, 357–388. Springer, 2017.
\bibitem[LYY+23]{BBSecurity/bbsecurity:id92}
\sphinxAtStartPar
Taotao Li, Changlin Yang, Qinglin Yang, Siqi Zhou, Huawei Huang, and Zibin Zheng. Metaopera: a cross\sphinxhyphen{}metaverse interoperability protocol. \sphinxstyleemphasis{arXiv preprint arXiv:2302.01600}, 2023.
\bibitem[PBHouM22]{BBSecurity/bbsecurity:id99}
\sphinxAtStartPar
Babu Pillai, Kamanashis Biswas, Zhé Hóu, and Vallipuram Muthukkumarasamy. Cross\sphinxhyphen{}blockchain technology: integration framework and security assumptions. \sphinxstyleemphasis{IEEE Access}, 10:41239–41259, 2022.
\bibitem[Szt15]{BBSecurity/bbsecurity:id94}
\sphinxAtStartPar
Paul Sztorc. Drivechain\sphinxhyphen{}the simple two way peg. 2015.
\bibitem[WSK+22]{BBSecurity/bbsecurity:id96}
\sphinxAtStartPar
Xuechao Wang, Peiyao Sheng, Sreeram Kannan, Kartik Nayak, and Pramod Viswanath. Trustboost: boosting trust among interoperable blockchains. \sphinxstyleemphasis{arXiv preprint arXiv:2210.11571}, 2022.
\bibitem[XZC+22]{BBSecurity/bbsecurity:id97}
\sphinxAtStartPar
Tiancheng Xie, Jiaheng Zhang, Zerui Cheng, Fan Zhang, Yupeng Zhang, Yongzheng Jia, Dan Boneh, and Dawn Song. Zkbridge: trustless cross\sphinxhyphen{}chain bridges made practical. \sphinxstyleemphasis{arXiv preprint arXiv:2210.00264}, 2022.
\bibitem[ZHL+19]{BBSecurity/bbsecurity:id98}
\sphinxAtStartPar
Alexei Zamyatin, Dominik Harz, Joshua Lind, Panayiotis Panayiotou, Arthur Gervais, and William Knottenbelt. Xclaim: trustless, interoperable, cryptocurrency\sphinxhyphen{}backed assets. In \sphinxstyleemphasis{2019 IEEE Symposium on Security and Privacy (SP)}, 193–210. IEEE, 2019.
\bibitem[ZLZ20]{BBSecurity/bbsecurity:id90}
\sphinxAtStartPar
Jianbiao Zhang, Yanhui Liu, and Zhaoqian Zhang. Research on cross\sphinxhyphen{}chain technology architecture system based on blockchain. In \sphinxstyleemphasis{Communications, Signal Processing, and Systems: Proceedings of the 8th International Conference on Communications, Signal Processing, and Systems 8th}, 2609–2617. Springer, 2020.
\bibitem[Ali20]{MTP/mtp:id24}
\sphinxAtStartPar
Ahmed Ali. Blockchain technology and business use\sphinxhyphen{}cases for cost reduction. pages, 12 2020.
\bibitem[AIT23]{MTP/mtp:id27}
\sphinxAtStartPar
Asha Iyengar, Jeff Borsecnik and Team. Perform a remote wipe on a mobile phone. \sphinxstyleemphasis{Microsoft}, 2023. URL: \sphinxurl{https://learn.microsoft.com/en-us/exchange/clients/exchange-activesync/remote-wipe?view=exchserver-2019}.
\bibitem[Chi23]{MTP/mtp:id20}
\sphinxAtStartPar
Chirag. Blockchain: the technology revolutionizing mobile app security. \sphinxstyleemphasis{Appinventive}, 2023. URL: \sphinxurl{https://appinventiv.com/blog/blockchain-technology-revolutionizing-mobile-app-security/}.
\bibitem[DD21]{MTP/mtp:id22}
\sphinxAtStartPar
Utpal Biswas Debashis Das, Sourav Banerjee. A secure vehicle theft detection framework using blockchain and smart contract. \sphinxstyleemphasis{Springer}, 2021. URL: \sphinxurl{https://doi.org/10.1007/s12083-020-01022-0}.
\bibitem[For22]{MTP/mtp:id17}
\sphinxAtStartPar
Savannah Fortis. Samsung uses blockchain\sphinxhyphen{}based security for devices in its network. \sphinxstyleemphasis{Cointelegraph}, 2022. URL: \sphinxurl{https://cointelegraph.com/news/web3-protection-platform-introduces-improved-detection-mechanics-in-latest-update}.
\bibitem[Gob18]{MTP/mtp:id19}
\sphinxAtStartPar
Andreas Göbel. Using blockchain to prevent mobile phone theft. \sphinxstyleemphasis{Camelot}, 2018. URL: \sphinxurl{https://blog.camelot-group.com/2018/12/using-blockchain-to-prevent-mobile-phone-theft/}.
\bibitem[Hen22]{MTP/mtp:id16}
\sphinxAtStartPar
Beatriz Henriquez. Mobile theft and loss report \sphinxhyphen{} 2020/2021 edition. \sphinxstyleemphasis{PREY Project}, 2022. URL: \sphinxurl{https://preyproject.com/blog/mobile-theft-and-loss-report-2020-2021-edition}.
\bibitem[Hic22]{MTP/mtp:id25}
\sphinxAtStartPar
Jacob Hicks. How to block a stolen iphone with an imei number. \sphinxstyleemphasis{DeviceTests}, 2022. URL: \sphinxurl{https://devicetests.com/how-to-block-a-stolen-iphone-with-an-imei-number}.
\bibitem[Hom16]{MTP/mtp:id15}
\sphinxAtStartPar
Elaine J. Hom. Mobile device security: startling statistics on data loss and data breaches. \sphinxstyleemphasis{ChannelProNetwork}, 2016. URL: \sphinxurl{https://www.channelpronetwork.com/article/mobile-device-security-startling-statistics-data-loss-and-data-breaches}.
\bibitem[Hua18]{MTP/mtp:id18}
\sphinxAtStartPar
Huawei. Huawei blockchain whitepaper. \sphinxstyleemphasis{Huawei}, 2018. URL: \sphinxurl{https://www.huaweicloud.com/content/dam/cloudbu-site/archive/hk/en-us/about/analyst-reports/images/4-201804-Huawei\%20Blockchain\%20Whitepaper-en.pdf}.
\bibitem[Ire21]{MTP/mtp:id21}
\sphinxAtStartPar
Gwyneth Iredale. The rise of private blockchain technologies. \sphinxstyleemphasis{101 Blockchains}, 2021. URL: \sphinxurl{https://101blockchains.com/private-blockchain/}.
\bibitem[Mar23]{MTP/mtp:id28}
\sphinxAtStartPar
Karen Marcus. The 8 best phone tracker apps of 2023. \sphinxstyleemphasis{Lifewire}, 2023. URL: \sphinxurl{https://learn.microsoft.com/en-us/exchange/clients/exchange-activesync/remote-wipe?view=exchserver-2019}.
\bibitem[Ram21]{MTP/mtp:id23}
\sphinxAtStartPar
Murali Ramakrishnan. How blockchain works in cross\sphinxhyphen{}border payments. \sphinxstyleemphasis{Springer}, 2021. URL: \sphinxurl{https://blogs.oracle.com/financialservices/post/how-blockchain-works-in-cross-border-payments-}.
\bibitem[Tre15]{MTP/mtp:id26}
\sphinxAtStartPar
Mobile ICT Trends. Erasing your device, blocking your sim card: how to be prepared when your phone gets stolen. \sphinxstyleemphasis{econocom}, 2015. URL: \sphinxurl{https://blog.econocom.com/en/blog/what-to-do-if-your-mobile-device-gets-stolen-how-do-you-block-your-sim-card-heres-how-to-be-prepared-for-the-loss-or-theft-of-your-mobile/}.
\bibitem[Ban23a]{LEGACY/legacy:id93}
\sphinxAtStartPar
European Central Bank. Crypto\sphinxhyphen{}assets: a new standard for banks. \sphinxstyleemphasis{European Central Bank}, 2023. URL: \sphinxurl{https://www.bankingsupervision.europa.eu/press/publications/newsletter/2023/html/ssm.nl230215\_1.en.html}.
\bibitem[Ban23b]{LEGACY/legacy:id94}
\sphinxAtStartPar
European Central Bank. Take\sphinxhyphen{}aways from the horizontal assessment of the survey on digital transformation and the use of fintech. \sphinxstyleemphasis{European Central Bank}, 2023. URL: \sphinxurl{https://www.bankingsupervision.europa.eu/ecb/pub/pdf/Takeaways\_horizontal\_assessment~de65261ad0.en.pdf}.
\bibitem[Com20]{LEGACY/legacy:id96}
\sphinxAtStartPar
European Commission. Regulation of the european parliament and of the council. \sphinxstyleemphasis{European Commission}, 2020. URL: \sphinxurl{https://eur-lex.europa.eu/legal-content/EN/TXT/?uri=CELEX\%3A52020PC0593}.
\bibitem[Uni14]{LEGACY/legacy:id99}
\sphinxAtStartPar
European Union. Markets in financial instruments regulation (mifir). \sphinxstyleemphasis{European Union}, 2014. URL: \sphinxurl{https://eur-lex.europa.eu/EN/legal-content/summary/markets-in-financial-instruments-regulation-mifir.html}.
\bibitem[Uni15]{LEGACY/legacy:id98}
\sphinxAtStartPar
European Union. Anti\sphinxhyphen{}money laundering directive. \sphinxstyleemphasis{European Union}, 2015. URL: \sphinxurl{https://eur-lex.europa.eu/legal-content/EN/TXT/?uri=celex:32015L0849}.
\bibitem[Uni18a]{LEGACY/legacy:id95}
\sphinxAtStartPar
European Union. Directive of the european parliament. \sphinxstyleemphasis{European Union}, 2018. URL: \sphinxurl{https://eur-lex.europa.eu/legal-content/EN/TXT/PDF}.
\bibitem[Uni18b]{LEGACY/legacy:id97}
\sphinxAtStartPar
European Union. General data protection regulation. \sphinxstyleemphasis{European Union}, 2018. URL: \sphinxurl{https://gdpr-info.eu}.
\bibitem[DDL+19]{SHARDING/sharding:id61}
\sphinxAtStartPar
Hung Dang, Tien Tuan Anh Dinh, Dumitrel Loghin, Ee\sphinxhyphen{}Chien Chang, Qian Lin, and Beng Chin Ooi. Towards scaling blockchain systems via sharding. In \sphinxstyleemphasis{Proceedings of the 2019 international conference on management of data}, 123–140. 2019.
\bibitem[HHS22]{SHARDING/sharding:id70}
\sphinxAtStartPar
Abdelatif Hafid, Abdelhakim Senhaji Hafid, and Mustapha Samih. A tractable probabilistic approach to analyze sybil attacks in sharding\sphinxhyphen{}based blockchain protocols. \sphinxstyleemphasis{IEEE Transactions on Emerging Topics in Computing}, 2022.
\bibitem[Hed20]{SHARDING/sharding:id69}
\sphinxAtStartPar
Hedera. Hedera hashgraph whitepaper. \sphinxstyleemphasis{Hedera}, 2020. URL: \sphinxurl{https://hedera.com/hh\_whitepaper\_v2.1-20200815.pdf}.
\bibitem[HPZ+22]{SHARDING/sharding:id71}
\sphinxAtStartPar
Huawei Huang, Xiaowen Peng, Jianzhou Zhan, Shenyang Zhang, Yue Lin, Zibin Zheng, and Song Guo. Brokerchain: a cross\sphinxhyphen{}shard blockchain protocol for account/balance\sphinxhyphen{}based state sharding. In \sphinxstyleemphasis{IEEE INFOCOM 2022\sphinxhyphen{}IEEE Conference on Computer Communications}, 1968–1977. IEEE, 2022.
\bibitem[KKJG+18]{SHARDING/sharding:id62}
\sphinxAtStartPar
Eleftherios Kokoris\sphinxhyphen{}Kogias, Philipp Jovanovic, Linus Gasser, Nicolas Gailly, Ewa Syta, and Bryan Ford. Omniledger: a secure, scale\sphinxhyphen{}out, decentralized ledger via sharding. In \sphinxstyleemphasis{2018 IEEE Symposium on Security and Privacy (SP)}, 583–598. IEEE, 2018.
\bibitem[KTTI22]{SHARDING/sharding:id67}
\sphinxAtStartPar
Alexander Kudzin, Kentaroh Toyoda, Satoshi Takayama, and Atsushi Ishigame. Scaling ethereum 2.0 s cross\sphinxhyphen{}shard transactions with refined data structures. \sphinxstyleemphasis{Cryptography}, 6(4):57, 2022.
\bibitem[LNZ+16]{SHARDING/sharding:id64}
\sphinxAtStartPar
Loi Luu, Viswesh Narayanan, Chaodong Zheng, Kunal Baweja, Seth Gilbert, and Prateek Saxena. A secure sharding protocol for open blockchains. In \sphinxstyleemphasis{Proceedings of the 2016 ACM SIGSAC conference on computer and communications security}, 17–30. 2016.
\bibitem[Nea20a]{SHARDING/sharding:id65}
\sphinxAtStartPar
Near. Near nightshade whitepaper. \sphinxstyleemphasis{Near}, 2020. URL: \sphinxurl{https://near.org/papers/nightshade/}.
\bibitem[Nea20b]{SHARDING/sharding:id68}
\sphinxAtStartPar
Near. Near runtime spec. \sphinxstyleemphasis{Near}, 2020. URL: \sphinxurl{https://nomicon.io/RuntimeSpec/Receipts}.
\bibitem[TG22]{SHARDING/sharding:id72}
\sphinxAtStartPar
Deepal Tennakoon and Vincent Gramoli. Dynamic blockchain sharding. In \sphinxstyleemphasis{5th International Symposium on Foundations and Applications of Blockchain 2022 (FAB 2022)}. Schloss Dagstuhl\sphinxhyphen{}Leibniz\sphinxhyphen{}Zentrum für Informatik, 2022.
\bibitem[ZMR18]{SHARDING/sharding:id63}
\sphinxAtStartPar
Mahdi Zamani, Mahnush Movahedi, and Mariana Raykova. Rapidchain: scaling blockchain via full sharding. In \sphinxstyleemphasis{Proceedings of the 2018 ACM SIGSAC conference on computer and communications security}, 931–948. 2018.
\bibitem[All16]{SSI/ssi:id61}
\sphinxAtStartPar
Christopher Allen. The path to self\sphinxhyphen{}sovereign identity. \sphinxstyleemphasis{Life With Alacrity}, 2016. URL: \sphinxurl{http://www.lifewithalacrity.com/2016/04/the-path-to-self-soverereign-identity.html}.
\bibitem[Ame22]{SSI/ssi:id46}
\sphinxAtStartPar
New America. Three self\sphinxhyphen{}sovereign identity platforms to watch. \sphinxstyleemphasis{New America}, 2022. URL: \sphinxurl{https://www.newamerica.org/future-land-housing/reports/nail-finds-hammer/three-self-sovereign-identity-platforms-to-watch/}.
\bibitem[BCHR+19]{SSI/ssi:id52}
\sphinxAtStartPar
Jorge Bernal Bernabe, Jose Luis Canovas, Jose L Hernandez\sphinxhyphen{}Ramos, Rafael Torres Moreno, and Antonio Skarmeta. Privacy\sphinxhyphen{}preserving solutions for blockchain: review and challenges. \sphinxstyleemphasis{IEEE Access}, 7:164908–164940, 2019.
\bibitem[dVBSFCustodio22]{SSI/ssi:id44}
\sphinxAtStartPar
Mauricio de Vasconcelos Barros, Frederico Schardong, and Ricardo Felipe Custódio. Leveraging self\sphinxhyphen{}sovereign identity, blockchain, and zero\sphinxhyphen{}knowledge proof to build a privacy\sphinxhyphen{}preserving vaccination pass. \sphinxstyleemphasis{Blockchain, and Zero\sphinxhyphen{}Knowledge Proof to Build a Privacy\sphinxhyphen{}Preserving Vaccination Pass}, 2022.
\bibitem[FIP23]{SSI/ssi:id47}
\sphinxAtStartPar
Md Sadek Ferdous, Andrei Ionita, and Wolfgang Prinz. Ssi4web: a self\sphinxhyphen{}sovereign identity (ssi) framework for the web. In \sphinxstyleemphasis{Blockchain and Applications, 4th International Congress}, 366–379. Springer, 2023.
\bibitem[FKustersS16]{SSI/ssi:id51}
\sphinxAtStartPar
Daniel Fett, Ralf Küsters, and Guido Schmitz. A comprehensive formal security analysis of oauth 2.0. In \sphinxstyleemphasis{Proceedings of the 2016 ACM SIGSAC Conference on Computer and Communications Security}, 1204–1215. 2016.
\bibitem[JC23]{SSI/ssi:id58}
\sphinxAtStartPar
CLAIRE CASHER JULIA CLARK, ANNA DIOFASI. 850 million people globally don’t have id—why this matters and what we can do about it. \sphinxstyleemphasis{World Bank}, 2023. URL: \sphinxurl{https://blogs.worldbank.org/digital-development/850-million-people-globally-dont-have-id-why-matters-and-what-we-can-do-about}.
\bibitem[LC22]{SSI/ssi:id48}
\sphinxAtStartPar
Mary Lacity and Erran Carmel. Implementing self\sphinxhyphen{}sovereign identity (ssi) for a digital staff passport at uk nhs. \sphinxstyleemphasis{University of Arkansas}, 2022.
\bibitem[LB15]{SSI/ssi:id49}
\sphinxAtStartPar
Maryline Laurent and Samia Bouzefrane. \sphinxstyleemphasis{Digital identity management}. Elsevier, 2015.
\bibitem[RR06]{SSI/ssi:id50}
\sphinxAtStartPar
David Recordon and Drummond Reed. Openid 2.0: a platform for user\sphinxhyphen{}centric identity management. In \sphinxstyleemphasis{Proceedings of the second ACM workshop on Digital identity management}, 11–16. 2006.
\bibitem[SSFU22]{SSI/ssi:id45}
\sphinxAtStartPar
Vincent Schlatt, Johannes Sedlmeir, Simon Feulner, and Nils Urbach. Designing a framework for digital kyc processes built on blockchain\sphinxhyphen{}based self\sphinxhyphen{}sovereign identity. \sphinxstyleemphasis{Information \& Management}, 59(7):103553, 2022.
\bibitem[SS22]{SSI/ssi:id59}
\sphinxAtStartPar
Cyber Security and Society. Estonia leads world in making digital voting a reality. \sphinxstyleemphasis{Cyber Security and Society}, 2022. URL: \sphinxurl{https://www.ft.com/content/b4425338-6207-49a0-bbfb-6ae5460fc1c1}.
\bibitem[SRR22]{SSI/ssi:id60}
\sphinxAtStartPar
Isa Sertkaya, Peter Roenne, and Peter YA Ryan. Estonian internet voting with anonymous credentials. \sphinxstyleemphasis{Turkish Journal of Electrical Engineering and Computer Sciences}, 30(2):420–435, 2022.
\bibitem[SHU+22]{SSI/ssi:id57}
\sphinxAtStartPar
Mohammed Shuaib, Noor Hafizah Hassan, Sahnius Usman, Shadab Alam, Surbhi Bhatia, Arwa Mashat, Adarsh Kumar, and Manoj Kumar. Self\sphinxhyphen{}sovereign identity solution for blockchain\sphinxhyphen{}based land registry system: a comparison. \sphinxstyleemphasis{Mobile Information Systems}, 2022:1–17, 2022.
\bibitem[CGC+20]{SDE/ScamDetec:id139}
\sphinxAtStartPar
Weili Chen, Xiongfeng Guo, Zhiguang Chen, Zibin Zheng, Yutong Lu, and Yin Li. Honeypot contract risk warning on ethereum smart contracts. In \sphinxstyleemphasis{IEEE International Conference on Joint Cloud Computing}, volume, 1–8. Oxford, UK, Aug. 2020. IEEE.
\bibitem[CZC+18]{SDE/ScamDetec:id138}
\sphinxAtStartPar
Weili Chen, Zibin Zheng, Jiahui Cui, Edith Ngai, Peilin Zheng, and Yuren Zhou. Detecting ponzi schemes on ethereum: towards healthier blockchain technology. In \sphinxstyleemphasis{Proceedings of the ACM Web Conference}, volume, 1409–1418. Lyon, France, April 2018. ACM.
\bibitem[CLS+21]{SDE/ScamDetec:id136}
\sphinxAtStartPar
Weimin Chen, Xinran Li, Yuting Sui, Ningyu He, Haoyu Wang, Lei Wu, and Xiapu Luo. Sadponzi: detecting and characterizing ponzi schemes in ethereum smart contracts. In \sphinxstyleemphasis{Proceedings of the ACM on Measurement and Analysis of Computing Systems}, volume 5. New York, NY, USA, June 2021. ACM.
\bibitem[JLTGG19]{SDE/ScamDetec:id137}
\sphinxAtStartPar
Eunjin Jung, Marion Le Tilly, Ashish Gehani, and Yunjie Ge. Data mining\sphinxhyphen{}based ethereum fraud detection. In \sphinxstyleemphasis{IEEE International Conference on Blockchain}, volume, 266–273. Atlanta, USA, July 2019. IEEE.
\bibitem[LMMSS23]{SDE/ScamDetec:id142}
\sphinxAtStartPar
Massimo La Morgia, Alessandro Mei, Francesco Sassi, and Julinda Stefa. The doge of wall street: analysis and detection of pump and dump cryptocurrency manipulations. \sphinxstyleemphasis{ACM Transactions on Internet Technology}, Feb. 2023.
\bibitem[MAD22]{SDE/ScamDetec:id141}
\sphinxAtStartPar
Bruno Mazorra, Victor Adan, and Vanesa Daza. Do not rug on me: leveraging machine learning techniques for automated scam detection. \sphinxstyleemphasis{Mathematics}, 10(6):949, Mar. 2022.
\bibitem[Tea22]{SDE/ScamDetec:id144}
\sphinxAtStartPar
Chainalysis Team. The 2022 crypto crime report. Feb. 2022. URL: \sphinxurl{go.chainalysis.com/2021-crypto-crime-report}.
\bibitem[WYL+22]{SDE/ScamDetec:id134}
\sphinxAtStartPar
Jiajing Wu, Qi Yuan, Dan Lin, Wei You, Weili Chen, Chuan Chen, and Zibin Zheng. Who are the phishers? phishing scam detection on ethereum via network embedding. \sphinxstyleemphasis{IEEE Transactions on Systems, Man, and Cybernetics: Systems}, 52(2):1156–1166, Feb. 2022.
\bibitem[XWG+21]{SDE/ScamDetec:id140}
\sphinxAtStartPar
Pengcheng Xia, Haoyu Wang, Bingyu Gao, Weihang Su, Zhou Yu, Xiapu Luo, Chao Zhang, Xusheng Xiao, and Guoai Xu. Trade or trick? detecting and characterizing scam tokens on uniswap decentralized exchange. In \sphinxstyleemphasis{Proceedings of the ACM on Measurement and Analysis of Computing Systems}, volume 5, 1–26. New York, NY, USA, December 2021. ACM.
\bibitem[XL19]{SDE/ScamDetec:id143}
\sphinxAtStartPar
Jiahua Xu and Benjamin Livshits. The anatomy of a cryptocurrency Pump\sphinxhyphen{}and\sphinxhyphen{}Dump scheme. In \sphinxstyleemphasis{Proceedings of the 28th USENIX Conference on Security Symposium}, 1609–1625. Santa Clara, CA, Aug. 2019. USENIX Association.
\bibitem[YLW23]{SDE/ScamDetec:id135}
\sphinxAtStartPar
Jingjing Yang, Jieli Liu, and Jiajing Wu. With trail to follow: measurements of real\sphinxhyphen{}world non\sphinxhyphen{}fungible token phishing attacks on ethereum. \sphinxstyleemphasis{arXiv preprint arXiv:2307.01579}, 2023.
\bibitem[Hed]{HED/hed:id126}
\sphinxAtStartPar
What is hedera hashgraph. URL: \sphinxurl{https://hedera.com/learning/hedera-hashgraph/what-is-hedera-hashgraph}.
\bibitem[BGT19]{HED/hed:id123}
\sphinxAtStartPar
Leemon Baird, Bryan Gross, and Donald Thibeau. Hedera consensus service. Technical Report, Hedera Hashgraph, 2019. Technical Whitepaper. URL: \sphinxurl{https://hedera.com/hh-consensus-service-whitepaper.pdf}.
\bibitem[BHM20]{HED/hed:id128}
\sphinxAtStartPar
Leemon Baird, Mance Harmon, and Paul Madsen. Hedera: a public hashgraph network \& governing council. Technical Report, Hedera, August 2020. URL: \sphinxurl{https://hedera.com/hh\_whitepaper\_v2.1-20200815.pdf}.
\bibitem[Cla22]{HED/hed:id125}
\sphinxAtStartPar
Kadeem Clarke. Smart contracts, done smarter: hedera ecosystem overview. 3 2022. URL: \sphinxurl{https://medium.com/momentum6/smart-contracts-done-smarter-hedera-ecosystem-overview-93c79d69e855}.
\bibitem[HH20]{HED/hed:id124}
\sphinxAtStartPar
LLC Hedera Hashgraph. Tokenization on hedera. Technical Report, Hedera Hashgraph, LLC, 2020. Technical Whitepaper. URL: \sphinxurl{https://hedera.com/hh\_tokenization-whitepaper\_v2\_20210101.pdf}.
\bibitem[HH23]{HED/hed:id122}
\sphinxAtStartPar
LLC Hedera Hashgraph. Understanding decentralization of hedera hashgraph. Technical Report, Hedera Hashgraph, LLC, 3 2023. Technical Report, Version 1. URL: \sphinxurl{https://files.hedera.com/hh-decentralization-of-consensus.pdf}.
\bibitem[Won19]{HED/hed:id127}
\sphinxAtStartPar
Evelyn Wong. Hedera hashgraph services — part 2: file storage service. 3 2019. URL: \sphinxurl{https://medium.com/hashingsystems/hedera-hashgraph-services-part-2-file-storage-service-f37f4323b667}.
\bibitem[20223]{ARM/arm:id147}
\sphinxAtStartPar
Custodial vs non\sphinxhyphen{}custodial wallets. Feb. 2023. URL: \sphinxurl{https://crypto.com/university/custodial-vs-non-custodial-wallets}.
\bibitem[DugganWaynePowellFarran23]{ARM/arm:id146}
\sphinxAtStartPar
Duggan,Wayne and Powell,Farran. Crypto lending: earn money from your crypto holdings by wayne daggan and farran powell. Jun. 2023. URL: \sphinxurl{https://www.forbes.com/advisor/investing/cryptocurrency/crypto-lending/}.
\bibitem[Ltd21]{ARM/arm:id150}
\sphinxAtStartPar
Revolut Ltd. Annual report and financial statements. Dec. 2021. URL: \sphinxurl{https://assets.revolut.com/pdf/Revolut\_Ltd\_YE\_2021\_Annual\%20Report.pdf}.
\bibitem[NBMG16]{ARM/arm:id144}
\sphinxAtStartPar
Arvind Narayanan, Joseph Bonneau, Edward Felten, Andrew Miller, and Steven Goldfeder. \sphinxstyleemphasis{Bitcoin and Cryptocurrency Technologies: A Comprehensive Introduction}. Princeton University Press, 2016.
\bibitem[Pan23]{ARM/arm:id145}
\sphinxAtStartPar
Fabio Panetta. Paradise lost? how crypto failed to deliver on its promises and what to do about it. Jun. 2023. URL: \sphinxurl{https://www.ecb.europa.eu/press/key/date/2023/html/ecb.sp230623\_1~80751450e6.en.html}.
\bibitem[Pec23]{ARM/arm:id151}
\sphinxAtStartPar
Marcel Pechman. How do the fed’s interest rates impact the crypto market? Mar. 2023. URL: \sphinxurl{https://cointelegraph.com/news/how-do-the-fed-s-interest-rates-impact-the-crypto-market}.
\bibitem[Ser23]{ARM/arm:id148}
\sphinxAtStartPar
Andrey Sergeenkov. How to manage risk when trading cryptocurrency. May 2023. URL: \sphinxurl{https://www.coindesk.com/learn/how-to-manage-risk-when-trading-cryptocurrency/}.
\bibitem[Tan23]{ARM/arm:id149}
\sphinxAtStartPar
Lynette Tan. Managing crypto investment risks. Jun. 2023. URL: \sphinxurl{https://www.dbs.com.sg/personal/articles/nav/investing/crypto-risk}.
\bibitem[CG22]{SBT/SBT:id80}
\sphinxAtStartPar
Tomer Jordi Chaffer and Justin Goldston. On the existential basis of self\sphinxhyphen{}sovereign identity and soulbound tokens: an examination of the “self” in the age of web3. \sphinxstyleemphasis{Journal of Strategic Innovation and Sustainability Vol}, 17(3):1, 2022.
\bibitem[GCOGI23]{SBT/SBT:id81}
\sphinxAtStartPar
Justin Goldston, Tomer Jordi Chaffer, Justyna Osowska, and Charles von Goins II. Digital inheritance in web3: a case study of soulbound tokens and the social recovery pallet within the polkadot and kusama ecosystems. \sphinxstyleemphasis{arXiv preprint arXiv:2301.11074}, 2023.
\bibitem[Hil22]{SBT/SBT:id82}
\sphinxAtStartPar
Felix Hildebrandt. The future of soulbound tokens and their blockchain accounts. In \sphinxstyleemphasis{Konferenzband zum Scientific Track der Blockchain Autumn School 2022}, number 2, 18–24. Hochschule Mittweida, 2022.
\bibitem[Lea22]{SBT/SBT:id88}
\sphinxAtStartPar
Juan Leal. What are soulbound tokens? \sphinxstyleemphasis{thirdweb}, 2022. URL: \sphinxurl{https://blog.thirdweb.com/soulbound-tokens/\#:~:text=Limitations\%20of\%20soulbound\%20tokens\%2C\%20or\%20non\%2Dtransferable\%20NFTs\&text=Currently\%2C\%20soulbound\%20tokens\%20lack\%20the,monitor\%20that\%20person\%27s\%20online\%20activities}.
\bibitem[Mor23]{SBT/SBT:id89}
\sphinxAtStartPar
Kirsty Moreland. What is a soulbound token? \sphinxstyleemphasis{Ledger Academy}, 2023. URL: \sphinxurl{https://www.ledger.com/academy/topics/blockchain/what-is-a-soulbound-token}.
\bibitem[SKSK23]{SBT/SBT:id83}
\sphinxAtStartPar
Sanskar Sharma, Aryan Kumar, Nidhi Sengar, and Ajay Kumar Kaushik. Implementation of property rental website using blockchain with soulbound tokens for reputation and review system. \sphinxstyleemphasis{ceur\sphinxhyphen{}ws.org}, 2023.
\bibitem[ShrishtiEth22]{SBT/SBT:id87}
\sphinxAtStartPar
Shrishti.Eth. Cbdcs and soulbound token explained. \sphinxstyleemphasis{HackerNoon}, 2022. URL: \sphinxurl{https://hackernoon.com/cbdcs-and-soulbound-token-explained}.
\bibitem[Tak23]{SBT/SBT:id90}
\sphinxAtStartPar
Akash Takyar. What are soulbound tokens, and how do they work? \sphinxstyleemphasis{LeewayHertz}, 2023. URL: \sphinxurl{https://www.leewayhertz.com/soulbound-tokens/}.
\bibitem[TT23]{SBT/SBT:id84}
\sphinxAtStartPar
Namrta Tanwar and Jawahar Thakur. Patient\sphinxhyphen{}centric soulbound nft framework for electronic health record (ehr). \sphinxstyleemphasis{Journal of Engineering and Applied Science}, 70(1):33, 2023.
\bibitem[Tit22]{SBT/SBT:id91}
\sphinxAtStartPar
Olusegun Joel Titus. Decentralized society's next stage: the soulbound token. \sphinxstyleemphasis{HackerNoon}, 2022. URL: \sphinxurl{https://hackernoon.com/cbdcs-and-soulbound-token-explained}.
\bibitem[UY22]{SBT/SBT:id85}
\sphinxAtStartPar
Ece Su Ustun and Melek Yuce. Smart legal contracts \& smarter dispute resolution. In \sphinxstyleemphasis{2022 IEEE 24th Conference on Business Informatics (CBI)}, volume 2, 111–117. IEEE, 2022.
\bibitem[WOB22]{SBT/SBT:id86}
\sphinxAtStartPar
E Glen Weyl, Puja Ohlhaver, and Vitalik Buterin. Decentralized society: finding web3's soul. \sphinxstyleemphasis{Available at SSRN 4105763}, 2022.
\end{sphinxthebibliography}







\renewcommand{\indexname}{Index}
\printindex
\end{document}